\documentclass[12pt]{amsart}
\usepackage[margin=1in]{geometry}
\usepackage{amssymb,amsfonts,amsmath}
\usepackage{color}
\usepackage{enumerate}
\usepackage{mathrsfs}
\usepackage{hyperref}
\usepackage[capitalise]{cleveref}
\usepackage{constants}
\usepackage{parskip}
\usepackage{indentfirst}
\usepackage{amsmath}
\usepackage{enumitem}
\setlength{\parindent}{2em}
\hfuzz=200pt

%----Table of Contents-----

%----Theorem Environments----
\newtheorem{theorem}{Theorem}[section]
\newtheorem{corollary}[theorem]{Corollary}
\newtheorem{hypothesis}[theorem]{Hypothesis}
\newtheorem{proposition}[theorem]{Proposition}
\newtheorem{lemma}[theorem]{Lemma}

\newtheorem{problem*}{Problem}

\theoremstyle{definition}
\newtheorem{definition}[theorem]{Definition}
\newtheorem{example}[theorem]{Example}
\newcommand{\exercise}[1]{\noindent {\bf Exercise #1.}}

\numberwithin{equation}{section}


\crefname{figure}{Figure}{Figures}
%MATH ENVIRONMENTS
\theoremstyle{plain}
\newtheorem*{theorem*}{Theorem}
\crefname{theorem}{Theorem}{Theorems}
\crefname{cor}{Corollary}{Corollaries}
\crefname{exercise}{Exercise}{Exercises}
\newtheorem*{cor*}{Corollary}
\crefname{cor*}{Corollary}{Corollaries}
\crefname{lem}{Lemma}{Lemmas}
\crefname{prop}{Proposition}{Propositions}
\crefname{conj}{Conjecture}{Conjectures}
\newtheorem*{conj*}{Conjecture}
\crefname{conj*}{Conjecture}{Conjectures}
\crefname{defn}{Definition}{Definitions}
\crefname{hyp}{Hypothesis}{Hypotheses}


\newcommand{\Z}{\mathbb{Z}}
\renewcommand{\C}{\mathbb{C}}
\newcommand{\R}{\mathbb{R}}
\newcommand{\Q}{\mathbb{Q}}
\newcommand{\F}{\mathbb{F}}
\newcommand{\N}{\mathbb{N}}
\newcommand{\re}{\textup{Re}}
\newcommand{\im}{\textup{Im}}
\renewcommand{\epsilon}{\varepsilon}
\newcommand{\Li}{\mathrm{Li}}


\title{Math 417, Homework 9}
\author{Charles Ancel}

%%%%%%%%%%%%%%%%%%%%%%%%%%%%%%%%%%%%%%%%%%%%%%%%%%%%%%%%%%%%%%%%%%%%%
%%%%%%%%%%%%%%%%%%%%%%%%%%%%%%%%%%%%%%%%%%%%%%%%%%%%%%%%%%%%%%%%%%%%%
%%%%%%%%%%%%%%%%%%%%%%%%%%%%%%%%%%%%%%%%%%%%%%%%%%%%%%%%%%%%%%%%%%%%%
\begin{document}
\maketitle
\section*{Chapter IV.19}
\begin{exercise}{3} Find all solutions of the equation $x^2 + 2x + 2 = 0$ in $\Z_6$.


    Alright, to solve the equation \( x^2 + 2x + 2 = 0 \) in \( \Z_6 \), we need to test all possible values of \( x \) in \( \Z_6 \) (which are \( 0, 1, 2, 3, 4, \) and \( 5 \)) and see which ones satisfy the equation.

    First, let's rewrite the equation:
    
    \[ x^2 + 2x + 2 \equiv 0 \mod 6 \]
    
    Now, let's plug in each possible value of \( x \) and see if it satisfies the equation:\\
\begin{enumerate}
    \item  \( x = 0 \): \( 0^2 + 2(0) + 2 = 2 \equiv 0 \mod 6 \)
    \item  \( x = 1 \): \( 1^2 + 2(1) + 2 = 5 \equiv 0 \mod 6 \)
    \item  \( x = 2 \): \( 2^2 + 2(2) + 2 = 10 = 4 \equiv 0 \mod 6 \)
    \item  \( x = 3 \): \( 3^2 + 2(3) + 2 = 17 = 5 \equiv 0 \mod 6 \)
    \item  \( x = 4 \): \( 4^2 + 2(4) + 2 = 26 = 2 \equiv 0 \mod 6 \)
    \item  \( x = 5 \): \( 5^2 + 2(5) + 2 = 37 = 1 \equiv 0 \mod 6 \)
\end{enumerate}
    
    \begin{proof}
    To show this, we substitute each element of \( \Z_6 \) into the equation and find that none of them satisfy the equation:\\
    \begin{enumerate}
        \item  \( x = 0 \): \( 0^2 + 2(0) + 2 \not\equiv 0 \mod 6 \)
        \item  \( x = 1 \): \( 1^2 + 2(1) + 2 \not\equiv 0 \mod 6 \)
        \item  \( x = 2 \): \( 2^2 + 2(2) + 2 \not\equiv 0 \mod 6 \)
        \item  \( x = 3 \): \( 3^2 + 2(3) + 2 \not\equiv 0 \mod 6 \)
        \item  \( x = 4 \): \( 4^2 + 2(4) + 2 \not\equiv 0 \mod 6 \)
        \item  \( x = 5 \): \( 5^2 + 2(5) + 2 \not\equiv 0 \mod 6 \)
    \end{enumerate}
    
    Hence, there are no solutions of the equation \(x^2 + 2x + 2 = 0\) in \( \Z_6 \).
    \end{proof}
\end{exercise}
\vspace*{20pt}
\begin{exercise}{9} Find the characteristic of the given ring. \[ \Z_3 \times \Z_4\]
    Given the ring \( \Z_3 \times \Z_4 \), the elements are ordered pairs of the form \((a, b)\) where \( a \) is an element of \( \Z_3 \) and \( b \) is an element of \( \Z_4 \). The multiplicative identity in this ring is \( (1, 1) \).
    
    To find the characteristic, we need to find the smallest positive integer \( n \) such that:
    
    \[ n \cdot (1, 1) = (n \mod 3, n \mod 4) = (0, 0) \]
    
    This will occur when \( n \) is a multiple of both 3 and 4, which is the least common multiple (LCM) of 3 and 4.
    The least common multiple (LCM) of 3 and 4 is 12.
    
    \begin{proof}
    To find the characteristic of the ring \( \Z_3 \times \Z_4 \), we need to find the smallest positive integer \( n \) such that:
    
    \[ n \cdot (1, 1) = (n \mod 3, n \mod 4) = (0, 0) \]
    
    Given that \( n \) needs to be a multiple of both 3 and 4, the smallest such value is the LCM of 3 and 4, which is 12.
    
    Thus, the characteristic of the ring \( \Z_3 \times \Z_4 \) is 12.
    \end{proof}
\end{exercise}
\vspace*{20pt}
\begin{exercise}{17f} True or false:\\
    \[\text{f. Every integral domain of characteristic $0$ is infinite.}\]    
    \begin{proof}
    Recall that the characteristic of a ring is the smallest positive integer \( n \) such that \( n \cdot 1 = 0 \) in that ring. If no such \( n \) exists, then the ring has characteristic \(0\).
    
    If an integral domain has characteristic \(0\), then no positive integer \( n \) exists such that \( n \cdot 1 = 0 \). This means that for every positive integer \( n \), the element \( n \cdot 1 \) is distinct from zero and from any other integer \( m \cdot 1 \) where \( m \neq n \). Therefore, there are infinitely many distinct elements in the ring, making the ring infinite.
    
    Thus, the statement is \textbf{True}.
    \end{proof}
    
\end{exercise}
\vspace*{20pt}
\begin{exercise}{17g} True or false:\\
    \[\text{g. The direct product of two integral domains is again an integral domain.}\]
    \begin{proof}
    Recall the definition of an integral domain: An integral domain is a commutative ring with unity (1) and no zero divisors. 
        
    Let's consider two integral domains, \( D_1 \) and \( D_2 \). Their direct product, \( D_1 \times D_2 \), consists of ordered pairs \((a, b)\) where \( a \) is from \( D_1 \) and \( b \) is from \( D_2 \).
        
    Now, let's take two non-zero elements from \( D_1 \times D_2 \): \((a_1, b_1)\) and \((a_2, b_2)\). The product of these elements is:
    \[(a_1, b_1) \cdot (a_2, b_2) = (a_1 \cdot a_2, b_1 \cdot b_2)\]
        
    For this product to be the zero element of \( D_1 \times D_2 \), i.e., \((0, 0)\), both \( a_1 \cdot a_2 \) and \( b_1 \cdot b_2 \) must be zero. However, since \( D_1 \) and \( D_2 \) are integral domains, this can only happen if \( a_1 = 0 \) or \( a_2 = 0 \) and \( b_1 = 0 \) or \( b_2 = 0 \). But this contradicts our assumption that both elements are non-zero.
        
    Therefore, \( D_1 \times D_2 \) does have zero divisors and is not an integral domain.
        
    Thus, the statement is \textbf{False}.
    \end{proof}
\end{exercise}
\vspace*{20pt}
\begin{exercise}{23} An element a of a ring $R$ is \textbf{idempotent} if $a^2 = a$. Show that a division ring contains exactly two idempotent elements.

    \begin{proof}
    Recall that a division ring is a ring in which every non-zero element has a multiplicative inverse. 
    
    Firstly, the element \( 0 \) is trivially idempotent because \( 0^2 = 0 \).
    
    Now, let's consider a non-zero idempotent element \( a \) in the division ring. Since \( a^2 = a \), we can factor out \( a \) to get:
    \[ a(a - 1) = 0 \]
    
    Now, since our ring is a division ring, no non-zero element is a zero divisor. This means that if \( ab = 0 \), then either \( a = 0 \) or \( b = 0 \). 
    
    From the above equation \( a(a - 1) = 0 \), it follows that either \( a = 0 \) or \( a - 1 = 0 \). We already know that \( a \) is non-zero, so the only possibility is \( a - 1 = 0 \), or \( a = 1 \).
    
    Thus, the element \( 1 \) is also idempotent because \( 1^2 = 1 \).
    
    No other element in the division ring can be idempotent because if there was another idempotent element \( b \), such that \( b^2 = b \), by the same reasoning as above, \( b \) would either have to be \( 0 \) or \( 1 \), which are the idempotents we already found.
    
    Therefore, a division ring contains exactly two idempotent elements: \( 0 \) and \( 1 \).
    
    \end{proof}
\end{exercise}
\vspace*{40pt}
\begin{exercise}{29} Show that the characteristic of an integral domain $D$ must be either 0 or a prime $p$. [Hint: If the characteristic
    of $D$ is $mn$, consider $(m \cdot 1)(n \cdot 1)$ in $D$.]

    \begin{proof}
    Let's denote the characteristic of \( D \) as \( n \). 
    
    If \( n = 0 \), then the statement holds, and we are done. 
    
    If \( n \neq 0 \), then it's either prime or composite. Let's consider the case where \( n \) is composite. This means \( n \) can be expressed as the product of two smaller positive integers \( m \) and \( n \) (neither being 1). 
    
    Consider the product:
    \[ (m \cdot 1)(n \cdot 1) \]
    
    Since the characteristic of \( D \) is \( n \), we have:
    \[ m \cdot 1 = m \text{ mod } n \]
    \[ n \cdot 1 = n \text{ mod } n \]
    
    Thus, the product becomes:
    \[ (m \cdot 1)(n \cdot 1) = mn \text{ mod } n = 0 \]
    
    But since \( m, n < n \) and neither \( m \) nor \( n \) are 1, neither \( m \cdot 1 \) nor \( n \cdot 1 \) are zero in the ring. 
    
    So, we have two non-zero elements in \( D \) whose product is zero. This means that \( D \) has zero divisors, which is a contradiction because an integral domain cannot have zero divisors.
    
    Thus, \( n \) cannot be composite. The only positive integers that are not composite and not equal to 1 are prime numbers.
    
    Therefore, the characteristic of an integral domain \( D \) must be either 0 or a prime \( p \).
    
    \end{proof}
\end{exercise}
\vspace*{60pt}
\section*{Chapter IV.21}
\begin{exercise}{2} Describe (in the sense of Exercise 1) the field $F$ of quotients of the integral subdomain $D =\{n +m
    \sqrt(2) | n,m \in \Z\} \text{ of } R$.
   
\begin{proof}
Firstly, recall the context from Exercise 1 for the Gaussian integers. The field of quotients for the subdomain \( D' = \{ n + mi \mid n, m \in \Z \} \) in \( \C \) consists of all ratios of the form:

\[
\frac{n_1 + m_1i}{n_2 + m_2i}
\]

where \( n_1, m_1, n_2, m_2 \) are integers and \( n_2 + m_2i \neq 0 \).

Similarly, the field \( $F$ \) of quotients for our given integral subdomain \( D \) will consist of all ratios of the form:

\[
\frac{n_1 + m_1 \sqrt{2}}{n_2 + m_2 \sqrt{2}}
\]

where \( n_1, m_1, n_2, m_2 \) are integers and \( n_2 + m_2 \sqrt{2} \neq 0 \). 

This ratio can be simplified by multiplying both the numerator and the denominator by the conjugate of the denominator:

\[
\frac{n_1 + m_1 \sqrt{2}}{n_2 + m_2 \sqrt{2}} \cdot \frac{n_2 - m_2 \sqrt{2}}{n_2 - m_2 \sqrt{2}}
\]

This simplification results in a ratio where the denominator no longer contains \( \sqrt{2} \). The simplified form represents the elements of the field of quotients \( $F$ \) for the integral domain \( D \).

Thus, \( $F$ \) is the set of all numbers of the form \( \frac{a + b \sqrt{2}}{c} \) where \( a, b, \) and \( c \) are integers, and \( c \neq 0 \).
\end{proof}
\end{exercise}
\vspace*{20pt}
\begin{exercise}{4} Mark each of the following true or false. \\
\begin{enumerate}[label=(\alph*.)]
    \item $\Q$ is a field of quotients of $\Z$.
    \item $\R$ is a field of quotients of $\Z$.
    \item $\R$ is a field of quotients of $\R$.
    \item $\C$ is a field of quotients of $\R$.
    \item If $D$ is a field, then any field of quotients of $D$ is isomorphic to $D$.
    \item The fact that D has no divisors of 0 was used strongly several times in the construction of a field $F$ of quotients of the integral domain $D$.
    \item Every element of an integral domain $D$ is a unit in a field $F$ of quotients of $D$.
    \item Every nonzero element of an integral domain $D$ is a unit in a field $F$ of quotients of $D$.
    \item A field of quotients $F$ of a subdomain $D\prime$ of an integral domain $D\prime$ can be regarded as a subfield of some field of quotients of $D$.
    \item Every field of quotients of $\Z$ is isomorphic to $\Q$.
\end{enumerate}
\begin{proof}

(a) \(\Q\) is a field of quotients of \(\Z\).\\
\textbf{True.} The rational numbers \(\Q\) are indeed constructed as quotients of integers. Every element in \(\Q\) can be expressed as a ratio of two integers.

(b) \(\R\) is a field of quotients of \(\Z\). \\
\textbf{False.} While \(\Q\) (the field of quotients of \(\Z\)) is a subset of \(\R\), not all real numbers can be expressed as a ratio of two integers.

(c) \(\R\) is a field of quotients of \(\R\). \\
\textbf{True.} Any field is a field of quotients of itself.

(d) \(\C\) is a field of quotients of \(\R\). \\
\textbf{False.} The complex numbers extend the real numbers by including imaginary numbers, which cannot be constructed merely as quotients of real numbers.

(e) If \(D\) is a field, then any field of quotients of \(D\) is isomorphic to \(D\). \\
\textbf{True.} A field is already a field of quotients of itself, so any field of quotients constructed from it would be isomorphic to the original field.

(f) The fact that \(D\) has no divisors of 0 was used strongly several times in the construction of a field \(F\) of quotients of the integral domain \(D\). \\
\textbf{True.} The absence of zero divisors is a crucial property of integral domains, and this property is essential when constructing a field of quotients.

(g) Every element of an integral domain \(D\) is a unit in a field \(F\) of quotients of \(D\). \\
\textbf{False.} Not every element of \(D\) is a unit in \(F\). Only the non-zero elements of \(D\) become units in \(F\), since they will have multiplicative inverses in \(F\).

(h) Every nonzero element of an integral domain \(D\) is a unit in a field \(F\) of quotients of \(D\). \\
\textbf{True.} This is because in the field of quotients, every non-zero element of \(D\) can be expressed as a ratio, and thus will have a multiplicative inverse.

(i) A field of quotients \(F\) of a subdomain \(D'\) of an integral domain \(D'\) can be regarded as a subfield of some field of quotients of \(D\). \\
\textbf{True.} If \(D'\) is a subdomain of \(D\), then the field of quotients of \(D'\) will naturally be a subfield of the field of quotients of \(D\).

(j) Every field of quotients of \(\Z\) is isomorphic to \(\Q\). \\
\textbf{True.} The field of quotients of \(\Z\) is, by definition, the set of rational numbers \(\Q\). Any other field of quotients constructed from \(\Z\) would be isomorphic to \(\Q\).

\end{proof}
\end{exercise}
\end{document}



































