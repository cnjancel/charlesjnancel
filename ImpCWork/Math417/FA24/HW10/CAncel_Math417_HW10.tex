\documentclass[12pt]{amsart}
\usepackage[margin=1in]{geometry}
\usepackage{amssymb,amsfonts,amsmath}
\usepackage{color}
\usepackage{enumerate}
\usepackage{mathrsfs}
\usepackage{hyperref}
\usepackage[capitalise]{cleveref}
\usepackage{constants}
\usepackage{parskip}
\usepackage{indentfirst}
\usepackage{amsmath}
\usepackage{enumitem}
\setlength{\parindent}{2em}
\hfuzz=200pt

%----Table of Contents-----

%----Theorem Environments----
\newtheorem{theorem}{Theorem}[section]
\newtheorem{corollary}[theorem]{Corollary}
\newtheorem{hypothesis}[theorem]{Hypothesis}
\newtheorem{proposition}[theorem]{Proposition}
\newtheorem{lemma}[theorem]{Lemma}

\newtheorem{problem*}{Problem}

\theoremstyle{definition}
\newtheorem{definition}[theorem]{Definition}
\newtheorem{example}[theorem]{Example}
\newcommand{\exercise}[1]{\noindent {\bf Exercise #1.}}

\numberwithin{equation}{section}


\crefname{figure}{Figure}{Figures}
%MATH ENVIRONMENTS
\theoremstyle{plain}
\newtheorem*{theorem*}{Theorem}
\crefname{theorem}{Theorem}{Theorems}
\crefname{cor}{Corollary}{Corollaries}
\crefname{exercise}{Exercise}{Exercises}
\newtheorem*{cor*}{Corollary}
\crefname{cor*}{Corollary}{Corollaries}
\crefname{lem}{Lemma}{Lemmas}
\crefname{prop}{Proposition}{Propositions}
\crefname{conj}{Conjecture}{Conjectures}
\newtheorem*{conj*}{Conjecture}
\crefname{conj*}{Conjecture}{Conjectures}
\crefname{defn}{Definition}{Definitions}
\crefname{hyp}{Hypothesis}{Hypotheses}


\newcommand{\Z}{\mathbb{Z}}
\renewcommand{\C}{\mathbb{C}}
\newcommand{\R}{\mathbb{R}}
\newcommand{\Q}{\mathbb{Q}}
\newcommand{\F}{\mathbb{F}}
\newcommand{\N}{\mathbb{N}}
\newcommand{\re}{\textup{Re}}
\newcommand{\im}{\textup{Im}}
\renewcommand{\epsilon}{\varepsilon}
\newcommand{\Li}{\mathrm{Li}}


\title{Math 417, Homework 10}
\author{Charles Ancel}

%%%%%%%%%%%%%%%%%%%%%%%%%%%%%%%%%%%%%%%%%%%%%%%%%%%%%%%%%%%%%%%%%%%%%
%%%%%%%%%%%%%%%%%%%%%%%%%%%%%%%%%%%%%%%%%%%%%%%%%%%%%%%%%%%%%%%%%%%%%
%%%%%%%%%%%%%%%%%%%%%%%%%%%%%%%%%%%%%%%%%%%%%%%%%%%%%%%%%%%%%%%%%%%%%
\begin{document}
\maketitle
\section*{Chapter IV.22}
\begin{exercise}{4} Find the sum and the product of the given polynomials in the given polynomial ring.
    \[f(x) = 2x^3 + 4x^2 + 3x + 2 , \  g(x) = 3x^4 + 2x + 4 \text{ in }\Z_5[x].\]

\begin{proof}
\textbf{For the Sum:}  
The sum \( h(x) \) is found by adding the polynomials term by term:
\begin{align*}
h(x) &= f(x) + g(x) \\
&= (2x^3 + 4x^2 + 3x + 2) + (3x^4 + 2x + 4) \\
&= 3x^4 + 2x^3 + 4x^2 + 5x + 6.
\end{align*}
However, since we're in \( \Z_5[x] \), we can reduce the coefficients modulo 5. This gives:
\[ h(x) = 3x^4 + 2x^3 + 4x^2 + x. \]

\textbf{For the Product:} 
The product \( p(x) \) is found by multiplying each term of \( f(x) \) with each term of \( g(x) \):
\begin{align*}
p(x) &= f(x)g(x) \\
&= (2x^3 + 4x^2 + 3x + 2)(3x^4 + 2x + 4) \\
&= 6x^7 + 12x^6 + 9x^5 + 10x^4 + 16x^3 + 22x^2 + 16x + 8.
\end{align*}
Again, we need to reduce the coefficients modulo 5 to get the polynomial in \( \Z_5[x] \):
\[ p(x) = x^7 + 2x^6 + 4x^5 + x^3 + 2x^2 + x + 3. \]

In conclusion, the sum and product in \( \Z_5[x] \) are:
\[ f(x) + g(x) = 3x^4 + 2x^3 + 4x^2 + x \]
\[ f(x)g(x) = x^7 + 2x^6 + 4x^5 + x^3 + 2x^2 + x + 3. \]
\end{proof}

The provided solution is well-structured and accurate.
    
\end{exercise}
\vspace*{20pt}
\begin{exercise}{6} How many polynomials are there of degree $\leq 2\text{ in } \Z_5[x]$? (Include 0.)

\begin{proof}
A polynomial of degree \( \leq 2 \) in \( \Z_5[x] \) has the general form:
\[ ax^2 + bx + c \]
where \( a \), \( b \), and \( c \) are coefficients from \( \Z_5 \) and can take on any value from the set \( \{0, 1, 2, 3, 4\} \).

1. For the coefficient \( a \) (which is the coefficient of \( x^2 \)):  
   Since we are considering polynomials up to and including degree 2, \( a \) can be 0 (for degree 0 or 1 polynomials) or any value between 1 and 4 (for degree 2 polynomials). Thus, there are 5 possibilities for \( a \).

2. For the coefficient \( b \) (which is the coefficient of \( x \)):  
   It can take on any value between 0 and 4, inclusive, regardless of the value of \( a \). Thus, there are 5 possibilities for \( b \).

3. For the constant term \( c \):  
   It can take on any value between 0 and 4, inclusive, regardless of the values of \( a \) and \( b \). Thus, there are 5 possibilities for \( c \).

Given these possibilities for each coefficient, the total number of polynomials of degree \( \leq 2 \) is:
\[ \text{Total polynomials} = 5 \times 5 \times 5 = 125. \]

Therefore, there are 125 polynomials of degree \( \leq 2 \) in \( \Z_5[x] \), which includes the polynomial 0.
\end{proof}
\end{exercise}
\vspace*{20pt}
\begin{exercise}{13} Find all zeros in the indicated finite field of the given polynomial with coefficients in
    that field. [Hint: One way is simply to try all candidates!] \[x^3 + 2x + 2\text{ in }\Z_7\]

    Certainly! Let's provide a detailed breakdown of the computation in LaTeX format.

\begin{proof}
To find the zeros of \( f(x) \) in \( \Z_7 \), we evaluate \( f(x) \) for each element of \( \Z_7 \):

1. For \( x = 0 \):
\[ f(0) = 0^3 + 2(0) + 2 = 2 \]

2. For \( x = 1 \):
\[ f(1) = 1^3 + 2(1) + 2 = 5 \]

3. For \( x = 2 \):
\[ f(2) = 2^3 + 2(2) + 2 = 16 \equiv 2 \pmod{7} \]
However, since \( 2^3 = 8 \equiv 1 \pmod{7} \), we have:
\[ f(2) = 1 + 4 + 2 = 7 \equiv 0 \pmod{7} \]

4. For \( x = 3 \):
\[ f(3) = 3^3 + 2(3) + 2 = 35 \equiv 0 \pmod{7} \]

5. For \( x = 4 \):
\[ f(4) = 4^3 + 2(4) + 2 = 74 \equiv 4 \pmod{7} \]

6. For \( x = 5 \):
\[ f(5) = 5^3 + 2(5) + 2 = 135 \equiv 3 \pmod{7} \]

7. For \( x = 6 \):
\[ f(6) = 6^3 + 2(6) + 2 = 224 \equiv 6 \pmod{7} \]

From the above computations, we see that \( f(x) \) evaluates to zero in \( \Z_7 \) only for \( x = 2 \) and \( x = 3 \). 

Thus, the zeros of \( f(x) \) in \( \Z_7 \) are \( x = 2 \) and \( x = 3 \).
\end{proof}
\end{exercise}
\vspace*{20pt}
\begin{exercise}{23}  Mark each of the following true or false. \\
    \begin{enumerate}[label=(\alph*.)]
        \item The polynomial $(a_n x^n + \cdots + a_1x + a_0) \in R[x]$ is $0$ if and only if $a_i = 0$, for $i = 0, 1, \cdots , n$.
        \item If $R$ is a commutative ring, then $R[x]$ is commutative.
        \item If $D$ is an integral domain, then $D[x]$ is an integral domain.
        \item If $R$ is a ring containing divisors of $0$, then $R[x]$ has divisors of $0$.
        \item If $R$ is a ring and $f(x)$ and $g(x)$ in $R[x]$ are of degrees 3 and 4, respectively, then $f(x)g(x)$ may be of degree 8 in $R[x]$.
        \item If $R$ is any ring and f (x) and g(x) in R[x] are of degrees 3 and 4, respectively, then $f(x)g(x)$ is always of degree 7.
        \item If $F$ is a subfield of $E$ and $\alpha \in E$ is a zero of $f(x) \in F[x]$, then $\alpha$ is a zero of $h(x) = f(x)g(x)$ for all $g(x) \in F[x]$.
        \item If $F$ is a field, then the units in $F[x]$ are precisely the units in $F$.
        \item If $R$ is a ring, then $x$ is never a divisor of $0$ in $R[x]$.
        \item If $R$ is a ring, then the zero divisors in $R[x]$ are precisely the zero divisors in $R$.
    \end{enumerate}

    \begin{proof} $ $ \\
    (a) The polynomial \( (a_n x^n + \cdots + a_1x + a_0) \in R[x] \) is \( 0 \) if and only if \( a_i = 0 \), for \( i = 0, 1, \cdots , n \).\\  
    \textbf{True.} By definition of polynomial equality, two polynomials are equal if and only if their coefficients are equal.
    
    (b) If \( R \) is a commutative ring, then \( R[x] \) is commutative.\\  
    \textbf{True.} Polynomial multiplication is defined in terms of the ring multiplication, so if the coefficients from \( R \) commute, so will the polynomials in \( R[x] \).
    
    (c) If \( D \) is an integral domain, then \( D[x] \) is an integral domain.\\
    \textbf{True.} An integral domain is a commutative ring without zero divisors. If two non-zero polynomials in \( D[x] \) are multiplied, the result will not be the zero polynomial, thus \( D[x] \) has no zero divisors.
    
    (d) If \( R \) is a ring containing divisors of \( 0 \), then \( R[x] \) has divisors of \( 0 \).\\
    \textbf{True.} If \( R \) has zero divisors, then so does \( R[x] \) since the coefficients of the polynomials come from \( R \).
    
    (e) If \( R \) is a ring and \( f(x) \) and \( g(x) \) in \( R[x] \) are of degrees 3 and 4, respectively, then \( f(x)g(x) \) may be of degree 8 in \( R[x] \).\\
    \textbf{False.} The degree of the product of two polynomials is the sum of their degrees, so the degree of \( f(x)g(x) \) will be \( 3 + 4 = 7 \).
    
    (f) If \( R \) is any ring and \( f(x) \) and \( g(x) \) in \( R[x] \) are of degrees 3 and 4, respectively, then \( f(x)g(x) \) is always of degree 7.\\
    \textbf{False.} As discussed, there are cases where this might not be true, such as when the leading coefficient of one of the polynomials is a zero divisor in \( R \).
    
    (g) If \( F \) is a subfield of \( E \) and \( \alpha \in E \) is a zero of \( f(x) \in F[x] \), then \( \alpha \) is a zero of \( h(x) = f(x)g(x) \) for all \( g(x) \in F[x] \).\\
    \textbf{True.} If \( \alpha \) is a zero of \( f(x) \), then \( f(\alpha) = 0 \). Thus, \( h(\alpha) = f(\alpha)g(\alpha) = 0 \times g(\alpha) = 0 \).
    
    (h) If \( F \) is a field, then the units in \( F[x] \) are precisely the units in \( F \).\\
    \textbf{True.} In a polynomial ring, the only polynomials that have multiplicative inverses (and are thus units) are the non-zero constant polynomials, which correspond to the units in \( F \).
    
    (i) If \( R \) is a ring, then \( x \) is never a divisor of \( 0 \) in \( R[x] \).\\
    \textbf{True.} In any ring, no non-zero element can be a divisor of 0 unless the ring contains zero divisors. But \( x \) multiplied by any non-zero polynomial in \( R[x] \) will not yield the zero polynomial.
    
    (j) If \( R \) is a ring, then the zero divisors in \( R[x] \) are precisely the zero divisors in \( R \).\\
    \textbf{False.} Consider \( R = \Z_4 \). In \( R[x] \), the polynomial \( 2x \) is also a zero divisor since \( 2x \times 2x = 4x^2 = 0 \), but \( 2x \) is not in \( R \). So, \( R[x] \) can have additional zero divisors not present in \( R \).
    \end{proof}
\end{exercise}
\vspace*{20pt}
\begin{exercise}{25} Let $D$ be an integral domain and $x$ an indeterminate.
  \begin{enumerate}[label=(\alph*.)]
    \item Describe the units in $D[x]$.
    \item Find the units in $\Z[x]$.
    \item Find the units in $\Z_7[x]$.
  \end{enumerate}



(a) Describe the units in \( D[x] \).

\begin{proof}
The units in the polynomial ring \( D[x] \) are precisely the units in \( D \). This is because, in a polynomial ring over an integral domain, only the non-zero constant polynomials (those polynomials which are just constants from \( D \) with no terms involving \( x \)) have multiplicative inverses. Any polynomial with a term involving \( x \) (degree 1 or higher) cannot have a multiplicative inverse in \( D[x] \) since its product with any other polynomial will always result in a polynomial of degree higher than 0, and thus cannot equal the multiplicative identity, which is 1. Therefore, the units in \( D[x] \) are precisely the non-zero elements of \( D \) which are units.
\end{proof}

(b) Find the units in \( \Z[x] \).

\begin{proof}
In \( \Z \), the only units are 1 and -1 because they are the only integers that have multiplicative inverses in \( \Z \). Specifically, \( 1 \times 1 = 1 \) and \( (-1) \times (-1) = 1 \). Therefore, the only units in \( \Z[x] \) are 1 and -1.
\end{proof}

(c) Find the units in \( \Z_7[x] \).

\begin{proof}
In \( \Z_7 \), the units are the numbers that have multiplicative inverses modulo 7. These are all the numbers in \( \Z_7 \) except for 0, since \( \Z_7 \) is a field. Specifically, the units in \( \Z_7 \) are \( \{1, 2, 3, 4, 5, 6\} \), and each of these numbers has a multiplicative inverse in \( \Z_7 \). For example, \( 3 \times 5 \equiv 1 \mod 7 \), so 3 and 5 are multiplicative inverses of each other in \( \Z_7 \). Therefore, the units in \( \Z_7[x] \) are \( \{1, 2, 3, 4, 5, 6\} \).
\end{proof}
    
\end{exercise}
\vspace*{20pt}
\begin{exercise}{27} Let F be a field of characteristic zero and let D be the formal polynomial differentiation map, so that:
    \[D(a_0+a_1x+a_2x^2+\cdots+a_nx^n) = a_1 + 2 \cdot a_2x + \cdots +n \cdot a_nX^{n-1}.\]

    \begin{enumerate}[label=(\alph*.)]
        \item Show that $D : F[x] \rightarrow F[x]$ is a group homomorphism of $\langle F[x], +\rangle$ into itself. Is $D$ a ring homomorphism?
        \item Find the kernel of $D$.
        \item Find the image of $F[x]$ under $D$.
    \end{enumerate}

    
(a) Show that \( D : F[x] \rightarrow F[x] \) is a group homomorphism of \( \langle F[x], +\rangle \) into itself. Is \( D \) a ring homomorphism?

\begin{proof}
For \( D \) to be a group homomorphism, it must satisfy the property:
\[ D(f(x) + g(x)) = D(f(x)) + D(g(x)) \]
for all \( f(x), g(x) \in F[x] \).

Given \( f(x) = a_0 + a_1x + \cdots + a_nx^n \) and \( g(x) = b_0 + b_1x + \cdots + b_mx^m \), where \( n \) and \( m \) are the degrees of \( f(x) \) and \( g(x) \) respectively, we differentiate:

\[ D(f(x) + g(x)) = D(a_0 + a_1x + \cdots + a_nx^n + b_0 + b_1x + \cdots + b_mx^m) \]
\[ = a_1 + 2a_2x + \cdots + na_nx^{n-1} + b_1 + 2b_2x + \cdots + mb_mx^{m-1} \]
\[ = D(f(x)) + D(g(x)) \]

This shows that \( D \) is a group homomorphism with respect to addition.

However, \( D \) is not a ring homomorphism because it does not preserve multiplication. For example, consider two constant polynomials \( f(x) = a \) and \( g(x) = b \) in \( F[x] \). We have:
\[ D(f(x)g(x)) = D(ab) = 0 \]
while
\[ D(f(x))D(g(x)) = 0 \times 0 = 0 \]

Although this example works, in general:
\[ D(f(x)g(x)) \neq D(f(x))D(g(x)) \]

For instance, take \( f(x) = x \) and \( g(x) = x \). Then:
\[ D(f(x)g(x)) = D(x^2) = 2x \]
while
\[ D(f(x))D(g(x)) = 1 \times 1 = 1 \]
\end{proof}
(b) Find the kernel of \( D \).

\begin{proof}
The kernel of \( D \) consists of all polynomials \( f(x) \) in \( F[x] \) such that \( D(f(x)) = 0 \). From the definition of differentiation, it is clear that all constant polynomials will have a derivative of zero. Moreover, no other polynomial will have a derivative of zero since any polynomial with terms of degree 1 or higher will have a non-zero derivative. 

Thus, the kernel of \( D \) is the set of all constant polynomials in \( F[x] \).
\end{proof}

(c) Find the image of \( F[x] \) under \( D \).

\begin{proof}
The image of \( F[x] \) under \( D \) is the set of all possible derivatives of polynomials in \( F[x] \). 

When differentiating a polynomial of degree \( n \), we get a polynomial of degree \( n-1 \). Thus, the image of \( F[x] \) under \( D \) will contain all polynomials of degree \( n-1 \) or less. However, since \( F \) has characteristic zero, even constant terms from the original polynomial (except the leading constant) will contribute to the derivative, ensuring that all possible coefficients in the field \( F \) can be achieved. 

Therefore, the image of \( F[x] \) under \( D \) is \( F[x] \) itself, with the exception of the constant polynomials since a constant term in the original polynomial will vanish upon differentiation.
\end{proof}

\end{exercise}
\vspace*{60pt}
\section*{Chapter IV.23}
\begin{exercise}{4}
    Find $q(x)$ and $r (x)$ as described by the division algorithm so that $f (x) = g(x)q(x) + r (x)$ with $r (x) = 0$ or of degree less than the degree of $g(x)$.

    \[f (x) = x^4 + 5x^3 - 3x^2 \text{ and }g(x) = 5x^2 - x + 2\text{ in }\Z_{11}[x].\]

\begin{proof}
Find \( q(x) \) and \( r(x) \) such that

\[ f(x) = g(x)q(x) + r(x) \]

where \( r(x) = 0 \) or the degree of \( r(x) \) is less than the degree of \( g(x) \).

\subsection*{Step 1: Normalize \( g(x) \)}

Multiply \( g(x) \) by the modular inverse of 5 in \( \mathbb{Z}_{11} \), which is 9:

\[ 9 \cdot g(x) = 9 \cdot (5x^2 - x + 2) = x^2 - 9x + 7 \]

\textbf{Step 2: Perform Polynomial Long Division}
Now, perform the long division \( \frac{f(x)}{g(x)} \):

\[
\begin{array}{r|l}
9x^2 + 5x + 10 & x^4 + 5x^3 - 3x^2 \\
\hline
x^4 - 9x^3 + 7x^2 & \\
\hline
14x^3 - 10x^2 & \\
14x^3 - 126x^2 + 98x & \\
\hline
116x^2 + 98x & \\
116x^2 - 1044x + 812 & \\
\hline
1142x + 812 & \\
1142x - 10278 + 7990 & \\
\hline
10802 &
\end{array}
\]

Reduce coefficients modulo 11:

\[ 116 \mod 11 = 6, \quad 1142 \mod 11 = 9, \quad 812 \mod 11 = 9, \quad 10802 \mod 11 = 2 \]

So, the quotient is \( q(x) = 9x^2 + 5x + 10 \) and the remainder is \( r(x) = 2 \).

This satisfies the division algorithm conditions:

\[ x^4 + 5x^3 - 3x^2 = (5x^2 - x + 2)(9x^2 + 5x + 10) + 2 \]

in \( \mathbb{Z}_{11}[x] \).
\end{proof}
\end{exercise}
\vspace*{20pt}
\begin{exercise}{9}
Find all generators of the cyclic multiplicative group of units of the given finite field. (Review Corollary 6.16.)
\[\text{The polynomial }x^4 + 4\text{ can be factored into linear factors in }\Z_5[x] \text{. Find this factorization.}\]

\begin{proof}

\textbf{Part 1: Factor the Polynomial}
First, we note that in \(\Z_5\), we can treat the number 4 as -1. Therefore,

\[ x^4 + 4 \equiv x^4 - 1 \mod 5 \]

This expression can be factored using the difference of squares:

\[ x^4 - 1 = (x^2 + 1)(x^2 - 1) \]

Further factorizing \( x^2 - 1 \) as a difference of squares, and noting that \( x^2 + 1 \) can be expressed as \( (x + 2)(x + 3) \) in \(\Z_5[x]\):

\[ x^4 - 1 = (x + 1)(x - 1)(x + 2)(x + 3) = (x + 1)(x + 4)(x + 2)(x + 3) \]

So, we have factored \(x^4 + 4\) into linear factors in \(\Z_5[x]\):

\[ x^4 + 4 = (x + 1)(x + 2)(x + 3)(x + 4) \]

\textbf{Part 2: Identify the Finite Field}
Since the polynomial \(x^4 + 4\) can be factored into linear factors, the finite field defined by this polynomial is isomorphic to \(\Z_5\), and its multiplicative group of units is \(\Z_5^*\).

\textbf{Part 3: Find the Generators}
The multiplicative group of units of a finite field of order \( p \) (where \( p \) is a prime) is cyclic of order \( p - 1 \). In this case, \(\Z_5^*\) is of order 4. The generators of this group are the elements that are relatively prime to 5, which are \{1, 2, 3, 4\}. All of these elements are generators because \(\Z_5^*\) is a cyclic group of order 4, and any element in a finite cyclic group of order \( n \) that is relatively prime to \( n \) is a generator.

Hence, all the elements \{1, 2, 3, 4\} in \(\Z_5^*\) are generators of the cyclic multiplicative group of units of the finite field defined by the polynomial \(x^4 + 4\) in \(\Z_5[x]\).
\end{proof}
\end{exercise}
\vspace*{20pt}
\begin{exercise}{15}
    Show that $g(x) = x^2 + 6x + 12$ is irreducible over $\Q$. Is $g (x)$ irreducible over $\R$? Over $\C$?
    
    \begin{proof}
        \textbf{Part 1: Irreducibility over \( \mathbb{Q} \)}
        To show that \( g(x) \) is irreducible over \( \mathbb{Q} \), we need to show that it cannot be factored into non-constant polynomials with coefficients in \( \mathbb{Q} \).
        
        The polynomial \( g(x) \) is a quadratic polynomial, and it is well-known that a quadratic polynomial is irreducible over \( \mathbb{Q} \) if and only if it has no roots in \( \mathbb{Q} \).
        
        Consider the discriminant of \( g(x) \):
        
        \[ \Delta = b^2 - 4ac = (6)^2 - 4(1)(12) = 36 - 48 = -12 \]
        
        Since the discriminant is negative, there are no real roots, and hence no rational roots. Therefore, \( g(x) \) is irreducible over \( \mathbb{Q} \).
        
        \textbf{ Part 2: Irreducibility over \( \mathbb{R} \)}
        Over the real numbers \( \mathbb{R} \), a polynomial is irreducible if it is linear or a quadratic with no real roots. Since \( g(x) \) is a quadratic polynomial with no real roots (as shown by the negative discriminant), it is irreducible over \( \mathbb{R} \).
        
        \textbf{Part 3: Irreducibility over \( \mathbb{C} \)}
        Over the complex numbers \( \mathbb{C} \), every non-constant polynomial can be factored into linear factors. Therefore, \( g(x) \) is not irreducible over \( \mathbb{C} \). In fact, we can find its roots using the quadratic formula:
        
        \[ x = \frac{-b \pm \sqrt{\Delta}}{2a} = \frac{-6 \pm \sqrt{-12}}{2} = \frac{-6 \pm 2i\sqrt{3}}{2} = -3 \pm i\sqrt{3} \]
        
        So, \( g(x) \) can be factored over \( \mathbb{C} \) as:
        
        \[ g(x) = (x - (-3 + i\sqrt{3}))(x - (-3 - i\sqrt{3})) \]
    \end{proof}
\end{exercise}
    \end{document}



































