\documentclass[12pt]{amsart}
\usepackage[margin=1in]{geometry}
\usepackage{amssymb,amsfonts,amsmath}
\usepackage{color}
\usepackage{enumerate}
\usepackage{mathrsfs}
\usepackage{hyperref}
\usepackage[capitalise]{cleveref}
\usepackage{constants}
\usepackage{parskip}
\usepackage{indentfirst}
\usepackage{amsmath}
\usepackage{enumitem}
\setlength{\parindent}{2em}
\hfuzz=200pt

%----Table of Contents-----

%----Theorem Environments----
\newtheorem{theorem}{Theorem}[section]
\newtheorem{corollary}[theorem]{Corollary}
\newtheorem{hypothesis}[theorem]{Hypothesis}
\newtheorem{proposition}[theorem]{Proposition}
\newtheorem{lemma}[theorem]{Lemma}

\newtheorem{problem*}{Problem}

\theoremstyle{definition}
\newtheorem{definition}[theorem]{Definition}
\newtheorem{example}[theorem]{Example}
\newcommand{\exercise}[1]{\noindent {\bf Exercise #1.}}

\numberwithin{equation}{section}


\crefname{figure}{Figure}{Figures}
%MATH ENVIRONMENTS
\theoremstyle{plain}
\newtheorem*{theorem*}{Theorem}
\crefname{theorem}{Theorem}{Theorems}
\crefname{cor}{Corollary}{Corollaries}
\crefname{exercise}{Exercise}{Exercises}
\newtheorem*{cor*}{Corollary}
\crefname{cor*}{Corollary}{Corollaries}
\crefname{lem}{Lemma}{Lemmas}
\crefname{prop}{Proposition}{Propositions}
\crefname{conj}{Conjecture}{Conjectures}
\newtheorem*{conj*}{Conjecture}
\crefname{conj*}{Conjecture}{Conjectures}
\crefname{defn}{Definition}{Definitions}
\crefname{hyp}{Hypothesis}{Hypotheses}


\newcommand{\Z}{\mathbb{Z}}
\renewcommand{\C}{\mathbb{C}}
\newcommand{\R}{\mathbb{R}}
\newcommand{\Q}{\mathbb{Q}}
\newcommand{\F}{\mathbb{F}}
\newcommand{\N}{\mathbb{N}}
\newcommand{\re}{\textup{Re}}
\newcommand{\im}{\textup{Im}}
\renewcommand{\epsilon}{\varepsilon}
\newcommand{\Li}{\mathrm{Li}}


\title{Math 417, Homework 7 \& 8}
\author{Charles Ancel}

%%%%%%%%%%%%%%%%%%%%%%%%%%%%%%%%%%%%%%%%%%%%%%%%%%%%%%%%%%%%%%%%%%%%%
%%%%%%%%%%%%%%%%%%%%%%%%%%%%%%%%%%%%%%%%%%%%%%%%%%%%%%%%%%%%%%%%%%%%%
%%%%%%%%%%%%%%%%%%%%%%%%%%%%%%%%%%%%%%%%%%%%%%%%%%%%%%%%%%%%%%%%%%%%%
\begin{document}
\maketitle
I know that homework for chapter III.16 was due last week so I understand that it may not be worth any points but please let me know if it is correct. Have a great night and thank you for grading all of our work!
\section*{Chapter III.16}

\begin{exercise}{8(dfgh)}
Mark each of the following true or false. \\
d. Let $X$ be a $G$-set with $x_1$, $x_2 \in X$ and $g \in G$. If $gx_1 = gx_2$, then $x_1 = x_2$. \\
f. Each orbit of a $G$-set $X$ is a transitive sub-$G$-set. \\
g. Let $X$ be a $G$-set and let $H \leq G$. Then $X$ can be regarded in a natural way as an $H$-set. \\
h. With reference to (g), the orbits in $X$ under $H$ are the same as the orbits in $X$ under $G$. \\
    
    \begin{proof}

\textbf{d. Let \(X\) be a \(G\)-set with \(x_1\), \(x_2 \in X\) and \(g \in G\). If \(gx_1 = gx_2\), then \(x_1 = x_2\).}

This statement is true if the action of \(G\) on \(X\) is faithful (or free). Recall that a group action is faithful if, whenever \(g \cdot x = h \cdot x\) for all \(x\) in \(X\), then \(g = h\). In the context of this statement, if the action of \(g\) on both \(x_1\) and \(x_2\) results in the same element, then \(x_1\) must equal \(x_2\).

\textbf{Answer: True}(assuming the action is faithful).

---

\textbf{f. Each orbit of a \(G\)-set \(X\) is a transitive sub-\(G\)-set.}

By definition, the orbit of an element \(x\) in \(X\) under the action of \(G\) is the set of all elements of \(X\) to which \(x\) can be moved by the action of some element in \(G\). Therefore, for any two elements \(y, z\) in the same orbit, there exists \(g_1, g_2 \in G\) such that \(g_1 \cdot x = y\) and \(g_2 \cdot x = z\). It follows that \(z = g_2 \cdot x = g_2 \cdot (g_1^{-1} \cdot y) = (g_2 g_1^{-1}) \cdot y\), where \(g_2 g_1^{-1} \in G\). This shows that any two elements in the same orbit can be related by the action of some element in \(G\), which means the orbit is a transitive subset of \(X\).

\textbf{Answer: True.}

---

\textbf{g. Let \(X\) be a \(G\)-set and let \(H \leq G\). Then \(X\) can be regarded in a natural way as an \(H\)-set.}

If \(H\) is a subgroup of \(G\), then every element of \(H\) is also an element of \(G\). This means that every action of \(H\) on \(X\) is also an action of \(G\) on \(X\). So, \(X\) can naturally be regarded as an \(H\)-set using the same group action.

\textbf{Answer: True}.

---

\textbf{h. With reference to (g), the orbits in \(X\) under \(H\) are the same as the orbits in \(X\) under \(G\).}

This is not necessarily true. Consider the group action of \(G\) on \(X\). It's possible that \(H\) (being a subgroup) might not move \(X\) around as much as \(G\) does. Thus, while \(G\) might move an element \(x\) in \(X\) to a wide variety of other positions, \(H\) might only move \(x\) to a subset of those positions. This means that the orbits of \(X\) under the action of \(H\) might be smaller or at least different than the orbits of \(X\) under the action of \(G\).

\textbf{Answer: False}.

---
    \end{proof}
    
\end{exercise}
\vspace*{20pt}
\begin{exercise}{12} Let $X$ be a $G$-set and let $Y \subseteq X$. Let $G_Y = \{g \in G \mid , g_y = y \forall y \in Y \}$. Show $G_Y$ is a subgroup of $G$, generalizing Theorem 16.12.
        
    \begin{proof}
        To prove that \( G_Y \) is a subgroup of \( G \), we'll use the subgroup test. For \( G_Y \) to be a subgroup, it must satisfy:

1. The identity element of \( G \), \( e \), is in \( G_Y \).
2. If \( g_1, g_2 \in G_Y \), then their product \( g_1g_2 \) is in \( G_Y \).
3. If \( g \in G_Y \), then its inverse \( g^{-1} \) is in \( G_Y \).

Let's prove each of these properties:

1. \textbf{Identity element:} For any \( y \in Y \), \( ey = y \). Thus, \( e \in G_Y \).

2. \textbf{Closure:} Let \( g_1, g_2 \in G_Y \). This means that \( g_1y = y \) and \( g_2y = y \) for all \( y \in Y \). We want to show that \( (g_1g_2)y = y \) for all \( y \in Y \).

Given \( g_1y = y \) and \( g_2y = y \), we have:

\[
(g_1g_2)y = g_1(g_2y) = g_1y = y
\]

This shows that \( g_1g_2 \in G_Y \) and thus \( G_Y \) is closed under the induced operation of \( G \).

3. \textbf{Inverse:} Let \( g \in G_Y \). This means \( G_Y = y \) for all \( y \in Y \). We want to show that \( g^{-1}y = y \) for all \( y \in Y \).

Since \( G_Y = y \), we have:

\[
y = ey = (g^{-1}g)y = g^{-1}(G_Y) = g^{-1}y
\]

Thus, \( g^{-1} \in G_Y \).

Since \( G_Y \) satisfies all three properties, we conclude that \( G_Y \) is a subgroup of \( G \), generalizing Theorem 16.12.
        \end{proof}
\end{exercise}
\vspace*{20pt}
\section*{Chapter III.17}
\begin{exercise}{1} Find the number of orbits in $\{1, 2, 3, 4, 5, 6, 7, 8\}$ under the cyclic subgroup $\langle(1, 3, 5, 6)\rangle$ of $S_8$.
    
    To find the number of orbits in \(\{1, 2, 3, 4, 5, 6, 7, 8\}\) under the cyclic subgroup \(\langle(1, 3, 5, 6)\rangle\) of \(S_8\), we'll first need to understand the action of the subgroup on the set.

    Recall that in the symmetric group \(S_8\), the cycle notation \((1, 3, 5, 6)\) represents the permutation that sends:
    1. 1 to 3
    2. 3 to 5
    3. 5 to 6
    4. 6 to 1
    5. And leaves all other numbers fixed.
    
    Since our subgroup is generated by this single cycle, the only elements in the subgroup are powers of this cycle and the identity:
    1. \(e\): the identity permutation (does nothing)
    2. \((1, 3, 5, 6)\): described above
    3. \((1, 3, 5, 6)^2 = (1, 5)(3, 6)\): sends 1 to 5, 5 to 1, 3 to 6, 6 to 3, and leaves all other numbers fixed.
    4. \((1, 3, 5, 6)^3 = (1, 6, 5, 3)\): the inverse of the original cycle.
    
    Let's compute the orbits:
    
    \begin{proof}
    1. Start with 1: 
    The powers of the cycle send 1 to 3, 5, 6, and then back to 1. So the orbit of 1 under the subgroup is \(\{1, 3, 5, 6\}\).
    
    2. Take 2:
    2 is not affected by any element in our subgroup (it remains fixed). So its orbit is \(\{2\}\).
    
    3. Take 4:
    Similar to 2, 4 is not affected by any element in our subgroup. So its orbit is \(\{4\}\).
    
    4. Take 7:
    Again, 7 remains fixed under all elements of our subgroup. So its orbit is \(\{7\}\).
    
    5. Take 8:
    Similarly, 8 remains fixed under all elements of our subgroup. So its orbit is \(\{8\}\).
    
    So, the orbits are:
    \(\{1, 3, 5, 6\}\), \(\{2\}\), \(\{4\}\), \(\{7\}\), and \(\{8\}\).
    
    Hence, there are \(5\) orbits in \(\{1, 2, 3, 4, 5, 6, 7, 8\}\) under the cyclic subgroup \(\langle(1, 3, 5, 6)\rangle\) of \(S_8\).
    \end{proof}
\end{exercise}
\vspace*{20pt}
\section*{Chapter VII.36}
\begin{exercise}{2}
To determine the order of a Sylow 3-subgroup of a group with order 54, let's first express 54 in terms of its prime factorization:

\[ 54 = 2 \times 3^3 \]

The highest power of 3 that divides 54 is \(3^3\).

Given the Sylow Theorems, a Sylow \( p \)-subgroup of a group \( G \) of order \( p^n \cdot m \) (where \( p \) does not divide \( m \)) has order \( p^n \).

In this case, \( p = 3 \) and \( n = 3 \). Therefore, a Sylow 3-subgroup of a group of order 54 has order \( 3^3 = 27 \).

Answer: A Sylow 3-subgroup of a group of order 54 has order \( \boxed{27} \).
\end{exercise}
\vspace*{20pt}
\begin{exercise}{10} 
Mark each of the following true or false.
a. Any two Sylow $p$-subgroups of a finite group are conjugate.\\
b. Theorem 36.11 shows that a group of order 15 has only one Sylow 5-subgroup.\\
c. Every Sylow $p$-subgroup of a finite group has order a power of $p$.\\
d. Every $p$-subgroup of every finite group is a Sylow $p$-subgroup.\\
e. Every finite abelian group has exactly one Sylow $p$-subgroup for each prime $p$ dividing the order of $G$.\\
f. The normalizer in G of a subgroup $H$ of $G$ is always a normal subgroup of G.\\
g. If $H$ is a subgroup of $G$, then $H$ is always a normal subgroup of $N[H]$.\\
h. A Sylow $p$-subgroup of a finite group $G$ is normal in $G$ if and only if it is the only Sylow $p$-subgroup of $G$.\\
i. If $G$ is an abelian group and $H$ is a subgroup of $G$, then $N[H] = H$.\\
j. A group of prime-power order $p^n$ has no Sylow $p$-subgroup.\\
    
    \begin{proof}
\textbf{a. Any two Sylow \( p \)-subgroups of a finite group are conjugate.}\\

\textbf{Answer: True.} This is one of the Sylow theorems.

---

\textbf{b. Theorem 36.11 shows that a group of order 15 has only one Sylow 5-subgroup.}\\

Without the explicit content of Theorem 36.11, it's hard to say for certain, but the statement itself is true. A group of order 15 has only one Sylow 5-subgroup, which must be normal.

\textbf{Answer: True.}

---

\textbf{c. Every Sylow \( p \)-subgroup of a finite group has order a power of \( p \).}\\

\textbf{Answer: True.} This is the definition of a Sylow \( p \)-subgroup.

---

\textbf{d. Every \( p \)-subgroup of every finite group is a Sylow \( p \)-subgroup.}\\

\textbf{Answer: False.} A Sylow \( p \)-subgroup is a \( p \)-subgroup of largest order. Not every \( p \)-subgroup has to be a Sylow \( p \)-subgroup.

---

\textbf{e. Every finite abelian group has exactly one Sylow \( p \)-subgroup for each prime \( p \) dividing the order of \( G \).}\\

\textbf{Answer: True.} In an abelian group, any two subgroups of the same order are conjugate, and since Sylow subgroups are conjugate by the Sylow theorems, there can only be one distinct Sylow \( p \)-subgroup.

---

\textbf{f. The normalizer in \( G \) of a subgroup \( H \) of \( G \) is always a normal subgroup of \( G \).}\\

\textbf{Answer: False.} The normalizer of \( H \) in \( G \), denoted \( N_G(H) \), is the largest subgroup of \( G \) in which \( H \) is normal. However, this doesn't mean \( N_G(H) \) itself is normal in \( G \).

---

\textbf{g. If \( H \) is a subgroup of \( G \), then \( H \) is always a normal subgroup of \( N[H] \).}\\

\textbf{Answer: True.} By definition, the normalizer \( N[H] \) is the largest subgroup of \( G \) in which \( H \) is normal.

---

\textbf{h. A Sylow \( p \)-subgroup of a finite group \( G \) is normal in \( G \) if and only if it is the only Sylow \( p \)-subgroup of \( G \).}\\

\textbf{Answer: True.} This follows from the Sylow theorems. If there is only one Sylow \( p \)-subgroup, it must be normal in \( G \).

---

\textbf{i. If \( G \) is an abelian group and \( H \) is a subgroup of \( G \), then \( N[H] = H \).}\\

\textbf{Answer: False.} In an abelian group, every subgroup is normal. So, the normalizer of any subgroup \( H \) is the entire group \( G \), i.e., \( N[H] = G \).

---

\textbf{j. A group of prime-power order \( p^n \) has no Sylow \( p \)-subgroup.}\\

\textbf{Answer: False.} A group of order \( p^n \) has a Sylow \( p \)-subgroup of order \( p^n \), which is the group itself.

    \end{proof}
\end{exercise}
\vspace*{20pt}
\begin{exercise}{13}
Show that every group of order 45 has a normal subgroup of order 9.\\

To show that every group \( G \) of order 45 has a normal subgroup of order 9, we can use the Sylow theorems. Let's break down the order of the group in terms of its prime factorization:

\[ 45 = 3^2 \times 5 \]

From the Sylow theorems, the number of Sylow 3-subgroups (let's call it \( n_3 \)) divides the order of the group, and \( n_3 \equiv 1 \mod 3 \). The possible values for \( n_3 \) are 1 or 5.

However, if \( n_3 \) were 5, then there would be \( 5(3^2 - 1) = 20 \) elements of order 3 (since each Sylow 3-subgroup contributes \( 3^2 - 1 \) elements of order 3, and these subgroups intersect trivially). This would leave only 25 elements in \( G \) not of order 3. But then, since the number of Sylow 5-subgroups divides 9 and is congruent to 1 mod 5, the only possible number of Sylow 5-subgroups is 1. This Sylow 5-subgroup has \( 5 - 1 = 4 \) elements of order 5, leaving 21 elements unaccounted for, which is a contradiction.

Thus, \( n_3 \) cannot be 5. So, \( n_3 = 1 \), which means there is only one Sylow 3-subgroup, and it is therefore normal in \( G \). This Sylow 3-subgroup has order \( 3^2 = 9 \).

\begin{proof}
Let \( G \) be a group of order 45, and let \( n_3 \) be the number of Sylow 3-subgroups of \( G \). From the Sylow theorems, \( n_3 \) divides 45 and \( n_3 \equiv 1 \mod 3 \). This means \( n_3 \) can only be 1 or 5.

Assume, for the sake of contradiction, that \( n_3 = 5 \). Then, there are \( 5(3^2 - 1) = 20 \) elements of order 3 in \( G \). This leaves only 25 elements not of order 3. Since the number of Sylow 5-subgroups divides 9 and is congruent to 1 mod 5, there can only be 1 Sylow 5-subgroup. This contributes 4 elements of order 5, leaving 21 elements unaccounted for, which is a contradiction.

Therefore, \( n_3 \) must be 1, and there is a unique Sylow 3-subgroup of \( G \) which is of order 9. Being the unique Sylow 3-subgroup, it is normal in \( G \).
\end{proof}
\end{exercise}
\vspace*{20pt}
\begin{exercise}{17}
Show that every group of order $(35)^3$ has a normal subgroup of order 125.

We will show that every group \(G\) of order \(35^3 = 5^3 \times 7^3\) has a normal subgroup of order \(125 = 5^3\). We'll use the Sylow Theorems in our proof.

According to the Sylow Theorems, the number of Sylow \(5\)-subgroups (denoted as \(n_5\)) satisfies
1. \(n_5 \equiv 1 \pmod{5}\),
2. \(n_5\) divides \(7^3\).

The possible values of \(n_5\) that satisfy these conditions are \(1, 7, 49\).

\begin{proof}
Assume that \(G\) is a group of order \(35^3\). We'll prove that \(G\) has a normal subgroup of order \(125\).

We first consider the Sylow \(5\)-subgroups of \(G\). Let \(n_5\) be the number of such subgroups. By the Sylow theorems, \(n_5\) must be congruent to \(1\) mod \(5\) and must divide \(7^3\). So, the possible values for \(n_5\) are \(1, 7, 49\).

We proceed by cases:

1. If \(n_5 = 1\), then there is a unique Sylow \(5\)-subgroup of \(G\), and it is normal by the Sylow theorems. This subgroup has order \(5^3 = 125\), which is what we want to prove.

2. If \(n_5 = 7\) or \(49\), we consider the action of \(G\) on the set of Sylow \(5\)-subgroups by conjugation. This action gives rise to a homomorphism

\[
\phi: G \to S_{n_5}
\]

where \(S_{n_5}\) is the symmetric group on \(n_5\) elements. The kernel of this homomorphism, denoted by \(\ker(\phi)\), is a normal subgroup of \(G\). By the first isomorphism theorem, \(G/\ker(\phi)\) is isomorphic to a subgroup of \(S_{n_5}\), and therefore

\[
[G : \ker(\phi)] \leq |S_{n_5}| = n_5!.
\]

Because the order of \(G\) is \(35^3\) and \(n_5\) is either \(7\) or \(49\), we have

\[
\ker(\phi) \geq \frac{35^3}{49!} \geq 5^3.
\]

Thus, the kernel has at least \(5^3\) elements. Since it is a subgroup of \(G\), its order must divide \(35^3\), so its order is at least \(5^3 = 125\). Moreover, as the kernel of a homomorphism, it is a normal subgroup of \(G\).

So, in all cases, we have shown that there exists a normal subgroup of \(G\) of order \(125\).
\end{proof}
\end{exercise}
\vspace*{20pt}
\begin{exercise}{18} Show that there are no simple groups of order 255 = (3)(5)(17).

To show that there are no simple groups of order \( 255 = 3 \times 5 \times 17 \), we will use the Sylow Theorems. Let's consider a group \( G \) of order 255. 

1. The number of Sylow 17-subgroups of \( G \) (denoted \( n_{17} \)) satisfies:\\
   $\bullet$ \( n_{17} \equiv 1 \pmod{17} \) \\
   $\bullet$ \( n_{17} \) divides \( 3 \times 5 = 15 \)

The only possibility is \( n_{17} = 1 \). Thus, there is a unique Sylow 17-subgroup, which is normal in \( G \) by the Sylow Theorems.

2. Similarly, the number of Sylow 5-subgroups of \( G \) (denoted \( n_{5} \)) satisfies:\\
   $\bullet$ \( n_{5} \equiv 1 \pmod{5} \)\\
   $\bullet$ \( n_{5} \) divides \( 3 \times 17 = 51 \)

The possible values for \( n_{5} \) are 1 and 51. If \( n_{5} = 1 \), then there is a unique Sylow 5-subgroup, which is normal in \( G \).

3. Lastly, the number of Sylow 3-subgroups of \( G \) (denoted \( n_{3} \)) satisfies:\\
   $\bullet$ \( n_{3} \equiv 1 \pmod{3} \)\\
   $\bullet$ \( n_{3} \) divides \( 5 \times 17 = 85 \)

The possible values for \( n_{3} \) are 1, 5, 17, and 85. If \( n_{3} = 1 \), then there is a unique Sylow 3-subgroup, which is normal in \( G \).

\begin{proof}
From the Sylow Theorems, we have determined that:\\
1. There is a unique Sylow 17-subgroup of \( G \), which is normal.\\
2. There may be a unique Sylow 5-subgroup of \( G \), which, if it exists, is normal.\\
3. There may be a unique Sylow 3-subgroup of \( G \), which, if it exists, is normal.\\

Since a simple group has no non-trivial normal subgroups, and we've established that \( G \) has at least one non-trivial normal subgroup (from point 1), it follows that \( G \) cannot be simple.

Therefore, there are no simple groups of order 255.
\end{proof}

\end{exercise}
\vspace*{20pt}
\section*{Chapter IV.18}
\begin{exercise}{3}
    Compute the product in the given ring: $(11)(-4)$ in $\Z_15$.
    \begin{proof}
    \begin{align*}
    (11)(-4) &= -44 \\
    -44 \mod 15 &= -14 \mod 15 \\
    &= 1 \quad \text{(since -14 is congruent to 1 modulo 15)}
    \end{align*}
    
    Thus, the product \( (11)(-4) \) in \(\Z_{15} \) is 1.
    \end{proof}
    
\end{exercise}
\vspace*{20pt}
\begin{exercise}{9} 
    Decide whether the indicated operations of addition and multiplication are defined (closed) on the set, and give a ring structure. 
    If a ring is not formed, tell why this is the case. If a ring is formed, state whether the ring is commutative, whether it has unity, and whether it is a field.
    
    $\bullet$ $\Z \times \Z$ with addition and multiplication by components.

    Consider the set \(\Z \times\Z \), which consists of ordered pairs of integers. We want to determine if the operations of addition and multiplication, defined component-wise, give a ring structure to this set.

\begin{proof} $ $ \\
1. \textbf{Addition:} For any \( (a, b), (c, d) \in\Z \times\Z \), the sum is defined as:
\[ (a, b) + (c, d) = (a+c, b+d) \]
Since the sum of two integers is an integer, the result \( (a+c, b+d) \) is still in \(\Z \times\Z \). Thus, addition is closed.

2. \textbf{Multiplication:} For any \( (a, b), (c, d) \in\Z \times\Z \), the product is defined as:
\[ (a, b) \cdot (c, d) = (a \cdot c, b \cdot d) \]
Again, the product of two integers is an integer, so the result \( (a \cdot c, b \cdot d) \) is in \(\Z \times\Z \). Thus, multiplication is closed.

The usual ring axioms (associativity, distributivity, existence of an additive identity, existence of additive inverses, etc.) hold for \(\Z \times\Z \) under these operations. 

$\bullet$ The additive identity is \( (0,0) \) because for any \( (a,b) \), we have:
\[ (a,b) + (0,0) = (a+0, b+0) = (a,b) \]

$\bullet$ The additive inverse of \( (a,b) \) is \( (-a,-b) \) because:
\[ (a,b) + (-a,-b) = (a+(-a), b+(-b)) = (0,0) \]

$\bullet$ The ring is \textbf{commutative} under both addition and multiplication. This is clear from the component-wise definitions of these operations.

$\bullet$ The ring has a \textbf{unity} or multiplicative identity, which is \( (1,1) \). For any \( (a,b) \), we have:
\[ (a,b) \cdot (1,1) = (a \cdot 1, b \cdot 1) = (a,b) \]

$\bullet$ However, the ring is \textbf{not a field}. To see why, consider the element \( (1,0) \). While \( (1,0) \) is not the zero element, it does not have a multiplicative inverse in \(\Z \times\Z \). Any potential inverse \( (a,b) \) would need to satisfy \( (1,0) \cdot (a,b) = (1,1) \), but this is impossible since the second component of the product is 0.

\end{proof}
\end{exercise}
\vspace*{20pt}
\begin{exercise}{11}$ $ \\
    Decide whether the indicated operations of addition and multiplication are defined (closed) on the set, and give a ring structure. 
    If a ring is not formed, tell why this is the case. If a ring is formed, state whether the ring is commutative, whether it has unity, and whether it is a field.

    $\bullet$ $\{a + b\sqrt{2} \mid a, b \in \Z\}$ with the usual addition and multiplication.

    
    Consider the set \( R = \{a + b\sqrt{2} \mid a, b \in\Z\} \). We want to determine if the operations of addition and multiplication, as usually defined for real numbers, give a ring structure to this set.

\begin{proof}
1. \textbf{Addition:} For any \( x = a + b\sqrt{2} \) and \( y = c + d\sqrt{2} \) in \( R \), the sum is defined as:
\[ x + y = (a + c) + (b + d)\sqrt{2} \]
Both \( a + c \) and \( b + d \) are integers, so \( x + y \) belongs to \( R \). Thus, addition is closed.

2. \textbf{Multiplication:} For any \( x = a + b\sqrt{2} \) and \( y = c + d\sqrt{2} \) in \( R \), the product is defined as:
\[ x \cdot y = (a \cdot c + 2b \cdot d) + (a \cdot d + b \cdot c)\sqrt{2} \]
Both \( a \cdot c + 2b \cdot d \) and \( a \cdot d + b \cdot c \) are integers, so \( x \cdot y \) belongs to \( R \). Thus, multiplication is closed.

The usual ring axioms (associativity, distributivity, existence of an additive identity, existence of additive inverses, etc.) hold for \( R \) under these operations.

$\bullet$ The additive identity is \( 0 \) because for any \( x = a + b\sqrt{2} \), we have:
\[ x + 0 = a + b\sqrt{2} + 0 = a + b\sqrt{2} = x \]

$\bullet$ The additive inverse of \( x = a + b\sqrt{2} \) is \( -x = -a - b\sqrt{2} \) because:
\[ x + (-x) = a - a + b\sqrt{2} - b\sqrt{2} = 0 \]

$\bullet$ The ring is \textbf{commutative} under both addition and multiplication. This is clear from the definitions of these operations.

$\bullet$ The ring has a \textbf{unity} or multiplicative identity, which is \( 1 \) because for any \( x = a + b\sqrt{2} \), we have:
\[ x \cdot 1 = (a + b\sqrt{2}) \cdot 1 = a + b\sqrt{2} = x \]

$\bullet$ To determine if the ring is a field, we'd need to check if every non-zero element has a multiplicative inverse in \( R \). However, not every non-zero element in \( R \) has an inverse in \( R \). For instance, the element \( 1 + \sqrt{2} \) does not have a multiplicative inverse in \( R \). Thus, \( R \) is \textbf{not a field}.

\end{proof}
\end{exercise}
\vspace*{20pt}
\begin{exercise}{15}
    Describe all units in the given ring: 
    $\bullet$ $\Z \times \Z$


    For the ring \(\Z \times\Z \), a unit (or invertible element) is an element that has a multiplicative inverse in the ring.

    \begin{proof}
    Consider an element \( (a,b) \) in \(\Z \times\Z \). For \( (a,b) \) to be a unit, there must exist an element \( (c,d) \) in \(\Z \times\Z \) such that:
    
    \[ (a,b) \cdot (c,d) = (1,1) \]
    
    From this, we get:
    
    1) \( ac = 1 \) and
    2) \( bd = 1 \)
    
    Given that \( a, b, c, \) and \( d \) are integers, the only way for \( ac \) to be 1 is if \( a = c = 1 \) or \( a = c = -1 \). Similarly, the only way for \( bd \) to be 1 is if \( b = d = 1 \) or \( b = d = -1 \).
    
    From the above, we can deduce that the units in \(\Z \times\Z \) are:
    
    \[ (1,1), \ (-1,-1), \ (1,-1), \ \text{and} \ (-1,1) \]
    
    These are the only elements in \(\Z \times\Z \) that have multiplicative inverses in the ring.
    \end{proof}
    
    So, the units in \(\Z \times\Z \) are \( (1,1), (-1,-1), (1,-1), \) and \( (-1,1) \).\\

\end{exercise}
\vspace*{20pt}
\begin{exercise}{24} 
    Describe all ring homomorphisms of $\Z$ into $\Z \times \Z$.

    To describe all ring homomorphisms \( \phi:\Z \to\Z \times\Z \), we need to consider the properties that must be satisfied by a ring homomorphism. Specifically, for all \( a, b \in\Z \):\\
1. \( \phi(a + b) = \phi(a) + \phi(b) \)\\
2. \( \phi(a \cdot b) = \phi(a) \cdot \phi(b) \)\\
3. \( \phi(1) = 1 \) where 1 in \(\Z \times\Z \) is the multiplicative identity, which is \( (1,1) \).\\

\begin{proof}
Let's use the properties of ring homomorphisms to determine the structure of \( \phi \):

1. Given \( \phi(1) = (1,1) \), we can determine \( \phi \) for all integers. For any positive integer \( n \), we have:
\[ \phi(n) = \phi(1 + 1 + \dots + 1) = \phi(1) + \phi(1) + \dots + \phi(1) = (n,n) \]

Similarly, for any negative integer \( -n \), we have:
\[ \phi(-n) = -\phi(n) = (-n,-n) \]

2. For \( a, b \in\Z \), using the properties of ring homomorphisms, we have:
\[ \phi(a \cdot b) = \phi(a) \cdot \phi(b) \]
\[ \phi(a + b) = \phi(a) + \phi(b) \]

These properties confirm our above derivation for \( \phi(n) \) and \( \phi(-n) \).

Given these results, we can conclude that the only ring homomorphism \( \phi:\Z \to\Z \times\Z \) is given by:
\[ \phi(n) = (n,n) \]
for all \( n \in\Z \).
\end{proof}

Thus, the only ring homomorphism from \(\Z \) into \(\Z \times\Z \) maps each integer \( n \) to the ordered pair \( (n,n) \).
\end{exercise}
\vspace*{20pt}
\begin{exercise}{33} 
    Mark each of the following true or false.\\
a. Every field is also a ring.\\
b. Every ring has a multiplicative identity.\\
c. Every ring with unity has at least two units.\\
d. Every ring with unity has at most two units.\\
e. It is possible for a subset of some field to be a ring but not a subfield, under the induced operations.\\
f. The distributive laws for a ring are not very important.\\
g. Multiplication in a field is commutative.\\
h. The nonzero elements of a field form a group under the multiplication in the field.\\
i. Addition in every ring is commutative.\\
j. Every element in a ring has an additive inverse.\\



\begin{proof} $ $ \\

\textbf{a. Every field is also a ring.} \\
\textbf{Answer: True.} By definition, a field is a set with two operations (addition and multiplication) that satisfy the properties of a ring, and in addition, every non-zero element has a multiplicative inverse. So, a field satisfies all properties of a ring.

\textbf{b. Every ring has a multiplicative identity.} \\
Answer: False. There are rings without a multiplicative identity. Such rings are called non-unital rings.

\textbf{c. Every ring with unity has at least two units.} \\
\textbf{Answer: True.} The unity (or 1) is a unit by definition, and the additive inverse of the unity, -1, is also a unit. Thus, there are at least these two units in any ring with unity.

\textbf{d. Every ring with unity has at most two units.} \\
\textbf{Answer: False.} Consider the ring \( \Z \). Both 1 and -1 are units, but so are any other integers coprime to the modulus.

\textbf{e. It is possible for a subset of some field to be a ring but not a subfield, under the induced operations.} \\
\textbf{Answer: True.} Consider the field \( \mathbb{R} \) of real numbers. The subset \( \Z \) of integers is a ring under the induced operations, but it's not a subfield because it lacks multiplicative inverses for all non-zero integers.

\textbf{f. The distributive laws for a ring are not very important.} \\
\textbf{Answer: False.} The distributive laws are essential for the structure of a ring. They ensure the compatibility of the two operations: addition and multiplication.

\textbf{g. Multiplication in a field is commutative.} \\
\textbf{Answer: True.} By definition, a field is a commutative ring where every non-zero element has a multiplicative inverse.

\textbf{h. The nonzero elements of a field form a group under the multiplication in the field.} \\
\textbf{Answer: True.} In a field, the set of non-zero elements is closed under multiplication, each element has a multiplicative inverse, and multiplication is associative. Thus, they form a group.

\textbf{i. Addition in every ring is commutative.} \\
\textbf{Answer: True.} Commutativity of addition is one of the defining properties of a ring.

\textbf{j. Every element in a ring has an additive inverse.} \\
\textbf{Answer: True.} The existence of additive inverses is one of the defining properties of a ring.
\end{proof}
\vspace*{20pt}
\end{exercise}
\begin{exercise}{41}
    (Freshman exponentiation) Let $p$ be a prime. Show that in the ring $\Z_p$ we have $(a + b)^p = a^p + b^p$ for all
$a, b \in Zp$. [Hint: Observe that the usual binomial expansion for $(a + b)^n$ is valid in a commutative ring.]
    \begin{proof}
        To prove the statement, we'll first consider the binomial expansion of \( (a + b)^p \) in a commutative ring:

\[
(a + b)^p = \binom{p}{0}a^p + \binom{p}{1}a^{p-1}b + \binom{p}{2}a^{p-2}b^2 + \dots + \binom{p}{p-1}ab^{p-1} + \binom{p}{p}b^p
\]

Now, for any integer \( 1 \leq k \leq p-1 \), the binomial coefficient \( \binom{p}{k} \) is given by:

\[
\binom{p}{k} = \frac{p!}{k!(p-k)!}
\]

Observe that the numerator \( p! \) has \( p \) as a factor (since \( p! = p \times (p-1)! \)), but neither \( k! \) nor \( (p-k)! \) have \( p \) as a factor (since \( 1 \leq k \leq p-1 \)). Therefore, when working in \( \Z_p \), \( \binom{p}{k} \) is equivalent to 0 for all \( 1 \leq k \leq p-1 \) since it's a multiple of \( p \).

So, the only terms that remain in the expansion of \( (a + b)^p \) in \( \Z_p \) are:

\[
(a + b)^p \equiv a^p + b^p \pmod{p}
\]

And since \( \Z_p \) is defined mod \( p \), we can conclude that:

\[
(a + b)^p = a^p + b^p
\]

in \( \Z_p \) for all \( a, b \in \Z_p \).
    \end{proof}
\end{exercise}
\begin{exercise}{44} 
    An element a of a ring $R$ is idempotent if $a^2 = a$.\\
    a. Show that the set of all idempotent elements of a commutative ring is closed under multiplication.\\
    b. Find all idempotents in the ring $\Z_6 \times \Z_{12}$.\\

    
    Let's address each part individually:

\textbf{a. Show that the set of all idempotent elements of a commutative ring is closed under multiplication.}

\begin{proof}
Let \( R \) be a commutative ring, and let \( a \) and \( b \) be idempotent elements of \( R \). This means \( a^2 = a \) and \( b^2 = b \).

To show closure under multiplication for the set of idempotents, we need to show that \( ab \) is also idempotent.

Considering \( (ab)^2 \):

\[ (ab)^2 = abab = a(b^2)a = aba = a^2b = ab \]

Thus, \( ab \) is also idempotent. Therefore, the set of idempotent elements in a commutative ring is closed under multiplication.
\end{proof}

\textbf{b. Find all idempotents in the ring \( \Z_6 \times \Z_{12} \).}

For the ring \( \Z_6 \), the idempotent elements are solutions to the equation \( x^2 \equiv x \pmod{6} \). Similarly, for \( \Z_{12} \), the idempotent elements are solutions to the equation \( x^2 \equiv x \pmod{12} \).

The idempotents in \( \Z_6 \) are \( \{0, 1, 3, 4\} \), and the idempotents in \( \Z_{12} \) are \( \{0, 1, 4, 9\} \).

Now, for the ring \( \Z_6 \times \Z_{12} \), the idempotents are given by the Cartesian product of these two sets. Let's list them out.

The idempotents in the ring \( \Z_6 \times \Z_{12} \) are:

\[
\{(0, 0), (0, 1), (0, 4), (0, 9), (1, 0), (1, 1), (1, 4), (1, 9), (3, 0), (3, 1), (3, 4), (3, 9), (4, 0), (4, 1), (4, 4), (4, 9)\}
\]
\end{exercise}
\vspace*{20pt}

\begin{exercise}{46}
    An element $a$ of a ring $R$ is \textbf{nilpotent} if $a^n = 0$ for some $n \in \Z^+$. 
    Show that if $a$ and $b$ are nilpotent elements of a commutative ring, then $a + b$ is also nilpotent.

    To prove the statement, let's assume \( a \) and \( b \) are nilpotent elements in a commutative ring \( R \). This means there exist positive integers \( m \) and \( n \) such that \( a^m = 0 \) and \( b^n = 0 \).

Let \( k = \max(m, n) \). We want to show that \( (a+b)^{2k} = 0 \).

\begin{proof}
Consider the expansion of \( (a+b)^{2k} \) using the binomial theorem:

\[
(a+b)^{2k} = \sum_{i=0}^{2k} \binom{2k}{i} a^i b^{2k-i}
\]

Given that \( a^m = 0 \) and \( b^n = 0 \), any term in the above sum where \( i \geq m \) or \( 2k-i \geq n \) will be zero. 

Since \( k \) is the larger of \( m \) and \( n \), and we're considering the expansion up to the power \( 2k \), all terms will have either \( a \) raised to a power greater than or equal to \( m \) or \( b \) raised to a power greater than or equal to \( n \). Thus, every term in the expansion will be zero.

Therefore, \( (a+b)^{2k} = 0 \), which means \( a+b \) is nilpotent.
\end{proof}

So, if \( a \) and \( b \) are nilpotent elements in a commutative ring, then \( a+b \) is also nilpotent.

\end{exercise}
\vspace*{20pt}
\begin{exercise}{52}
    (Chinese Remainder Theorem for two congruences) Let $r$ and $s$ be positive integers such that $\gcd(r, s) = 1$.
Use the isomorphism in Example 18.15 to show that for $m , \ n \in Z$, there exists an integer x such that $x \equiv m
\pmod r $ and $x \equiv n \pmod s$.


To prove the Chinese Remainder Theorem for two congruences, we'll make use of the isomorphism provided in Example 18.15. This example shows that if \( \gcd(r, s) = 1 \), then the rings \( \Z_{rs} \) and \( \Z_r \times \Z_s \) are isomorphic.

Given this isomorphism, let's denote the isomorphism by \( \phi \), such that:

\[ \phi: \Z_{rs} \to \Z_r \times \Z_s \]

The isomorphism is given by:

\[ \phi(x) = (x \pmod r, x \pmod s) \]

\begin{proof}
Given the congruences:\\
1. \( x \equiv m \pmod r \)\\
2. \( x \equiv n \pmod s \)\\

We can look for an element in \( \Z_r \times \Z_s \) of the form \( (m, n) \). Since \( \phi \) is an isomorphism, there exists an element \( x \) in \( \Z_{rs} \) such that:

\[ \phi(x) = (m, n) \]

From the definition of \( \phi \), this gives:

\[ x \pmod r = m \]
\[ x \pmod s = n \]

Which are precisely the given congruences. Hence, an integer \( x \) exists such that it satisfies both congruences.

Since \( \gcd(r, s) = 1 \), the isomorphism ensures that we can find a unique solution \( x \) modulo \( rs \) that satisfies both congruences.
\end{proof}

Therefore, for any integers \( m \) and \( n \), there exists an integer \( x \) such that \( x \equiv m \pmod r \) and \( x \equiv n \pmod s \) when \( \gcd(r, s) = 1 \).
\end{exercise}
\end{document}



































