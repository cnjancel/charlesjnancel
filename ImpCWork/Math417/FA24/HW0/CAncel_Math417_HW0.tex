\documentclass[12pt]{amsart}
\usepackage[margin=1in]{geometry}
\usepackage{amssymb,amsfonts,amsmath}
\usepackage{color}
\usepackage{enumerate}
\usepackage{mathrsfs}
\usepackage{hyperref}
\usepackage[capitalise]{cleveref}
\usepackage{constants}
\usepackage{parskip}
\usepackage{indentfirst}
\usepackage{amsmath}
\usepackage{enumitem}
\setlength{\parindent}{2em}
\hfuzz=200pt

%----Table of Contents-----

%----Theorem Environments----
\newtheorem{theorem}{Theorem}[section]
\newtheorem{corollary}[theorem]{Corollary}
\newtheorem{hypothesis}[theorem]{Hypothesis}
\newtheorem{proposition}[theorem]{Proposition}
\newtheorem{lemma}[theorem]{Lemma}

\newtheorem{problem*}{Problem}

\theoremstyle{definition}
\newtheorem{definition}[theorem]{Definition}
\newtheorem{example}[theorem]{Example}
\newcommand{\exercise}[1]{\noindent {\bf Exercise #1.}}

\numberwithin{equation}{section}


\crefname{figure}{Figure}{Figures}
%MATH ENVIRONMENTS
\theoremstyle{plain}
\newtheorem*{theorem*}{Theorem}
\crefname{theorem}{Theorem}{Theorems}
\crefname{cor}{Corollary}{Corollaries}
\crefname{exercise}{Exercise}{Exercises}
\newtheorem*{cor*}{Corollary}
\crefname{cor*}{Corollary}{Corollaries}
\crefname{lem}{Lemma}{Lemmas}
\crefname{prop}{Proposition}{Propositions}
\crefname{conj}{Conjecture}{Conjectures}
\newtheorem*{conj*}{Conjecture}
\crefname{conj*}{Conjecture}{Conjectures}
\crefname{defn}{Definition}{Definitions}
\crefname{hyp}{Hypothesis}{Hypotheses}


\newcommand{\Z}{\mathbb{Z}}
\renewcommand{\C}{\mathbb{C}}
\newcommand{\R}{\mathbb{R}}
\newcommand{\Q}{\mathbb{Q}}
\newcommand{\F}{\mathbb{F}}
\newcommand{\N}{\mathbb{N}}
\newcommand{\re}{\textup{Re}}
\newcommand{\im}{\textup{Im}}
\renewcommand{\epsilon}{\varepsilon}
\newcommand{\Li}{\mathrm{Li}}


\title{Math 417, Homework 0}
\author{Charles Ancel}

%%%%%%%%%%%%%%%%%%%%%%%%%%%%%%%%%%%%%%%%%%%%%%%%%%%%%%%%%%%%%%%%%%%%%
%%%%%%%%%%%%%%%%%%%%%%%%%%%%%%%%%%%%%%%%%%%%%%%%%%%%%%%%%%%%%%%%%%%%%
%%%%%%%%%%%%%%%%%%%%%%%%%%%%%%%%%%%%%%%%%%%%%%%%%%%%%%%%%%%%%%%%%%%%%
\begin{document}
\maketitle

\section*{Chapter 1.6}
\begin{exercise}{1.6.4}
    For each of the following pairs of numbers $m$, $n$, compute $\gcd(m, n)$ and write $\gcd(m, n)$ explicitly as an integer linear combination of $m$ and $n$.
\begin{enumerate}[label=(\alph*.)]
    \item $m = 60$ and $n = 8$
    \item $m = 32242$ and $n = 42$
\end{enumerate}
    \begin{proof}
(a.) \(m = 60\) and \(n = 8\)

The Euclidean Algorithm goes as follows:
\begin{enumerate}
    \item Divide \(m\) by \(n\), and let the remainder be \(r\). 
    \item Replace \(m\) with \(n\) and \(n\) with \(r\). 
    \item Repeat until \(n\) becomes 0. The non-zero remainder is the $\gcd$
\end{enumerate}

\(60\) divided by \(8\) gives quotient \(7\) and remainder \(4\).
\(8\) divided by \(4\) gives quotient \(2\) and remainder \(0\).

Thus, $\gcd(60,8)$ = \(4\).

To express \(4\) as an integer linear combination of \(60\) and \(8\):
Start from the second to the last step:
\[ 8 = 8(1) + 60(0) \]
\[ 4 = 60 - 8(7) \]

Thus, \(4\) can be expressed as: 
\[ 4 = 60(1) - 8(7) \]

(b.) \(m = 32242\) and \(n = 42\)

Using the Euclidean Algorithm:

\[ 32242 \div 42 = 768 \text{ remainder } 6 \]
\[ 42 \div 6 = 7 \text{ remainder } 0 \]

Thus, $\gcd(32242,42)$ = \(6\).

To express \(6\) as an integer linear combination of \(32242\) and \(42\):
\[ 42 = 32242(0) + 42(1) \]
\[ 6 = 32242 - 42(768) \]

Thus, \(6\) can be expressed as: 
\[ 6 = 32242(1) - 42(768) \]
    \end{proof}
    
\end{exercise}

\begin{exercise}{1.6.9} Show that if a prime number $p$ divides a product $a_1 a_2 \dots a_r$ of nonzero integers, then $p$ divides one of the factors.
        
    \begin{proof}
\textbf{Base Case}: When \( r = 1 \), the product is just \( a_1 \). If \( p \) divides \( a_1 \), then \( p \) divides one of the factors since there's only one factor.

\textbf{Inductive Step}: 

Assume that the statement holds for some \( r = k \) such that if a prime \( p \) divides a product of \( k \) factors, then \( p \) divides at least one of these \( k \) factors. This is our inductive hypothesis.

We need to prove that it holds for \( r = k + 1 \).

Let's consider a product of \( k+1 \) integers: \( a_1 a_2 \dots a_k a_{k+1} \).

Now, if \( p \) divides one of the factors \( a_1, a_2, \dots, a_k \), then we are done by our inductive hypothesis.

If not, consider the product of the first \( k \) integers: \( a_1 a_2 \dots a_k \). Since \( p \) doesn't divide any of them individually (by our assumption), it doesn't divide their product either (by our inductive hypothesis). Let's call this product \( b \) for simplicity. So, \( p \) doesn't divide \( b \).

Now, if \( p \) divides \( b \cdot a_{k+1} \) but doesn't divide \( b \), then \( p \) must divide \( a_{k+1} \). 

And thus, for \( r = k + 1 \), if \( p \) divides the product, then it divides at least one of the factors.

By induction, the statement is true for all positive integers \( r \).

Thus, if a prime number \( p \) divides a product \( a_1 a_2 \dots a_r \) of nonzero integers, then \( p \) divides one of the factors.
        \end{proof}
\end{exercise}

\begin{exercise}{1.6.11}
    Let n1, \dots, $n_k$ be nonzero integers. Let $d =\gcd n1, \dots, n_k$,
and let
\begin{align*}
    I &=I(n1, n2, \dots, n_k)\\
      &=\{m_1n_1 + m_2n_2 + \dots m_kn_k : m1, \dots,m_k \in Z\}
\end{align*}

\begin{enumerate}[label=(\alph*.)]
    \item Show that if $x, y\in I$ , then $x + y\in I$ and $-x\in I$ . Show that if
$x\in Z$ and $a\in I$ , then $xa\in I$.
    \item Show that $\gcd(n1, n2, \dots, n_k)$ is the smallest element of $I \cap \N$.
    \item Show that $I=\Z d$.
\end{enumerate}
    \begin{proof}
        (a.) Show that if \( x, y \in I \) , then \( x + y \in I \) and \( -x \in I \) . Show that if \( x \in \Z \) and \( a \in I \) , then \( xa \in I \).
        \textit{Proof:}
        
        1. Let \( x, y \in I \). Then,
        \[ x = m_1n_1 + m_2n_2 + \dots + m_kn_k \]
        \[ y = l_1n_1 + l_2n_2 + \dots + l_kn_k \]
        for some integers \( m_1, m_2, \dots, m_k \) and \( l_1, l_2, \dots, l_k \).
        
        Adding the two:
        \[ x + y = (m_1 + l_1)n_1 + (m_2 + l_2)n_2 + \dots + (m_k + l_k)n_k \]
        The right side is an integer linear combination of \( n_1, n_2, \dots, n_k \). Thus, \( x + y \in I \).
        
        2. For \( -x \):
        \[ -x = -m_1n_1 - m_2n_2 - \dots - m_kn_k \]
        This is again an integer linear combination of \( n_1, n_2, \dots, n_k \). Thus, \( -x \in I \).
        
        3. Let \( x \in \Z \) and \( a \in I \). Then, \( a \) can be written as:
        \[ a = m_1n_1 + m_2n_2 + \dots + m_kn_k \]
        
        Multiplying by \( x \):
        \[ xa = xm_1n_1 + xm_2n_2 + \dots + xm_kn_k \]
        This is still an integer linear combination of \( n_1, n_2, \dots, n_k \). Thus, \( xa \in I \).
        
        (b.) Show that \( \gcd(n_1, n_2, \dots, n_k) \) is the smallest element of \( I \cap \N \).
        
        \textit{Proof:}
        
        By definition of the gcd, for each \( n_i \), there exists integers \( m_i \) such that:
        \[ d = m_1n_1 + m_2n_2 + \dots + m_kn_k \]
        This means \( d \in I \). Moreover, \( d \) is positive by definition of the gcd.
        
        If there exists another positive element \( d' \in I \) such that \( d' < d \), then \( d' \) would also be a common divisor of \( n_1, n_2, \dots, n_k \) which is larger than the gcd, a contradiction.
        
        Thus, \( d \) is the smallest positive element in \( I \).
        
        (c.) Show that \( I = \Z d \).
        
        \textit{Proof:}
        
        From part (b), \( d \) is the smallest positive element in \( I \). 
        
        1. \( I \subseteq \Z d \):
        Any element \( x \in I \) can be written as:
        \[ x = m_1n_1 + m_2n_2 + \dots + m_kn_k \]
        Since \( d \) divides each \( n_i \), it divides their integer combination. Therefore, \( x \) is an integer multiple of \( d \), so \( x \in \Z d \).
        
        2. \( \Z d \subseteq I \):
        Any element \( x = kd \) (for some \( k \in \Z \)) can be written using the integer combination of \( n_1, n_2, \dots, n_k \) since \( d \) is in \( I \). So, \( x \in I \).
        
        Combining these, \( I = \Z d \).
    \end{proof}
\end{exercise}

\begin{exercise}{1.6.12}
    Let $n_1, \dots, n_k$ be nonzero integers.

    \begin{enumerate}[label=(\alph*.)]
        \item Is it true that the integers $n_1, \dots, n_k$ are relatively prime if and only if they are pairwise relatively prime?
        \item Show that $n_1, \dots, n_k$ are relatively prime if and only if $1 \in I(n_1, \dots, n_k)$. 
    \end{enumerate}
    
    \begin{proof}
        Exercise 1.6.12

(a) Is it true that the integers \( n_1, \dots, n_k \) are relatively prime if and only if they are pairwise relatively prime?

Proof:

($\Rightarrow$) Suppose the integers \( n_1, \dots, n_k \) are relatively prime. This means that their greatest common divisor (gcd) is 1. If any pair of them, say \( n_i \) and \( n_j \), were not relatively prime, then their gcd would be some integer greater than 1. This would then also be a divisor of the set of numbers \( n_1, \dots, n_k \), contradicting our assumption that they are relatively prime. Hence, they must be pairwise relatively prime.

($\Leftarrow$) Suppose \( n_1, \dots, n_k \) are pairwise relatively prime. This means that for any pair \( n_i \) and \( n_j \), the \(\gcd(n_i, n_j) = 1\). Suppose, for the sake of contradiction, that the numbers \( n_1, \dots, n_k \) are not relatively prime. Then their gcd is some integer greater than 1. This gcd would be a common divisor for some pair \( n_i \) and \( n_j \), contradicting our assumption that they are pairwise relatively prime. Hence, the numbers must be relatively prime.

So, the integers \( n_1, \dots, n_k \) are relatively prime if and only if they are pairwise relatively prime.

(b) Show that \( n_1, \dots, n_k \) are relatively prime if and only if \( 1 \in I(n_1, \dots, n_k) \).

Proof:

Recall the definition of \( I(n_1, \dots, n_k) \):
\[ I = \{m_1n_1 + m_2n_2 + \dots + m_kn_k : m_1, \dots, m_k \in \Z \} \]

($\Rightarrow$) Suppose \( n_1, \dots, n_k \) are relatively prime. By the property of the gcd, there exist integers \( m_1, \dots, m_k \) such that:
\[ m_1n_1 + m_2n_2 + \dots + m_kn_k = \gcd(n_1, \dots, n_k) \]
Given they're relatively prime, this means:
\[ m_1n_1 + m_2n_2 + \dots + m_kn_k = 1 \]
This shows that \( 1 \in I(n_1, \dots, n_k) \).

($\Leftarrow$) Conversely, suppose \( 1 \in I(n_1, \dots, n_k) \). This means that there exist integers \( m_1, \dots, m_k \) such that:
\[ m_1n_1 + m_2n_2 + \dots + m_kn_k = 1 \]
Any common divisor of \( n_1, \dots, n_k \) would also be a divisor of the left side of the equation, which is 1. Hence, the only common divisor is 1. This implies that \( n_1, \dots, n_k \) are relatively prime.

Thus, \( n_1, \dots, n_k \) are relatively prime if and only if \( 1 \in I(n_1, \dots, n_k) \).

    \end{proof}
\end{exercise}
\section*{Chapter 1.7}
\begin{exercise}{1.7.4}
    Compute the congruence class modulo $12$ of $4^{237}$.
\begin{proof}
    To compute the congruence class modulo \(12\) of \(4^{237}\), we want to determine \(4^{237} \mod 12\).
    
    \textit{Proof:}
    
    Let's calculate powers of 4 modulo 12 to find a pattern:
    
    \(4^1 \equiv 4 \mod 12\)
    
    \(4^2 \equiv 16 \equiv 4 \mod 12\)
    
    \(4^3 \equiv 64 \equiv 4 \mod 12\)
    
    And so on...
    
    We can see that higher powers of 4 are always congruent to 4 modulo 12. Therefore:
    
    \[4^{237} \equiv 4 \mod 12\]
    
    Thus, the congruence class modulo \(12\) of \(4^{237}\) is \(4\).
\end{proof}
\end{exercise}
\begin{exercise}{1.7.5}
    Can an element of $\Z_n$ be both invertible and a zero divisor?
\begin{proof} 
    No, an element of \( \Z_n \) cannot be both invertible and a zero divisor.

    Let's break down the definitions:
    
    1. An element \( a \) in \( \Z_n \) is \textit{invertible} if there exists an element \( b \) in \( \Z_n \) such that:
    \[ a \cdot b \equiv 1 \mod n \]
    That is, \( a \) has a multiplicative inverse modulo \( n \).
    
    2. An element \( a \) in \( \Z_n \) is a \textit{zero divisor} if \( a \neq 0 \) and there exists a nonzero element \( b \) in \( \Z_n \) such that:
    \[ a \cdot b \equiv 0 \mod n \]
    
    Let's suppose, for contradiction, that there exists an element \( a \) in \( \Z_n \) which is both invertible and a zero divisor.
    
    Then, by definition, there exist elements \( b \) and \( c \) (with \( c \neq 0 \)) in \( \Z_n \) such that:
    
    \[ a \cdot b \equiv 1 \mod n \]
    \[ a \cdot c \equiv 0 \mod n \]
    
    Now, let's multiply the second equation by \( b \):
    \[ a \cdot c \cdot b \equiv 0 \mod n \]
    Given that \( a \cdot b \equiv 1 \mod n \), this gives:
    \[ c \equiv 0 \mod n \]
    But this contradicts our definition of a zero divisor, where \( c \) is supposed to be nonzero.
    
    Thus, our initial assumption that an element could be both invertible and a zero divisor is incorrect. An element of \( \Z_n \) cannot be both invertible and a zero divisor.
\end{proof}
\end{exercise}

\begin{exercise}{1.7.11}
    \begin{proof} $ $\newline

    \begin{enumerate}
        \item If \( a \) and \( n \) are relatively prime, by Bezout's Lemma, there are integers \( s \) and \( t \) such that: \[ as + nt = 1 \]
        \item Considering this equation in the modular context, taking both sides modulo \( n \) yields: \[ as \equiv 1 \mod n \]
    \end{enumerate}
    
    This means that, within the modular arithmetic system modulo \( n \), multiplying \( a \) by \( s \) results in a remainder of 1 when divided by \( n \).
    \begin{enumerate}
        \item In \( \Z_n \), this congruence implies that the class \( [a] \) has an inverse, namely \( [s] \), because their product is \( [1] \), the multiplicative identity in \( \Z_n \).
        \item Therefore, the fact that \( a \) is relatively prime to \( n \) ensures the existence of its multiplicative inverse in \( \Z_n \). Specifically, \( [a] \) is invertible in \( \Z_n \) and its inverse is \( [s] \).
    \end{enumerate}
    \end{proof}
\end{exercise}

\begin{exercise}{1.7.14a}
    Suppose $a$ is relatively prime to $n$.\\
    (a.) Show that for all $b\in \Z$, the congruence $ax \equiv b \mod n$ has a solution.
    \begin{proof}
If \( a \) is relatively prime to \( n \), then by Bezout's Lemma, there exist integers \( s \) and \( t \) such that:
\[ as + nt = 1 \]

Taking this equation modulo \( n \), we get:
\[ as \equiv 1 \mod n \]

This means \( s \) is the multiplicative inverse of \( a \) modulo \( n \). Now, to find a solution for \( ax \equiv b \mod n \), we can multiply both sides of this congruence by \( s \).

\[ s(ax) \equiv s(b) \mod n \]
\[ a(sx) \equiv sb \mod n \]

Given that \( as \equiv 1 \mod n \), this equation becomes:
\[ x \equiv sb \mod n \]

Therefore, \( x = sb \) is a solution to the congruence \( ax \equiv b \mod n \). Since \( b \) was an arbitrary integer from \( \Z \), this proves that for all \( b \in \Z \), the congruence \( ax \equiv b \mod n \) has a solution.
    \end{proof}
\end{exercise}
\begin{exercise}{1.7.14c}
    Suppose $a$ is relatively prime to $n$.\\
    (c.) Solve the congruence $8x \equiv 12 \mod 125$.

    \begin{proof}
        To solve the congruence \( 8x \equiv 12 \mod 125 \), we first want to determine if 8 is relatively prime to 125. 
        
        125 is equal to \( 5^3 \). Since 8 and 5 are relatively prime (their greatest common divisor is 1), 8 is relatively prime to 125.
        
        This means we can find an inverse for 8 modulo 125. This inverse, when multiplied by 8, will be congruent to 1 mod 125.
        
        Using the Extended Euclidean Algorithm, we can find integers \( s \) and \( t \) such that \( 8s + 125t = 1 \). (I'll spare the full details of the algorithm here for brevity.) For this congruence, the multiplicative inverse of 8 mod 125 is 47.
        
        Given this inverse, to solve the original congruence, we multiply both sides by 47:
        \[ 8x \equiv 12 \mod 125 \]
        \[ 47(8x) \equiv 47(12) \mod 125 \]
        \[ x \equiv 564 \mod 125 \]
        
        Breaking 564 down mod 125, we have \( 564 = 4(125) + 64 \). Thus:
        \[ x \equiv 64 \mod 125 \]
        
        So, the solution to the congruence \( 8x \equiv 12 \mod 125 \) is \( x \equiv 64 \mod 125 \).
    \end{proof}
    \end{exercise}
\end{document}