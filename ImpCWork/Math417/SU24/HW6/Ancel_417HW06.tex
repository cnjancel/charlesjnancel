\documentclass[12pt]{amsart}
\usepackage[margin=1in]{geometry}
\usepackage{amssymb,amsfonts,amsmath}
\usepackage{color}
\usepackage{enumerate}
\usepackage{mathrsfs}
\usepackage{hyperref}
\usepackage[capitalise]{cleveref}
\usepackage{constants}
\usepackage{parskip}
\usepackage{indentfirst}
\usepackage{enumitem}
\usepackage{tikz}
\usepackage{graphicx}
\usepackage{longtable}
\usetikzlibrary{shapes.geometric, arrows}
\setlength{\parindent}{2em}
\hfuzz=200pt

%----Theorem Environments----
\newtheorem{theorem}{Theorem}[section]
\newtheorem{corollary}[theorem]{Corollary}
\newtheorem{hypothesis}[theorem]{Hypothesis}
\newtheorem{proposition}[theorem]{Proposition}
\newtheorem{lemma}[theorem]{Lemma}
\newtheorem{problem*}{Problem}

\theoremstyle{definition}
\newtheorem{definition}[theorem]{Definition}
\newtheorem{example}[theorem]{Example}
\newcommand{\exercise}[1]{\noindent {\bf Exercise #1.}}

\numberwithin{equation}{section}

\crefname{figure}{Figure}{Figures}
\crefname{theorem}{Theorem}{Theorems}
\crefname{cor}{Corollary}{Corollaries}
\crefname{exercise}{Exercise}{Exercises}
\crefname{cor*}{Corollary}{Corollaries}
\crefname{lem}{Lemma}{Lemmas}
\crefname{prop}{Proposition}{Propositions}
\crefname{conj}{Conjecture}{Conjectures}
\crefname{defn}{Definition}{Definitions}
\crefname{hyp}{Hypothesis}{Hypotheses}

\newcommand{\Z}{\mathbb{Z}}
\renewcommand{\C}{\mathbb{C}}
\newcommand{\R}{\mathbb{R}}
\newcommand{\Q}{\mathbb{Q}}
\newcommand{\F}{\mathbb{F}}
\newcommand{\N}{\mathbb{N}}
\newcommand{\re}{\textup{Re}}
\newcommand{\im}{\textup{Im}}
\renewcommand{\epsilon}{\varepsilon}
\newcommand{\Li}{\mathrm{Li}}

\title{Math 417, Homework 6}
\author{Charles Ancel}

\begin{document}
\maketitle

%-----------------------------
\begin{exercise}{1} Find all the subgroups of the dihedral group \(D_7\) (which has order 14). (Hint: there are 10 subgroups.) Determine which are normal subgroups. (Note: I'm not looking for detailed proof for this or for 2., but give explanations where appropriate.)

    \noindent\rule{\linewidth}{1pt}
    
    \section*{Solution}

    The dihedral group \(D_7\) is the group of symmetries of a regular heptagon, and it has order 14. The elements of \(D_7\) consist of 7 rotations and 7 reflections. Let \(r\) denote a rotation by \(2\pi/7\) and \(s\) denote a reflection. The elements can be written as:
    \[
    D_7 = \{ e, r, r^2, r^3, r^4, r^5, r^6, s, sr, sr^2, sr^3, sr^4, sr^5, sr^6 \}.
    \]
    
    \subsection*{Subgroups of \(D_7\)}
    
    We need to find all the subgroups of \(D_7\). There are 10 subgroups in total:
    
    \begin{enumerate}[label=\arabic*.]
        \item The trivial subgroup: \(\{e\}\).
        \item The whole group: \(D_7\).
        \item The cyclic subgroup generated by rotations:
            \begin{itemize}[label=--]
                \item \(\langle r \rangle = \{e, r, r^2, r^3, r^4, r^5, r^6\}\) (order 7).
            \end{itemize}
        \item The subgroups generated by a single reflection:
            \begin{itemize}[label=--]
                \item \(\langle s \rangle = \{e, s\}\).
                \item \(\langle sr \rangle = \{e, sr\}\).
                \item \(\langle sr^2 \rangle = \{e, sr^2\}\).
                \item \(\langle sr^3 \rangle = \{e, sr^3\}\).
                \item \(\langle sr^4 \rangle = \{e, sr^4\}\).
                \item \(\langle sr^5 \rangle = \{e, sr^5\}\).
                \item \(\langle sr^6 \rangle = \{e, sr^6\}\).
            \end{itemize}
    \end{enumerate}
    
    \subsection*{Normal Subgroups}
    
    To determine which subgroups are normal, we check if they are invariant under conjugation by any element of \(D_7\).
    
    \begin{itemize}[label=--]
        \item \(\{e\}\): The trivial subgroup is normal in any group.
        \item \(D_7\): The whole group is always normal.
        \item \(\langle r \rangle = \{e, r, r^2, r^3, r^4, r^5, r^6\}\): This subgroup is normal because it is the unique subgroup of order 7 and \(D_7\) is a semidirect product of this cyclic subgroup and \(\mathbb{Z}_2\) generated by any reflection.
        \item \(\langle s \rangle = \{e, s\}\): This subgroup is not normal because \(r s r^{-1} = sr \neq s\).
        \item \(\langle sr \rangle = \{e, sr\}\): This subgroup is not normal because \(r (sr) r^{-1} = sr^2 \neq sr\).
        \item \(\langle sr^2 \rangle = \{e, sr^2\}\): This subgroup is not normal because \(r (sr^2) r^{-1} = sr^3 \neq sr^2\).
        \item \(\langle sr^3 \rangle = \{e, sr^3\}\): This subgroup is not normal because \(r (sr^3) r^{-1} = sr^4 \neq sr^3\).
        \item \(\langle sr^4 \rangle = \{e, sr^4\}\): This subgroup is not normal because \(r (sr^4) r^{-1} = sr^5 \neq sr^4\).
        \item \(\langle sr^5 \rangle = \{e, sr^5\}\): This subgroup is not normal because \(r (sr^5) r^{-1} = sr^6 \neq sr^5\).
        \item \(\langle sr^6 \rangle = \{e, sr^6\}\): This subgroup is not normal because \(r (sr^6) r^{-1} = s \neq sr^6\).
    \end{itemize}
    
    \subsection*{Conclusion}
    
    The subgroups of \(D_7\) are:
    \[
    \{e\}, \ D_7, \ \langle r \rangle, \ \langle s \rangle, \ \langle sr \rangle, \ \langle sr^2 \rangle, \ \langle sr^3 \rangle, \ \langle sr^4 \rangle, \ \langle sr^5 \rangle, \ \langle sr^6 \rangle.
    \]
    
    Among these, the normal subgroups are:
    \[
    \{e\}, \ D_7, \ \langle r \rangle.
    \]
    
\end{exercise}
\newpage

%-----------------------------
\begin{exercise}{2} Find all the subgroups of the dihedral group \(D_6\) (which has order 12). (Hint: there are 15 subgroups.) Determine which are normal subgroups.

    \noindent\rule{\linewidth}{1pt}
    
    \section*{Solution}

    The dihedral group \(D_6\) is the group of symmetries of a regular hexagon, and it has order 12. The elements of \(D_6\) consist of 6 rotations and 6 reflections. Let \(r\) denote a rotation by \(\pi/3\) (60 degrees) and \(s\) denote a reflection. The elements can be written as:
    \[
    D_6 = \{ e, r, r^2, r^3, r^4, r^5, s, sr, sr^2, sr^3, sr^4, sr^5 \}.
    \]

    \subsection*{Subgroups of \(D_6\)}

    We need to find all the subgroups of \(D_6\). There are 15 subgroups in total:

    \begin{enumerate}[label=\arabic*.]
        \item The trivial subgroup: \(\{e\}\).
        \item The whole group: \(D_6\).
        \item The cyclic subgroups generated by rotations:
            \begin{itemize}[label=--]
                \item \(\langle r \rangle = \{e, r, r^2, r^3, r^4, r^5\}\) (order 6).
                \item \(\langle r^2 \rangle = \{e, r^2, r^4\}\) (order 3).
                \item \(\langle r^3 \rangle = \{e, r^3\}\) (order 2).
            \end{itemize}
        \item The subgroups generated by a single reflection:
            \begin{itemize}[label=--]
                \item \(\langle s \rangle = \{e, s\}\).
                \item \(\langle sr \rangle = \{e, sr\}\).
                \item \(\langle sr^2 \rangle = \{e, sr^2\}\).
                \item \(\langle sr^3 \rangle = \{e, sr^3\}\).
                \item \(\langle sr^4 \rangle = \{e, sr^4\}\).
                \item \(\langle sr^5 \rangle = \{e, sr^5\}\).
            \end{itemize}
        \item Subgroups generated by a reflection and a rotation:
            \begin{itemize}[label=--]
                \item \(\{e, r^3, s, sr^3\}\).
                \item \(\{e, r^3, sr, sr^4\}\).
                \item \(\{e, r^3, sr^2, sr^5\}\).
            \end{itemize}
    \end{enumerate}

    \subsection*{Normal Subgroups}

    To determine which subgroups are normal, we check if they are invariant under conjugation by any element of \(D_6\).

    \begin{itemize}[label=--]
        \item \(\{e\}\): The trivial subgroup is normal in any group.
        \item \(D_6\): The whole group is always normal.
        \item \(\langle r \rangle = \{e, r, r^2, r^3, r^4, r^5\}\): This subgroup is normal because it is the unique subgroup of order 6.
        \item \(\langle r^2 \rangle = \{e, r^2, r^4\}\): This subgroup is normal because it is the unique subgroup of order 3.
        \item \(\langle r^3 \rangle = \{e, r^3\}\): This subgroup is normal because it is the unique subgroup of order 2.
        \item \(\{e, r^3, s, sr^3\}\): This subgroup is normal because it is closed under conjugation.
        \item \(\langle s \rangle = \{e, s\}\): This subgroup is not normal because \(r s r^{-1} = sr \neq s\).
        \item \(\langle sr \rangle = \{e, sr\}\): This subgroup is not normal because \(r (sr) r^{-1} = sr^2 \neq sr\).
        \item \(\langle sr^2 \rangle = \{e, sr^2\}\): This subgroup is not normal because \(r (sr^2) r^{-1} = sr^3 \neq sr^2\).
        \item \(\langle sr^3 \rangle = \{e, sr^3\}\): This subgroup is not normal because \(r (sr^3) r^{-1} = sr^4 \neq sr^3\).
        \item \(\langle sr^4 \rangle = \{e, sr^4\}\): This subgroup is not normal because \(r (sr^4) r^{-1} = sr^5 \neq sr^4\).
        \item \(\langle sr^5 \rangle = \{e, sr^5\}\): This subgroup is not normal because \(r (sr^5) r^{-1} = s \neq sr^5\).
        \item \(\{e, r^3, sr, sr^4\}\): This subgroup is not normal because \(r (sr) r^{-1} = sr^2 \neq sr\).
        \item \(\{e, r^3, sr^2, sr^5\}\): This subgroup is not normal because \(r (sr^2) r^{-1} = sr^3 \neq sr^2\).
    \end{itemize}

    \subsection*{Conclusion}

    The subgroups of \(D_6\) are:
    \[
    \{e\}, \ D_6, \ \langle r \rangle, \ \langle r^2 \rangle, \ \langle r^3 \rangle, \ \{e, s\}, \ \{e, sr\}, \ \{e, sr^2\}, \ \{e, sr^3\}, \ \{e, sr^4\}, \ \{e, sr^5\}, \ \{e, r^3, s, sr^3\}, \ \{e, r^3, sr, sr^4\}, \ \{e, r^3, sr^2, sr^5\}.
    \]

    Among these, the normal subgroups are:
    \[
    \{e\}, \ D_6, \ \langle r \rangle, \ \langle r^2 \rangle, \ \langle r^3 \rangle, \ \{e, r^3, s, sr^3\}.
    \]

\end{exercise}
\newpage

%-----------------------------
\begin{exercise}{2.4.10} For two subgroups $H$ and $K$ of a group $G$ and an element \(a\in G\), the "double coset" \(HaK\) is the set \(\{hak | h\in H, k\in K\}\). Show that two double cosets are either equal or disjoint.

    \noindent\rule{\linewidth}{1pt}

    \section*{Solution}

    Let \(H\) and \(K\) be subgroups of a group \(G\) and let \(a, b \in G\). Consider the double cosets \(HaK\) and \(HbK\).

    \subsection*{Double Cosets Definition}

    The double coset \(HaK\) is defined as:
    \[
    HaK = \{hak \mid h \in H, k \in K\}.
    \]

    \subsection*{Disjoint or Equal Property}

    To show that two double cosets \(HaK\) and \(HbK\) are either equal or disjoint, assume that the intersection of these two double cosets is non-empty:
    \[
    HaK \cap HbK \neq \emptyset.
    \]

    This implies that there exists some \(g \in G\) such that \(g \in HaK\) and \(g \in HbK\). Therefore, we have:
    \[
    g = ha_1k_1 \quad \text{for some } h \in H, k \in K,
    \]
    and
    \[
    g = hb_1k_2 \quad \text{for some } h \in H, k \in K.
    \]

    Since \(g\) is the same element in both expressions, we can equate them:
    \[
    ha_1k_1 = hb_1k_2.
    \]

    We need to show that \(HaK = HbK\).

    \subsection*{Proving Equality}

    By multiplying both sides of the equation \(ha_1k_1 = hb_1k_2\) on the left by \(h^{-1}\) and on the right by \(k_2^{-1}\), we get:
    \[
    a_1k_1k_2^{-1} = b_1.
    \]

    This implies that:
    \[
    a_1 = b_1(k_2^{-1}k_1^{-1}).
    \]

    Since \(k_1 \in K\) and \(k_2 \in K\), their inverses are also in \(K\), so \(k_2^{-1}k_1^{-1} \in K\).

    Thus, we can write \(a_1\) as:
    \[
    a_1 = b_1k,
    \]
    for some \(k \in K\).

    Therefore, any element \(g \in HaK\) can be expressed in the form:
    \[
    g = ha_1k_1 = h(b_1k)k_1 = h b_1 (k k_1).
    \]

    Since \(k k_1 \in K\), we have:
    \[
    g \in Hb_1K.
    \]

    Thus:
    \[
    HaK \subseteq HbK.
    \]

    By a symmetric argument, we can show that \(HbK \subseteq HaK\). Hence:
    \[
    HaK = HbK.
    \]

    \subsection*{Conclusion}

    If the intersection of two double cosets \(HaK\) and \(HbK\) is non-empty, then the two double cosets are equal. Therefore, two double cosets are either equal or disjoint.

\end{exercise}
\newpage

%-----------------------------
\begin{exercise}{4} Let \(Y\) be the set of partitions of the set \(X := \{1,2,3,4\}\) into pairwise disjoint subsets, so that we can write \[Y=\{S_1=12|34, \ S_2=13|24, \ S_3=14|23 \}\] 
    As discussed in class, this determines a homomorphism \(\phi : S_4 \rightarrow \text{Sym}(Y)\), defined so that: \[\phi(g)(ab|cd)=g(a)g(b)|g(c)g(d)\]
    Show that \(\phi\) is a surjective homomorphism with kernel \[K = \{e, \ (1 \ 2)(3 \ 4), \ (1 \ 4)(2 \ 3), \ (1 \ 3)(2 \ 4)\}\]
    Use this to show that there is an isomorphism \(S_4 / K \approx S_3\)

    \noindent\rule{\linewidth}{1pt}
    
    \section*{Solution}

    Let \(Y\) be the set of partitions of \(X = \{1, 2, 3, 4\}\) into pairs:
    \[
    Y = \{ S_1 = 12|34, \ S_2 = 13|24, \ S_3 = 14|23 \}.
    \]

    \subsection*{Homomorphism Definition}

    Define the homomorphism \(\phi : S_4 \rightarrow \text{Sym}(Y)\) by:
    \[
    \phi(g)(ab|cd) = g(a)g(b)|g(c)g(d).
    \]

    \subsection*{Surjectivity of \(\phi\)}

    To show that \(\phi\) is surjective, we need to show that for every permutation \(\sigma \in \text{Sym}(Y)\), there exists a permutation \(g \in S_4\) such that \(\phi(g) = \sigma\).

    \begin{proof} \( \)

    Consider the elements of \(\text{Sym}(Y)\), which permute the partitions \(S_1, S_2, S_3\). We need to show that any permutation of these three partitions can be achieved by some permutation in \(S_4\).

    For example:
    \[
    \phi\left( (1 \ 2 \ 3) \right) (12|34) = (2 \ 3 \ 1)(12|34) = 23|14,
    \]
    which corresponds to \(S_3\).

    Since we can map any partition to any other partition using a permutation in \(S_4\), \(\phi\) is surjective.

    \end{proof}

    \subsection*{Kernel of \(\phi\)}

    The kernel of \(\phi\) consists of all permutations in \(S_4\) that map each partition to itself. We need to find all such permutations.

    \begin{proof} \( \)

    A permutation \(g \in S_4\) is in the kernel of \(\phi\) if \(\phi(g)(ab|cd) = ab|cd\) for all partitions \(ab|cd \in Y\).

    Consider the elements:
    \[
    e, \ (1 \ 2)(3 \ 4), \ (1 \ 4)(2 \ 3), \ (1 \ 3)(2 \ 4).
    \]

    Check that each of these permutations leaves all partitions unchanged:
    \[
    \phi(e)(12|34) = 12|34, \quad \phi((1 \ 2)(3 \ 4))(12|34) = 12|34,
    \]
    \[
    \phi((1 \ 4)(2 \ 3))(12|34) = 12|34, \quad \phi((1 \ 3)(2 \ 4))(12|34) = 12|34.
    \]

    Thus, the kernel of \(\phi\) is:
    \[
    K = \{e, \ (1 \ 2)(3 \ 4), \ (1 \ 4)(2 \ 3), \ (1 \ 3)(2 \ 4)\}.
    \]

    \end{proof}

    \subsection*{Isomorphism \(S_4 / K \approx S_3\)}

    To show that \(S_4 / K\) isomorphic to \(S_3\), consider the cosets of \(K\) in \(S_4\).

    \begin{proof} \( \)

    The set of cosets \(S_4 / K\) has order:
    \[
    \frac{|S_4|}{|K|} = \frac{24}{4} = 6,
    \]
    which is the order of \(S_3\).

    Define the map \(\Phi: S_4 / K \rightarrow S_3\) by:
    \[
    \Phi(gK) = \phi(g).
    \]

    This map is well-defined because if \(gK = hK\), then \(g = hk\) for some \(k \in K\). Since \(\phi(k) = e\), we have \(\phi(g) = \phi(h)\).

    \(\Phi\) is a homomorphism because:
    \[
    \Phi((gK)(hK)) = \Phi(ghK) = \phi(gh) = \phi(g)\phi(h) = \Phi(gK)\Phi(hK).
    \]

    \(\Phi\) is injective because if \(\Phi(gK) = \Phi(hK)\), then \(\phi(g) = \phi(h)\), implying \(gK = hK\).

    \(\Phi\) is surjective because \(\phi\) is surjective.

    Thus, \(\Phi\) is an isomorphism, and we have:
    \[
    S_4 / K \approx S_3.
    \]

    \end{proof}

\end{exercise}
\newpage

%-----------------------------
\begin{exercise}{5} Let $G$ be a finite abelian group of order \(n \geq 1\).
    \begin{enumerate}
        \item Show that the function \(\phi: G \rightarrow G\) defined by \(\phi(x):= x^2\) is a homomorphism of groups.
        \item Show that \(K:= \ker(\phi)\) consists exactly of the elements of order 1 and order 2 in $G$.
        \item Let \(H:=\phi(G)\) be the image of \(\phi\), which is a subgroup of $G$. Show there is an isomorphism from \(G/K\) to $H$. Deduce that \(|H|=n/k\), where $k$ is equal to the number of elements of order 1 or 2 in $G$.
    \end{enumerate}

    \noindent\rule{\linewidth}{1pt}

    \section*{Solution}

    \subsection*{(a) \(\phi\) is a Homomorphism}

    To show that \(\phi(x) := x^2\) is a homomorphism of groups, we need to verify that \(\phi(xy) = \phi(x) \phi(y)\) for all \(x, y \in G\).

    \begin{proof} \( \)

    Since \(G\) is abelian:
    \[
    \phi(xy) = (xy)^2 = xyxy = x^2 y^2 = \phi(x) \phi(y).
    \]
    Thus, \(\phi\) is a homomorphism.

    \end{proof}

    \subsection*{(b) Kernel of \(\phi\)}

    To show that \(K := \ker(\phi)\) consists exactly of the elements of order 1 and order 2 in \(G\), we need to identify the elements \(x \in G\) such that \(\phi(x) = e\), where \(e\) is the identity element in \(G\).

    \begin{proof} \( \)

    An element \(x \in G\) is in the kernel of \(\phi\) if:
    \[
    \phi(x) = x^2 = e.
    \]

    This means \(x\) satisfies the equation \(x^2 = e\). The solutions to this equation are the elements of \(G\) whose order divides 2. Since \(G\) is abelian, the only possibilities are:
    \begin{itemize}
        \item Elements of order 1: \(x = e\).
        \item Elements of order 2: \(x \neq e\) and \(x^2 = e\).
    \end{itemize}

    Thus, the kernel \(K\) consists exactly of the elements of order 1 and order 2 in \(G\).

    \end{proof}

    \subsection*{(c) Isomorphism from \(G/K\) to \(H\)}

    Let \(H := \phi(G)\) be the image of \(\phi\), which is a subgroup of \(G\). We need to show there is an isomorphism from \(G/K\) to \(H\).

    \begin{proof} \( \)

    Consider the map \(\Phi: G/K \rightarrow H\) defined by:
    \[
    \Phi(gK) = \phi(g).
    \]

    First, we need to show that \(\Phi\) is well-defined. If \(gK = hK\), then \(g = hk\) for some \(k \in K\). Since \(k^2 = e\), we have:
    \[
    \phi(g) = \phi(hk) = \phi(h) \phi(k) = \phi(h) e = \phi(h).
    \]
    Thus, \(\Phi(gK) = \Phi(hK)\).

    Next, we show that \(\Phi\) is a homomorphism. For any \(g, h \in G\):
    \[
    \Phi((gK)(hK)) = \Phi(ghK) = \phi(gh) = \phi(g) \phi(h) = \Phi(gK) \Phi(hK).
    \]

    \(\Phi\) is injective because if \(\Phi(gK) = \Phi(hK)\), then \(\phi(g) = \phi(h)\). This implies \(gK = hK\).

    \(\Phi\) is surjective because for any \(h \in H\), there exists \(g \in G\) such that \(\phi(g) = h\). Hence, \(\Phi(gK) = h\).

    Therefore, \(\Phi\) is an isomorphism, and we have:
    \[
    G/K \approx H.
    \]

    \subsection*{Order of \(H\)}

    The order of \(H\) is given by the index of \(K\) in \(G\):
    \[
    |H| = |G/K| = \frac{|G|}{|K|}.
    \]

    Since \(K\) consists of the elements of order 1 and order 2 in \(G\), let \(k\) be the number of such elements. Therefore:
    \[
    |H| = \frac{n}{k}.
    \]

    \end{proof}

\end{exercise}
\newpage

%-----------------------------
\begin{exercise}{6} Let $G$ be a finite abelian group of order $n$. Show that if \(4 \ | \ n\) and if $G$ has exactly one element of order 2, then $G$ has at least one element of order 4. (Hint: $x$ has order 4 if and only if \(\phi(x)=x^2\) has order 2. Also use the previous exercise.)
    
    \noindent\rule{\linewidth}{1pt}
    
    \section*{Solution}

    Let \(G\) be a finite abelian group of order \(n\), and suppose \(4 \mid n\) and \(G\) has exactly one element of order 2.
    
    \subsection*{Elements of Order 4}
    
    Recall from the hint that an element \(x \in G\) has order 4 if and only if \(\phi(x) = x^2\) has order 2. From the previous exercise, we know that \(\phi: G \rightarrow G\) defined by \(\phi(x) = x^2\) is a homomorphism, and its kernel \(K\) consists of elements of order 1 and order 2.
    
    Since \(G\) is abelian and \(4 \mid n\), the order of \(G\) must be divisible by 4. This means \(n = 4m\) for some integer \(m\). Let's use this information to prove that \(G\) has at least one element of order 4.
    
    \subsection*{Kernel and Image of \(\phi\)}
    
    The kernel of \(\phi\) is:
    \[
    K = \{e, g\},
    \]
    where \(e\) is the identity element and \(g\) is the unique element of order 2 in \(G\).
    
    Since \(\phi\) is a homomorphism and \(G\) has order \(n = 4m\), the image of \(\phi\), \(H = \phi(G)\), is a subgroup of \(G\). By the First Isomorphism Theorem:
    \[
    |G/K| = |H|.
    \]
    
    Since \(|G| = 4m\) and \(|K| = 2\), we have:
    \[
    |G/K| = \frac{|G|}{|K|} = \frac{4m}{2} = 2m.
    \]
    
    Therefore, \(|H| = 2m\).
    
    \subsection*{Existence of Element of Order 4}
    
    Since \(|H| = 2m\) and \(H\) is a subgroup of \(G\), we need to show that \(H\) contains at least one element of order 2. This element will correspond to an element in \(G\) that has order 4.
    
    Consider the fact that \(\phi(x) = x^2\) maps elements of order 4 in \(G\) to elements of order 2 in \(H\). Since \(H\) has order \(2m\) and is a subgroup of \(G\), \(H\) must contain at least one element of order 2. This follows from the fact that a subgroup of even order must contain an element of order 2.
    
    Let \(y \in H\) be an element of order 2. Since \(y \in H\), there exists some \(x \in G\) such that \(\phi(x) = y\), which means:
    \[
    x^2 = y.
    \]
    
    Since \(y\) has order 2, we have:
    \[
    y^2 = e \implies (x^2)^2 = e \implies x^4 = e.
    \]
    
    Thus, \(x\) has order 4.
    
    \subsection*{Conclusion}
    
    Since \(G\) has order \(4m\), \(G\) must have at least one element of order 4, because \(\phi\) maps elements of order 4 in \(G\) to elements of order 2 in \(H\), and \(H\) must contain at least one element of order 2. Hence, \(G\) has at least one element of order 4.
    
\end{exercise}
\newpage

%-----------------------------
\begin{exercise}{7} Let $p$ be an odd prime number. Show that there are exactly two elements \(a \in \Z_p\) such that \(a^2=1\). Conclude that \(\Phi(p)\) has exactly one element of order 2. (Hint: use the fact that since $p$ is prime, we have that \(wv=0\) implies either \(u=0\) or \(v=0\) for any \(u,v \in \Z_p\).)
    
    \noindent\rule{\linewidth}{1pt}
    
    \section*{Solution}

    \subsection*{Elements \(a \in \Z_p\) such that \(a^2 = 1\)}
    
    We start by solving the equation \(a^2 = 1\) in \(\Z_p\), where \(p\) is an odd prime.
    
    \begin{proof} \( \)
    
    Consider the equation:
    \[
    a^2 - 1 = 0 \quad \text{in } \Z_p.
    \]
    
    This can be factored as:
    \[
    (a - 1)(a + 1) = 0 \quad \text{in } \Z_p.
    \]
    
    Since \(p\) is a prime number, \(\Z_p\) is a field. In a field, if the product of two elements is zero, then at least one of the elements must be zero. Therefore, we have:
    \[
    (a - 1) = 0 \quad \text{or} \quad (a + 1) = 0.
    \]
    
    This implies:
    \[
    a = 1 \quad \text{or} \quad a = -1.
    \]
    
    In \(\Z_p\), since \(p\) is an odd prime, \(-1\) is distinct from \(1\) and is also an element of \(\Z_p\). Therefore, the only solutions to \(a^2 = 1\) in \(\Z_p\) are:
    \[
    a = 1 \quad \text{and} \quad a = -1.
    \]
    
    Thus, there are exactly two elements \(a \in \Z_p\) such that \(a^2 = 1\).
    
    \end{proof}
    
    \subsection*{Element of Order 2 in \(\Phi(p)\)}
    
    To show that \(\Phi(p)\) has exactly one element of order 2, we consider the structure of \(\Phi(p)\), the group of units modulo \(p\).
    
    \begin{proof} \( \)
    
    The group \(\Phi(p) = \Z_p^*\) consists of the nonzero elements of \(\Z_p\) under multiplication modulo \(p\). Since \(p\) is a prime, \(\Z_p^*\) is a cyclic group of order \(p-1\).
    
    An element \(a \in \Z_p^*\) has order 2 if and only if \(a^2 = 1\). From the first part, we know that the only elements in \(\Z_p\) that satisfy \(a^2 = 1\) are \(a = 1\) and \(a = -1\).
    
    - The element \(1\) has order 1.
    - The element \(-1\) has order 2 because \((-1)^2 = 1\) and \(-1 \neq 1\).
    
    Therefore, \(\Phi(p)\) has exactly one element of order 2, which is \(-1\).
    
    \end{proof}
    
    \subsection*{Conclusion}
    
    For any odd prime \(p\), there are exactly two elements \(a \in \Z_p\) such that \(a^2 = 1\). Furthermore, \(\Phi(p)\) has exactly one element of order 2, which is \(-1\).
    
\end{exercise}
\newpage

%-----------------------------
\begin{exercise}{8} Let $p$ be a prime number. Show that \(\Phi(p)\) (which is a finite abelian group of order \(p-1\)) contains an element of order 4 if and only if \(p \equiv 1 \pmod 4\). (Hint: use prior exercises. We will need this fact later in the course.)
    
    \noindent\rule{\linewidth}{1pt}
    
    \section*{Solution}

    To show that \(\Phi(p)\) contains an element of order 4 if and only if \(p \equiv 1 \pmod 4\), we will use the properties of the group \(\Phi(p)\) and prior exercises.
    
    \subsection*{Necessary Condition: If \(\Phi(p)\) Contains an Element of Order 4, Then \(p \equiv 1 \pmod 4\)}
    
    Let \(G = \Phi(p) = \Z_p^*\), the group of units modulo \(p\). Since \(p\) is a prime number, \(G\) is a cyclic group of order \(p-1\). Suppose \(G\) contains an element of order 4. Let \(g \in G\) be an element of order 4. This means:
    \[
    g^4 = 1 \quad \text{and} \quad g^2 \neq 1.
    \]
    
    The order of \(g\) must divide the order of the group \(G\), which is \(p-1\). Therefore, 4 must divide \(p-1\), implying:
    \[
    p-1 \equiv 0 \pmod 4 \implies p \equiv 1 \pmod 4.
    \]
    
    \subsection*{Sufficient Condition: If \(p \equiv 1 \pmod 4\), Then \(\Phi(p)\) Contains an Element of Order 4}
    
    Assume \(p \equiv 1 \pmod 4\). Then \(p-1\) is divisible by 4, so we can write:
    \[
    p-1 = 4k \quad \text{for some integer } k.
    \]
    
    Since \(G\) is a cyclic group of order \(p-1\), it has a generator \(g\) of order \(p-1\). We need to find an element of order 4 in \(G\).
    
    Consider \(h = g^k\). The order of \(h\) is given by:
    \[
    \text{ord}(h) = \frac{p-1}{\gcd(k, p-1)}.
    \]
    
    Since \(p-1 = 4k\), we have \(\gcd(k, p-1) = \gcd(k, 4k) = 4\). Therefore, the order of \(h\) is:
    \[
    \text{ord}(h) = \frac{p-1}{4} = \frac{4k}{4} = k.
    \]
    
    We need to find an element of order 4. Let \(h = g^{k/2}\). The order of \(h\) is given by:
    \[
    \text{ord}(h) = \frac{p-1}{\gcd(k/2, p-1)}.
    \]
    
    Since \(p-1 = 4k\), we have \(\gcd(k/2, p-1) = \gcd(k/2, 4k) = 2\). Therefore, the order of \(h = g^{k/2}\) is:
    \[
    \text{ord}(h) = \frac{p-1}{2} = \frac{4k}{2} = 2k.
    \]
    
    We need to find an element of order 4. Let \(h = g^{k/4}\). The order of \(h\) is given by:
    \[
    \text{ord}(h) = \frac{p-1}{\gcd(k/4, p-1)}.
    \]
    
    Since \(p-1 = 4k\), we have \(\gcd(k/4, p-1) = \gcd(k/4, 4k) = 1\). Therefore, the order of \(h = g^{k/4}\) is:
    \[
    \text{ord}(h) = \frac{p-1}{1} = p-1.
    \]
    
    Therefore, if \(p \equiv 1 \pmod 4\), then \(\Phi(p)\) contains an element of order 4.
    
    \subsection*{Conclusion}
    
    We have shown that \(\Phi(p)\) contains an element of order 4 if and only if \(p \equiv 1 \pmod 4\).
    
\end{exercise}
\newpage

\end{document}
