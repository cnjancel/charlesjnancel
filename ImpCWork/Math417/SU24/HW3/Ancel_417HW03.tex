\documentclass[12pt]{amsart}
\usepackage[margin=1in]{geometry}
\usepackage{amssymb,amsfonts,amsmath}
\usepackage{color}
\usepackage{enumerate}
\usepackage{mathrsfs}
\usepackage{hyperref}
\usepackage[capitalise]{cleveref}
\usepackage{constants}
\usepackage{parskip}
\usepackage{indentfirst}
\usepackage{enumitem}
\usepackage{tikz}
\usepackage{graphicx}
\usepackage{longtable}
\usetikzlibrary{shapes.geometric, arrows}
\setlength{\parindent}{2em}
\hfuzz=200pt

%----Theorem Environments----
\newtheorem{theorem}{Theorem}[section]
\newtheorem{corollary}[theorem]{Corollary}
\newtheorem{hypothesis}[theorem]{Hypothesis}
\newtheorem{proposition}[theorem]{Proposition}
\newtheorem{lemma}[theorem]{Lemma}
\newtheorem{problem*}{Problem}

\theoremstyle{definition}
\newtheorem{definition}[theorem]{Definition}
\newtheorem{example}[theorem]{Example}
\newcommand{\exercise}[1]{\noindent {\bf Exercise #1.}}

\numberwithin{equation}{section}

\crefname{figure}{Figure}{Figures}
\crefname{theorem}{Theorem}{Theorems}
\crefname{cor}{Corollary}{Corollaries}
\crefname{exercise}{Exercise}{Exercises}
\crefname{cor*}{Corollary}{Corollaries}
\crefname{lem}{Lemma}{Lemmas}
\crefname{prop}{Proposition}{Propositions}
\crefname{conj}{Conjecture}{Conjectures}
\crefname{defn}{Definition}{Definitions}
\crefname{hyp}{Hypothesis}{Hypotheses}

\newcommand{\Z}{\mathbb{Z}}
\renewcommand{\C}{\mathbb{C}}
\newcommand{\R}{\mathbb{R}}
\newcommand{\Q}{\mathbb{Q}}
\newcommand{\F}{\mathbb{F}}
\newcommand{\N}{\mathbb{N}}
\newcommand{\re}{\textup{Re}}
\newcommand{\im}{\textup{Im}}
\renewcommand{\epsilon}{\varepsilon}
\newcommand{\Li}{\mathrm{Li}}

\title{Math 417, Homework 3}
\author{Charles Ancel}

\begin{document}
\maketitle

%-----------------------------
\begin{exercise}{1.7.6 From Last Week} 
If an element \(\Z_n\) has a multiplicative inverse, is that multiplicative inverse unique? That is, if \([a]\) is invertible, can there be two distinct elements \([b]\) and \([c]\) of \(\Z_n\) such that \([a][b] = [1]\) and \([a][c] = [1]\)?

\noindent\rule{\linewidth}{1pt}

\section*{Solution}

We need to show whether the multiplicative inverse of an element in \(\mathbb{Z}_n\) is unique. Specifically, we want to determine if there can be two distinct elements \([b]\) and \([c]\) in \(\mathbb{Z}_n\) such that \([a][b] = [1]\) and \([a][c] = [1]\).

\begin{proof} \( \)

Assume that \([a]\) is an invertible element in \(\mathbb{Z}_n\). Suppose there exist two elements \([b]\) and \([c]\) in \(\mathbb{Z}_n\) such that:
\[
[a][b] = [1] \quad \text{and} \quad [a][c] = [1].
\]

This implies:
\[
ab \equiv 1 \pmod{n} \quad \text{and} \quad ac \equiv 1 \pmod{n}.
\]

\subsection*{Showing \(b\) and \(c\) Must Be Congruent Modulo \(n\)}
Since \([a]\) has a multiplicative inverse, there exists some integer \(k\) such that:
\[
ab = 1 + kn \quad \text{for some integer } k.
\]
Similarly, there exists some integer \(m\) such that:
\[
ac = 1 + mn \quad \text{for some integer } m.
\]

We need to show that \(b \equiv c \pmod{n}\).

Starting from the equations \(ab \equiv 1 \pmod{n}\) and \(ac \equiv 1 \pmod{n}\), we have:
\[
ab \equiv ac \pmod{n}.
\]

Since \(a \in \mathbb{Z}_n\) and \(a\) is invertible, there exists an inverse \([a]^{-1}\) such that:
\[
[a][a]^{-1} = [1].
\]

Multiplying both sides of \(ab \equiv ac \pmod{n}\) by \([a]^{-1}\), we get:
\[
b \equiv c \pmod{n}.
\]

\subsection*{Conclusion}
Thus, \(b\) and \(c\) are congruent modulo \(n\), meaning \([b] = [c]\) in \(\mathbb{Z}_n\). Therefore, the multiplicative inverse of \([a]\) in \(\mathbb{Z}_n\) is unique.

\end{proof}
\end{exercise}
\newpage

%-----------------------------
\begin{exercise}{2} 
Let \(p, q, r\) be distinct primes, and let \(n = pqr\) and \(m = (p-1)(q-1)(r-1)\). Show that for any \(a \in \Z\) and \(h \in \N\) that 
\[ h \equiv 1 \pmod{m} \text{ implies } a^h \equiv a \pmod{n}\]

\noindent\rule{\linewidth}{1pt}

\section*{Solution}

To show that \(a^h \equiv a \pmod{n}\) if \(h \equiv 1 \pmod{m}\), where \(m = (p-1)(q-1)(r-1)\) and \(n = pqr\), we will use the properties of modular arithmetic and Euler's theorem.

\begin{proof} \( \)

Let \(a \in \mathbb{Z}\) and \(h \in \mathbb{N}\) such that \(h \equiv 1 \pmod{m}\). That is, there exists an integer \(k\) such that:
\[
h = 1 + km.
\]

We need to show that:
\[
a^h \equiv a \pmod{n}.
\]

\subsection*{Application of Euler's Theorem}
Euler's theorem states that if \(a\) is coprime to \(p\), \(q\), and \(r\) (i.e., \(\gcd(a, p) = \gcd(a, q) = \gcd(a, r) = 1\)), then:
\[
a^{p-1} \equiv 1 \pmod{p}, \quad a^{q-1} \equiv 1 \pmod{q}, \quad a^{r-1} \equiv 1 \pmod{r}.
\]

Since \(m = (p-1)(q-1)(r-1)\), we know that \(m\) is a common multiple of \((p-1)\), \((q-1)\), and \((r-1)\). Therefore:
\[
a^m \equiv 1 \pmod{p}, \quad a^m \equiv 1 \pmod{q}, \quad a^m \equiv 1 \pmod{r}.
\]

Since \(h = 1 + km\), we have:
\[
a^h = a^{1 + km} = a \cdot (a^m)^k.
\]

Using the property \(a^m \equiv 1 \pmod{p}\), \(a^m \equiv 1 \pmod{q}\), and \(a^m \equiv 1 \pmod{r}\):
\[
a^h \equiv a \cdot 1^k \equiv a \pmod{p}, \quad a^h \equiv a \cdot 1^k \equiv a \pmod{q}, \quad a^h \equiv a \cdot 1^k \equiv a \pmod{r}.
\]

\subsection*{Combining Results}
By the Chinese Remainder Theorem (CRT), since \(p\), \(q\), and \(r\) are distinct primes, the congruences:
\[
a^h \equiv a \pmod{p}, \quad a^h \equiv a \pmod{q}, \quad a^h \equiv a \pmod{r}
\]
imply that:
\[
a^h \equiv a \pmod{pqr}.
\]

Therefore:
\[
a^h \equiv a \pmod{n}.
\]

\subsection*{Conclusion}
For any \(a \in \mathbb{Z}\) and \(h \in \mathbb{N}\), if \(h \equiv 1 \pmod{m}\), then \(a^h \equiv a \pmod{n}\), where \(n = pqr\) and \(m = (p-1)(q-1)(r-1)\).

\end{proof}

\end{exercise}
\newpage

%-----------------------------
\begin{exercise}{1.9.10} 
Let \(p\) be a prime number, \(k \in \Z\), and \(s \in \Z_{\geq 0}\). Show that 
\[
(1+kp)^{p^s} \equiv 1 + kp^{s+1} \pmod{p^{s+1}}
\]

(Hint: induction on \(s\).)

\noindent\rule{\linewidth}{1pt}

\section*{Solution}

We will use mathematical induction on \(s\) to prove that 
\[
(1+kp)^{p^s} \equiv 1 + kp^{s+1} \pmod{p^{s+1}}.
\]

\begin{proof} \( \)

\subsection*{Base Case \(s = 0\)}
For \(s = 0\):
\[
(1 + kp)^{p^0} = (1 + kp)^1 = 1 + kp.
\]
We need to show:
\[
1 + kp \equiv 1 + kp \pmod{p}.
\]
This is trivially true since both sides are identical.

\subsection*{Inductive Step}
Assume the statement holds for some \(s = n\), that is:
\[
(1 + kp)^{p^n} \equiv 1 + kp^{n+1} \pmod{p^{n+1}}.
\]
We need to show it holds for \(s = n+1\), i.e.:
\[
(1 + kp)^{p^{n+1}} \equiv 1 + kp^{n+2} \pmod{p^{n+2}}.
\]

Using the inductive hypothesis:
\[
(1 + kp)^{p^n} = 1 + kp^{n+1} + p^{n+1}x \quad \text{for some integer } x.
\]

Raise both sides to the power \(p\):
\[
(1 + kp)^{p^{n+1}} = \left[(1 + kp)^{p^n}\right]^p.
\]
Expanding using the binomial theorem:
\[
\left[1 + kp^{n+1} + p^{n+1}x\right]^p = \sum_{i=0}^p \binom{p}{i} (1 + kp^{n+1})^{p-i} (p^{n+1}x)^i.
\]

Simplify the terms modulo \(p^{n+2}\):
- For \(i = 0\):
\[
\binom{p}{0} (1 + kp^{n+1})^p (p^{n+1}x)^0 = (1 + kp^{n+1})^p.
\]

Using the binomial theorem again:
\[
(1 + kp^{n+1})^p = 1 + p(kp^{n+1}) + \binom{p}{2}(kp^{n+1})^2 + \cdots \equiv 1 + kp^{n+2} \pmod{p^{n+2}}.
\]

- For \(i > 0\):
\[
\binom{p}{i} (1 + kp^{n+1})^{p-i} (p^{n+1}x)^i \equiv 0 \pmod{p^{n+2}}
\]
since \(p^{n+1}x\) is divisible by \(p^{n+1}\) and thus \(p^{n+2}\).

Summing all terms modulo \(p^{n+2}\):
\[
(1 + kp)^{p^{n+1}} = (1 + kp)^{p^n \cdot p} = 1 + kp^{n+2} \pmod{p^{n+2}}.
\]

\subsection*{Conclusion}
By induction, the statement holds for all \(s \geq 0\):
\[
(1 + kp)^{p^s} \equiv 1 + kp^{s+1} \pmod{p^{s+1}}.
\]

\end{proof}

\end{exercise}
\newpage

%-----------------------------
\begin{exercise}{4} 
Compute the multiplication table for \(\Phi(8)\).

\noindent\rule{\linewidth}{1pt}

\section*{Solution}

\subsection*{Finding Elements of \(\mathbb{Z}_8^*\)}

The set \(\mathbb{Z}_8^*\) (the group of units modulo 8) consists of the integers less than 8 that are coprime to 8. An integer \(a\) is coprime to 8 if \(\gcd(a, 8) = 1\).

We check each integer from 1 to 7:
\begin{itemize}
    \item \( \gcd(1, 8) = 1 \implies 1 \in \mathbb{Z}_8^*\)
    \item \( \gcd(2, 8) = 2 \implies 2 \not\in \mathbb{Z}_8^*\)
    \item \( \gcd(3, 8) = 1 \implies 3 \in \mathbb{Z}_8^*\)
    \item \( \gcd(4, 8) = 4 \implies 4 \not\in \mathbb{Z}_8^*\)
    \item \( \gcd(5, 8) = 1 \implies 5 \in \mathbb{Z}_8^*\)
    \item \( \gcd(6, 8) = 2 \implies 6 \not\in \mathbb{Z}_8^*\)
    \item \( \gcd(7, 8) = 1 \implies 7 \in \mathbb{Z}_8^*\)
\end{itemize}

Thus, the elements of \(\mathbb{Z}_8^*\) are:
\[
\mathbb{Z}_8^* = \{1, 3, 5, 7\}.
\]

\subsection*{Multiplication Table for \(\mathbb{Z}_8^*\)}

We compute the products of these elements modulo 8:

\[
\begin{array}{c|cccc}
  \cdot & 1 & 3 & 5 & 7 \\
  \hline
  1 & 1 & 3 & 5 & 7 \\
  3 & 3 & 9 \equiv 1 & 15 \equiv 7 & 21 \equiv 5 \\
  5 & 5 & 15 \equiv 7 & 25 \equiv 1 & 35 \equiv 3 \\
  7 & 7 & 21 \equiv 5 & 35 \equiv 3 & 49 \equiv 1 \\
\end{array}
\]

The multiplication table for \(\mathbb{Z}_8^*\) is:

\[
\begin{array}{c|cccc}
  \cdot & 1 & 3 & 5 & 7 \\
  \hline
  1 & 1 & 3 & 5 & 7 \\
  3 & 3 & 1 & 7 & 5 \\
  5 & 5 & 7 & 1 & 3 \\
  7 & 7 & 5 & 3 & 1 \\
\end{array}
\]

\end{exercise}
\newpage

%-----------------------------
\begin{exercise}{5} 
Let \(G\) be the set of all functions \(\N \rightarrow \N\).
\begin{enumerate}[label=\textbf{\alph*.}]
    \item Show that \((G, \circ)\), where \(f, g \mapsto f \circ g\) is the operation of composition of functions, is a monoid, but not a group.

    \item Give an example of elements \(f, g \in G\) such that \(f \circ g = \text{id}\), but \(g \circ f \neq \text{id}\). (So an element in a monoid with a "one-sided inverse" might not have an actual inverse.)
\end{enumerate}

\noindent\rule{\linewidth}{1pt}

\section*{Solution}

\subsection*{\textbf{a. Showing that \((G, \circ)\) is a Monoid but not a Group}}

\begin{proof} \( \)

To show that \((G, \circ)\) is a monoid, we need to verify two properties: associativity and the existence of an identity element.

\subsubsection*{Associativity}
Let \(f, g, h \in G\). We need to show that function composition is associative:
\[
f \circ (g \circ h) = (f \circ g) \circ h.
\]
For any \(x \in \mathbb{N}\):
\[
f \circ (g \circ h)(x) = f(g(h(x))) = (f \circ g)(h(x)) = (f \circ g) \circ h (x).
\]
Since this holds for all \(x \in \mathbb{N}\), function composition is associative.

\subsubsection*{Identity Element}
The identity function \(\text{id} \in G\) is defined by:
\[
\text{id}(x) = x \quad \text{for all } x \in \mathbb{N}.
\]
We need to show that \(\text{id}\) is the identity element for function composition:
\[
f \circ \text{id} = f \quad \text{and} \quad \text{id} \circ f = f.
\]
For any \(x \in \mathbb{N}\):
\[
f \circ \text{id}(x) = f(\text{id}(x)) = f(x),
\]
\[
\text{id} \circ f (x) = \text{id}(f(x)) = f(x).
\]
Since these hold for all \(f \in G\), \(\text{id}\) is the identity element.

\subsubsection*{Not a Group}
To show that \((G, \circ)\) is not a group, we need to show that there exists at least one element in \(G\) that does not have an inverse.

Consider the function \(f \in G\) defined by:
\[
f(x) = x+1.
\]
Assume there exists \(g \in G\) such that \(f \circ g = \text{id}\) and \(g \circ f = \text{id}\). For \(f \circ g = \text{id}\), we need:
\[
f(g(x)) = x \implies g(x)+1 = x \implies g(x) = x-1.
\]
For \(g \circ f = \text{id}\), we need:
\[
g(f(x)) = x \implies g(x+1) = x \implies (x+1)-1 = x.
\]
However, \(g(x) = x-1\) is not a valid function from \(\mathbb{N}\) to \(\mathbb{N}\) because it maps 0 to -1, which is not in \(\mathbb{N}\). Therefore, \(f\) does not have an inverse in \(G\).

Thus, \((G, \circ)\) is a monoid but not a group.

\end{proof}

\subsection*{\textbf{b. Example of One-Sided Inverse}}

\begin{proof} \( \)

Consider the functions \(f, g \in G\) defined by:
\[
f(x) = \begin{cases}
0 & \text{if } x = 0, \\
x-1 & \text{if } x > 0.
\end{cases}
\]
\[
g(x) = x+1.
\]
We show that \(f \circ g = \text{id}\) but \(g \circ f \neq \text{id}\).

\subsubsection*{Verification}
For \(f \circ g\):
\[
(f \circ g)(x) = f(g(x)) = f(x+1) = (x+1)-1 = x \quad \text{for all } x \in \mathbb{N}.
\]
Thus:
\[
f \circ g = \text{id}.
\]

For \(g \circ f\):
\[
(g \circ f)(x) = g(f(x)) = g \left(\begin{cases}
0 & \text{if } x = 0, \\
x-1 & \text{if } x > 0.
\end{cases}\right) = \begin{cases}
g(0) = 1 & \text{if } x = 0, \\
g(x-1) = x & \text{if } x > 0.
\end{cases}
\]
Thus:
\[
g \circ f(0) = 1 \neq 0 \quad \text{and} \quad g \circ f(x) = x \quad \text{for } x > 0.
\]

\subsubsection*{Conclusion}
We have \(f \circ g = \text{id}\) but \(g \circ f \neq \text{id}\). This demonstrates that an element in a monoid with a one-sided inverse might not have an actual inverse.

\end{proof}
\end{exercise}
\newpage

%-----------------------------
\begin{exercise}{2.1.10} 
Show that any group with four elements must have a nonidentity element whose square is the identity. That is, some nonidentity element must be its own inverse. (See description of problem in Goodman for hints.)

\noindent\rule{\linewidth}{1pt}

\section*{Solution}

Let \(G\) be a group with four elements. We need to show that there is a nonidentity element \(a \in G\) such that \(a^2 = e\), where \(e\) is the identity element in \(G\).

\section*{Structure of Group \(G\)}

Let the elements of \(G\) be \(\{e, a, b, c\}\), where \(e\) is the identity element.

\subsection*{Case 1: All Nonidentity Elements Are Their Own Inverse}

\begin{itemize}
    \item Suppose \(a^2 = e\), \(b^2 = e\), and \(c^2 = e\).
    \item In this case, each nonidentity element is its own inverse.
    \item Hence, there is nothing more to show.
\end{itemize}

\subsection*{Case 2: Some Element Does Not Have Its Square Equal to the Identity}

\begin{itemize}
    \item Suppose \(a^2 \neq e\).
    \item Then \(a \neq a^{-1}\), so the inverse \(a^{-1}\) must be one of the other nonidentity elements. Without loss of generality, let \(b = a^{-1}\).
    \[
    b = a^{-1} \quad \text{and} \quad a = b^{-1}.
    \]
    \item We then have:
    \[
    a^2 = ab = e \quad \text{and} \quad b^2 = ba = e.
    \]
\end{itemize}

\section*{Determine the Structure of \(c\)}

\begin{itemize}
    \item The element \(c\) must also satisfy \(c = c^{-1}\). We must check the possible squares of \(c\):
    \begin{itemize}
        \item If \(c \neq e\) and \(c \neq a\) and \(c \neq b\), it implies \(c^2 = e\).
        \item For \(c\), we check the possibility of it forming any additional elements, but since \(G\) is closed and there are only 4 elements, \(c\) must be either \(a\) or \(b\) already included in \(G\), or its own inverse.
    \end{itemize}
\end{itemize}

\section*{Verification with Concrete Examples}

\subsection*{Cyclic Group \(\mathbb{Z}_4\)}

\begin{itemize}
    \item The elements are \(\{0, 1, 2, 3\}\) with addition modulo 4.
    \item The elements \(1\) and \(3\) are their own inverses because:
    \[
    1 + 1 = 2 \equiv 0 \pmod{4}, \quad 3 + 1 = 4 \equiv 0 \pmod{4}.
    \]
\end{itemize}

\subsection*{Klein Four-Group \(V_4\)}

\begin{itemize}
    \item The elements are \(\{e, r_1, r_2, r_3\}\), where each \(r_i\) (nonidentity) has its square equal to \(e\).
\end{itemize}

\section*{Conclusion}

In both possible cases, at least one nonidentity element of the group \(G\) has its square equal to the identity element. Therefore, any group with four elements must have a nonidentity element that is its own inverse.

\end{exercise}
\newpage

%-----------------------------

\end{document}