\documentclass[12pt]{amsart}
\usepackage[margin=1in]{geometry}
\usepackage{amssymb,amsfonts,amsmath}
\usepackage{color}
\usepackage{enumerate}
\usepackage{mathrsfs}
\usepackage{hyperref}
\usepackage[capitalise]{cleveref}
\usepackage{constants}
\usepackage{parskip}
\usepackage{indentfirst}
\usepackage{enumitem}
\usepackage{tikz}
\usepackage{graphicx}
\usepackage{longtable}
\usetikzlibrary{shapes.geometric, arrows}
\setlength{\parindent}{2em}
\hfuzz=200pt

%----Theorem Environments----
\newtheorem{theorem}{Theorem}[section]
\newtheorem{corollary}[theorem]{Corollary}
\newtheorem{hypothesis}[theorem]{Hypothesis}
\newtheorem{proposition}[theorem]{Proposition}
\newtheorem{lemma}[theorem]{Lemma}
\newtheorem{problem*}{Problem}

\theoremstyle{definition}
\newtheorem{definition}[theorem]{Definition}
\newtheorem{example}[theorem]{Example}
\newcommand{\exercise}[1]{\noindent {\bf Exercise #1.}}

\numberwithin{equation}{section}

\crefname{figure}{Figure}{Figures}
\crefname{theorem}{Theorem}{Theorems}
\crefname{cor}{Corollary}{Corollaries}
\crefname{exercise}{Exercise}{Exercises}
\crefname{cor*}{Corollary}{Corollaries}
\crefname{lem}{Lemma}{Lemmas}
\crefname{prop}{Proposition}{Propositions}
\crefname{conj}{Conjecture}{Conjectures}
\crefname{defn}{Definition}{Definitions}
\crefname{hyp}{Hypothesis}{Hypotheses}

\newcommand{\Z}{\mathbb{Z}}
\renewcommand{\C}{\mathbb{C}}
\newcommand{\R}{\mathbb{R}}
\newcommand{\Q}{\mathbb{Q}}
\newcommand{\F}{\mathbb{F}}
\newcommand{\N}{\mathbb{N}}
\newcommand{\re}{\textup{Re}}
\newcommand{\im}{\textup{Im}}
\renewcommand{\epsilon}{\varepsilon}
\newcommand{\Li}{\mathrm{Li}}

\title{Math 417, Homework 2}
\author{Charles Ancel}

\begin{document}
\maketitle

\begin{exercise}{\textbf{1}} Consider the formula \(x \star y \stackrel{\text{def}}{=} 2xy \). Show that \((\R_{>0}, \star)\) is a group.
    
    \noindent\rule{\linewidth}{1pt}

    \vspace*{-10pt}\section*{Solution}
    To show that \((\R_{>0}, \star)\) is a group under the operation \(x \star y = 2xy\), we need to verify the following group axioms:
\begin{proof} \( \)
\subsection*{Closure}
Let \(x, y \in \R_{>0}\). We need to show that \(x \star y = 2xy \in \R_{>0}\).
\begin{align*}
x &> 0 \quad \text{and} \quad y > 0 \\
\Rightarrow 2xy &> 0 \quad \text{since } 2 > 0, \, x > 0, \, y > 0.
\end{align*}
Thus, \(x \star y \in \R_{>0}\), proving closure.

\subsection*{Associativity}
Let \(x, y, z \in \R_{>0}\). We need to show that \((x \star y) \star z = x \star (y \star z)\).
\begin{align*}
(x \star y) \star z &= (2xy) \star z \\
                    &= 2(2xy)z \quad \text{(by definition of } \star\text{)} \\
                    &= 4xyz, \\
x \star (y \star z) &= x \star (2yz) \\
                    &= 2x(2yz) \quad \text{(by definition of } \star\text{)} \\
                    &= 4xyz.
\end{align*}
Thus, \((x \star y) \star z = x \star (y \star z)\), proving associativity.

\subsection*{Identity Element}
We need to find an element \(e \in \R_{>0}\) such that \(e \star x = x \star e = x\) for all \(x \in \R_{>0}\).
\begin{align*}
e \star x &= 2ex = x \quad \Rightarrow e = \frac{1}{2}, \\
x \star e &= 2x \left(\frac{1}{2}\right) = x.
\end{align*}
Thus, the identity element is \(e = \frac{1}{2}\).

\subsection*{Inverses}
For each \(x \in \R_{>0}\), we need to find an element \(y \in \R_{>0}\) such that \(x \star y = y \star x = e\), where \(e = \frac{1}{2}\).
\begin{align*}
x \star y &= 2xy = \frac{1}{2} \\
\Rightarrow y &= \frac{1}{4x}, \\
y \star x &= 2y x = 2 \left(\frac{1}{4x}\right) x = \frac{1}{2}.
\end{align*}
Thus, the inverse of \(x\) is \(y = \frac{1}{4x}\).

\end{proof}

\noindent Since the operation \(\star\) satisfies closure, associativity, identity, and inverses, we conclude that \((\R_{>0}, \star)\) is a group.
\newpage
\end{exercise}
\newpage
\begin{exercise}{\textbf{2}} State and prove a formula for the parity of a permutation in terms of its cycle type.

\noindent\rule{\linewidth}{1pt}

\vspace*{-10pt}\section*{Solution}

\subsection*{Statement of the Formula}
Let \(\sigma \in S_n\) be a permutation, and let \(c_1, c_2, \ldots, c_k\) be the lengths of the disjoint cycles in the cycle decomposition of \(\sigma\). The parity of the permutation \(\sigma\) (i.e., whether it is even or odd) can be determined by the formula:
\[
\text{parity}(\sigma) = \sum_{i=1}^{k} (c_i - 1) \mod 2.
\]
In other words, a permutation is even if the sum of \(c_i - 1\) for all cycles is even, and odd if this sum is odd.

\subsection*{Proof of the Formula}

To prove this formula, we need to show that the sum \(\sum_{i=1}^{k} (c_i - 1)\) modulo 2 corresponds to the parity of the permutation \(\sigma\).

\begin{proof} \(\)
    \section*{}\vspace*{-40pt}
    \subsection*{Cycle Decomposition and Transpositions}
Any permutation \(\sigma \in S_n\) can be written as a product of disjoint cycles. Let \(\sigma = \tau_1 \tau_2 \cdots \tau_k\), where each \(\tau_i\) is a cycle of length \(c_i\).

A \(c_i\)-cycle \((a_1 \, a_2 \, \ldots \, a_{c_i})\) can be decomposed into \((c_i - 1)\) transpositions:
\[
(a_1 \, a_2 \, \ldots \, a_{c_i}) = (a_1 \, a_{c_i})(a_1 \, a_{c_i-1}) \cdots (a_1 \, a_2).
\]
The number of transpositions in the decomposition of \(\tau_i\) is \(c_i - 1\).

 \subsection*{Sum of Transpositions}
The total number of transpositions required to express \(\sigma\) as a product of transpositions is:
\[
\sum_{i=1}^{k} (c_i - 1).
\]

 \subsection*{Parity Calculation}
The parity of \(\sigma\) is even if this sum is even and odd if this sum is odd. This can be seen as follows:
- If the total number of transpositions (2-cycles) used to express \(\sigma\) is even, then \(\sigma\) is an even permutation.
- If the total number of transpositions used to express \(\sigma\) is odd, then \(\sigma\) is an odd permutation.

Thus, the parity of the permutation \(\sigma\) can be calculated as:
\[
\text{parity}(\sigma) = \sum_{i=1}^{k} (c_i - 1) \mod 2.
\]

 \subsection*{Example for Verification}
Consider the permutation \(\sigma = (1 \, 2 \, 3)(4 \, 5)\) in \(S_5\):
\begin{itemize}[label=--]
    \item The cycle \((1 \, 2 \, 3)\) is a 3-cycle and can be decomposed into 2 transpositions: \((1 \, 3)(1 \, 2)\).
    \item The cycle \((4 \, 5)\) is a 2-cycle and can be decomposed into 1 transposition: \((4 \, 5)\).
\end{itemize}
The total number of transpositions is \(2 + 1 = 3\). Thus, \(\sigma\) is an odd permutation. According to the formula:
\[
\text{parity}(\sigma) = ((3-1) + (2-1)) \mod 2 = (2 + 1) \mod 2 = 1.
\]
The parity is 1 (odd), matching our calculation.

\end{proof}

\noindent This proves that the formula for the parity of a permutation in terms of its cycle type is correct.
\end{exercise}
\newpage
\begin{exercise}{\textbf{3}} Suppose \(b,c,d,e\) are integers. Show that if \(d,e \in I(b,c)\), then \(I(d,e) \subseteq I(b,c)\)
    
    \noindent\rule{\linewidth}{1pt}

    \section*{Solution}

Let \(I(b,c)\) denote the ideal in \(\mathbb{Z}\) generated by \(b\) and \(c\), and \(I(d,e)\) denote the ideal generated by \(d\) and \(e\). To show that \(I(d,e) \subseteq I(b,c)\), we proceed as follows:

\begin{proof} \(\)
    \section*{}\vspace*{-20pt}
\subsection*{Definition of Ideals}
The ideal \(I(b,c)\) generated by \(b\) and \(c\) in \(\mathbb{Z}\) consists of all linear combinations of \(b\) and \(c\):
\[
I(b,c) = \{xb + yc \mid x, y \in \mathbb{Z}\}.
\]
Similarly, the ideal \(I(d,e)\) generated by \(d\) and \(e\) is:
\[
I(d,e) = \{ud + ve \mid u, v \in \mathbb{Z}\}.
\]

\subsection*{Given Conditions}
We are given that \(d, e \in I(b,c)\). Therefore, there exist integers \(x_1, y_1, x_2, y_2\) such that:
\[
d = x_1 b + y_1 c,
\]
\[
e = x_2 b + y_2 c.
\]

\subsection*{Subset Inclusion}
We need to show that any element in \(I(d,e)\) is also in \(I(b,c)\). Take any element in \(I(d,e)\):
\[
k = ud + ve \quad \text{for some } u, v \in \mathbb{Z}.
\]
Substituting \(d = x_1 b + y_1 c\) and \(e = x_2 b + y_2 c\):
\[
k = u(x_1 b + y_1 c) + v(x_2 b + y_2 c).
\]
Expanding and simplifying, we get:
\[
k = (ux_1 + vx_2)b + (uy_1 + vy_2)c.
\]
Let \(x = ux_1 + vx_2\) and \(y = uy_1 + vy_2\). Then:
\[
k = xb + yc.
\]

\subsection*{Conclusion}
Since \(x, y \in \mathbb{Z}\), we have \(k = xb + yc\). This shows that \(k \in I(b,c)\). Therefore, every element \(k \in I(d,e)\) is also in \(I(b,c)\), proving that:
\[
I(d,e) \subseteq I(b,c).
\]

\end{proof}
\end{exercise}
\newpage
\begin{exercise}{\textbf{4}} Let \(a,b,d \in \Z\). Show that if the equation \(d = ax+by\) has at least one solution \((x,y) \in \Z^2\), then it has infinitely many such solutions.
    
    \noindent\rule{\linewidth}{1pt}

    \section*{Solution}

Given the equation \(d = ax + by\) with \(a, b, d \in \mathbb{Z}\), we aim to show that if there exists at least one integer solution \((x_0, y_0)\), then there are infinitely many integer solutions.


\begin{proof}\(\)
    \section*{}\vspace*{-20pt}
\subsection*{Existence of a Particular Solution}
Suppose there exists a solution \((x_0, y_0) \in \mathbb{Z}^2\) such that:
\[
d = ax_0 + by_0.
\]

\subsection*{General Solution}
We will derive the general solution to the equation \(d = ax + by\).

Consider any integer \(t\). Let \(x\) and \(y\) be parameterized by \(t\):
\[
x = x_0 + bt, \quad y = y_0 - at.
\]

\subsection*{Verification of the General Solution}
Substitute \(x = x_0 + bt\) and \(y = y_0 - at\) into the equation \(d = ax + by\):
\begin{align*}
d &= a(x_0 + bt) + b(y_0 - at) \\
  &= ax_0 + abt + by_0 - bat \\
  &= ax_0 + by_0 + (abt - bat) \\
  &= ax_0 + by_0 + 0 \\
  &= d.
\end{align*}
Thus, \(d = ax + by\) for \(x = x_0 + bt\) and \(y = y_0 - at\).

\subsection*{Conclusion}
Since \(t\) can be any integer, there are infinitely many pairs \((x, y)\) given by:
\[
x = x_0 + bt, \quad y = y_0 - at
\]
that satisfy the equation \(d = ax + by\).

\noindent Therefore, if there is at least one solution to the equation \(d = ax + by\), there are infinitely many integer solutions.
\end{proof}
\end{exercise}
\newpage
\begin{exercise}{\textbf{1.6.3}} Suppose that a natural number \(p > 1\) has the property that for all nonzero integers $a$, $b$, if $p$ divides the product \(ab\), then $p$ divides $a$ or $p$ divides $b$. Show that $p$ is prime. (This is the converse of Proposition 1.6.19 in the book.)
    
    \noindent\rule{\linewidth}{1pt}

    \section*{Solution}

To prove that \(p\) is a prime number, we assume the given property and use a proof by contradiction.

\begin{proof}\(\)
    \section*{}\vspace*{-20pt}
\subsection*{Assumption}
Assume \(p\) is not prime. Then \(p\) can be factored into two positive integers greater than 1:
\[
p = mn,
\]
where \(1 < m < p\) and \(1 < n < p\).

\subsection*{Applying the Property}
Consider the integers \(m\) and \(n\):
\[
p \mid mn \quad \text{(since } p = mn\text{)}.
\]
By the given property, since \(p\) divides the product \(mn\), \(p\) must divide either \(m\) or \(n\).

\subsection*{Contradiction}
Since \(p = mn\):
\begin{itemize}[label=--]
    \item If \(p \mid m\), then \(m = kp\) for some integer \(k\), which implies \(m \geq p\), contradicting \(1 < m < p\).
    \item If \(p \mid n\), then \(n = kp\) for some integer \(k\), which implies \(n \geq p\), contradicting \(1 < n < p\).
\end{itemize}

In both cases, we reach a contradiction. Therefore, our assumption that \(p\) is not prime must be false.

\subsection*{Conclusion}
Since assuming \(p\) is not prime leads to a contradiction, we conclude that \(p\) must be prime.

\end{proof}
\end{exercise}
\newpage
\begin{exercise}{\textbf{1.7.1}} Prove that addition and multiplication in \(\Z_n\) are both commutative and associative.
    
    \noindent\rule{\linewidth}{1pt}

    \vspace*{-10pt}\section*{Solution}

In \(\mathbb{Z}_n\), addition and multiplication are defined modulo \(n\). We will prove the commutativity and associativity of these operations.

\subsection*{Addition in \(\mathbb{Z}_n\)}

\subsubsection*{Commutativity}
Let \(a, b \in \mathbb{Z}_n\). We need to show that \(a + b = b + a\) in \(\mathbb{Z}_n\).
\begin{proof} \( \)

Addition in \(\mathbb{Z}_n\) is defined as:
\[
a + b \equiv a + b \pmod{n}.
\]
Using the commutativity of addition in \(\mathbb{Z}\), we have:
\[
a + b = b + a.
\]
Therefore:
\[
a + b \equiv b + a \pmod{n}.
\]
Hence, addition in \(\mathbb{Z}_n\) is commutative.
\end{proof}

\subsubsection*{Associativity}
Let \(a, b, c \in \mathbb{Z}_n\). We need to show that \((a + b) + c = a + (b + c)\) in \(\mathbb{Z}_n\).
\begin{proof} \( \)

Addition in \(\mathbb{Z}_n\) is defined as:
\[
a + b \equiv a + b \pmod{n}.
\]
Using the associativity of addition in \(\mathbb{Z}\), we have:
\[
(a + b) + c = a + (b + c).
\]
Therefore:
\[
(a + b) + c \equiv a + (b + c) \pmod{n}.
\]
Hence, addition in \(\mathbb{Z}_n\) is associative.
\end{proof}
\pagebreak
\subsection*{Multiplication in \(\mathbb{Z}_n\)}

\subsubsection*{Commutativity}
Let \(a, b \in \mathbb{Z}_n\). We need to show that \(a \cdot b = b \cdot a\) in \(\mathbb{Z}_n\).
\begin{proof}
Multiplication in \(\mathbb{Z}_n\) is defined as:
\[
a \cdot b \equiv a \cdot b \pmod{n}.
\]
Using the commutativity of multiplication in \(\mathbb{Z}\), we have:
\[
a \cdot b = b \cdot a.
\]
Therefore:
\[
a \cdot b \equiv b \cdot a \pmod{n}.
\]
Hence, multiplication in \(\mathbb{Z}_n\) is commutative.
\end{proof}

\subsubsection*{Associativity}
Let \(a, b, c \in \mathbb{Z}_n\). We need to show that \((a \cdot b) \cdot c = a \cdot (b \cdot c)\) in \(\mathbb{Z}_n\).
\begin{proof}
Multiplication in \(\mathbb{Z}_n\) is defined as:
\[
a \cdot b \equiv a \cdot b \pmod{n}.
\]
Using the associativity of multiplication in \(\mathbb{Z}\), we have:
\[
(a \cdot b) \cdot c = a \cdot (b \cdot c).
\]
Therefore:
\[
(a \cdot b) \cdot c \equiv a \cdot (b \cdot c) \pmod{n}.
\]
Hence, multiplication in \(\mathbb{Z}_n\) is associative.
\end{proof}

\section*{Conclusion}
We have shown that both addition and multiplication in \(\mathbb{Z}_n\) are commutative and associative, completing the proof.

\end{exercise}
\newpage
\begin{exercise}{\textbf{1.7.16}} Find an integer $x$ such that \(x \equiv 3 \pmod 4\) and \(x \equiv 5\pmod9\).
    
    \noindent\rule{\linewidth}{1pt}

    \vspace*{-10pt}\section*{Solution}

We need to find an integer \(x\) that satisfies the following system of congruences:
\[
\begin{cases}
x \equiv 3 \pmod{4} \\
x \equiv 5 \pmod{9}
\end{cases}
\]

To solve this, we can use the method of successive substitutions or apply the Chinese Remainder Theorem.

\begin{proof} \(\)
    \section*{} \vspace*{-30pt}
\subsection*{Method of Successive Substitutions}
\begin{enumerate}[label=\arabic*.]
    \item \textbf{Express \(\mathbf{x}\) in terms of one congruence:} From the first congruence, write \(x\) as:
\[
x = 4k + 3 \quad \text{for some integer } k.
\]

\item \textbf{Substitute into the second congruence:} Substitute \(x\) in the second congruence:
\[
4k + 3 \equiv 5 \pmod{9}.
\]
Simplify this to solve for \(k\):
\begin{align*}
4k + 3 &\equiv 5 \pmod{9} \\
4k &\equiv 2 \pmod{9} \\
k &\equiv 2 \cdot 4^{-1} \pmod{9}.
\end{align*}

\item \textbf{Find the inverse of 4 modulo 9:} The multiplicative inverse of 4 modulo 9 is an integer \(y\) such that:
\[
4y \equiv 1 \pmod{9}.
\]
By checking values, we find that \(4 \cdot 7 \equiv 28 \equiv 1 \pmod{9}\). Thus, the inverse is 7.

\item \textbf{Solve for \(\mathbf{k}\):} Substitute the inverse back into the equation:
\[
k \equiv 2 \cdot 7 \pmod{9} \\
k \equiv 14 \pmod{9} \\
k \equiv 5 \pmod{9}.
\]

\item \textbf{Find \(\mathbf{x}\):} Substitute \(k\) back into the expression for \(x\):
\[
x = 4k + 3 = 4 \cdot 5 + 3 = 20 + 3 = 23.
\]
Thus,
\[
x \equiv 23 \pmod{36}.
\]

\end{enumerate}
\subsection*{Verification}
To verify:
\begin{itemize}[label=--]
    \item Check \(x \equiv 3 \pmod{4}\):
    \[
    23 \div 4 = 5 \text{ remainder } 3 \quad \Rightarrow \quad 23 \equiv 3 \pmod{4}.
    \]
    \item Check \(x \equiv 5 \pmod{9}\):
    \[
    23 \div 9 = 2 \text{ remainder } 5 \quad \Rightarrow \quad 23 \equiv 5 \pmod{9}.
    \]
\end{itemize}
Both congruences are satisfied.

\subsection*{Conclusion}
The integer \(x\) that satisfies both congruences is:
\[
\boxed{23}.
\]

\end{proof}
\end{exercise}
    \end{document}