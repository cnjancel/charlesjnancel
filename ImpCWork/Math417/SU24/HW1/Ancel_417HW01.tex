\documentclass[12pt]{amsart}
\usepackage[margin=1in]{geometry}
\usepackage{amssymb,amsfonts,amsmath}
\usepackage{color}
\usepackage{enumerate}
\usepackage{mathrsfs}
\usepackage{hyperref}
\usepackage[capitalise]{cleveref}
\usepackage{constants}
\usepackage{parskip}
\usepackage{indentfirst}
\usepackage{enumitem}
\usepackage{tikz}
\usepackage{graphicx}
\usepackage{longtable}
\usetikzlibrary{shapes.geometric, arrows}
\setlength{\parindent}{2em}
\hfuzz=200pt

%----Theorem Environments----
\newtheorem{theorem}{Theorem}[section]
\newtheorem{corollary}[theorem]{Corollary}
\newtheorem{hypothesis}[theorem]{Hypothesis}
\newtheorem{proposition}[theorem]{Proposition}
\newtheorem{lemma}[theorem]{Lemma}
\newtheorem{problem*}{Problem}

\theoremstyle{definition}
\newtheorem{definition}[theorem]{Definition}
\newtheorem{example}[theorem]{Example}
\newcommand{\exercise}[1]{\noindent {\bf Exercise #1.}}

\numberwithin{equation}{section}

\crefname{figure}{Figure}{Figures}
\crefname{theorem}{Theorem}{Theorems}
\crefname{cor}{Corollary}{Corollaries}
\crefname{exercise}{Exercise}{Exercises}
\crefname{cor*}{Corollary}{Corollaries}
\crefname{lem}{Lemma}{Lemmas}
\crefname{prop}{Proposition}{Propositions}
\crefname{conj}{Conjecture}{Conjectures}
\crefname{defn}{Definition}{Definitions}
\crefname{hyp}{Hypothesis}{Hypotheses}

\newcommand{\Z}{\mathbb{Z}}
\renewcommand{\C}{\mathbb{C}}
\newcommand{\R}{\mathbb{R}}
\newcommand{\Q}{\mathbb{Q}}
\newcommand{\F}{\mathbb{F}}
\newcommand{\N}{\mathbb{N}}
\newcommand{\re}{\textup{Re}}
\newcommand{\im}{\textup{Im}}
\renewcommand{\epsilon}{\varepsilon}
\newcommand{\Li}{\mathrm{Li}}

\title{Math 417, Homework 1}
\author{Charles Ancel}

\begin{document}
\maketitle

\begin{exercise}{1.1.3} Determine the rotational symmetries of a brick with three unequal sides.
    \noindent\rule{\linewidth}{1pt}
    \section{Introduction}
    A brick is a rectangular prism with dimensions \(a \times b \times c\), where \(a \neq b \neq c\). We aim to determine the rotational symmetries of such a brick, i.e., all rotations that map the brick onto itself while preserving its geometric structure.
    
    \begin{proof}
    \section{Rotational Symmetries}
    To find the rotational symmetries, we consider rotations around the principal axes passing through the centers of the faces of the brick. These axes are:
    \begin{itemize}[label=--]
        \item \(x\)-axis: through the centers of the faces of dimension \(b \times c\)
        \item \(y\)-axis: through the centers of the faces of dimension \(a \times c\)
        \item \(z\)-axis: through the centers of the faces of dimension \(a \times b\)
    \end{itemize}
        
    \subsection{Symmetries Catalog}

    \begin{longtable}{|c|c|c|p{8cm}|}
    \hline
    \textbf{Notation} & \textbf{Angle} & \textbf{Axis}          & \textbf{Effect on Brick} \\ \hline
    \endfirsthead
    \multicolumn{4}{c}%
    {{\bfseries \tablename\ \thetable{} -- continued from previous page}} \\
    \hline
    \textbf{Notation} & \textbf{Angle} & \textbf{Axis}          & \textbf{Effect on Brick} \\ \hline
    \endhead
    \hline \multicolumn{4}{r}{{Continued on next page}} \\
    \endfoot
    \endlastfoot
    \( e \)           & 0°             & --                     & No change (Identity rotation) \\ \hline
    \( r_x^2 \)       & 180°           & \(x\)-axis             & Swaps front/back, reverses top/bottom \\ \hline
    \( r_y^2 \)       & 180°           & \(y\)-axis             & Swaps left/right, reverses top/bottom \\ \hline
    \( r_z^2 \)       & 180°           & \(z\)-axis             & Swaps top/bottom, reverses front/back \\ \hline
    \( r_x \)         & 90°            & \(x\)-axis             & Rotates front to top, top to back, back to bottom, etc. \\ \hline
    \( r_y \)         & 90°            & \(y\)-axis             & Rotates left to top, top to right, right to bottom, etc. \\ \hline
    \( r_z \)         & 90°            & \(z\)-axis             & Rotates top to left, left to bottom, bottom to right, etc. \\ \hline
    \( r_x^3 \)       & 270°           & \(x\)-axis             & Rotates front to bottom, bottom to back, back to top, etc. \\ \hline
    \( r_y^3 \)       & 270°           & \(y\)-axis             & Rotates left to bottom, bottom to right, right to top, etc. \\ \hline
    \( r_z^3 \)       & 270°           & \(z\)-axis             & Rotates top to right, right to bottom, bottom to left, etc. \\ \hline
    \caption{Rotational symmetries of a brick with unequal sides.}
    \label{tab:symmetries}
    \end{longtable}

    \section{Conclusion}
    Summarizing the distinct rotational symmetries, we have:
    \begin{itemize}[label=--]
        \item Identity rotation: \(1\)
        \item 180° rotations around principal axes: \(3\)
        \item 90° and 270° rotations around principal axes: \(6\)
    \end{itemize}

    \noindent Therefore, the total number of rotational symmetries of a brick with three unequal sides is:
    \begin{equation*}
    1 + 3 + 6 = 10
    \end{equation*}

    \noindent Considering only rotational symmetries, the count is:
    \begin{equation*}
    \boxed{10}
    \end{equation*}
    \end{proof}
    \end{exercise}

    
\newpage
\begin{exercise}{1.3.3} Here is another way to list the symmetries of the square. 
\begin{enumerate}[label=\alph*.]
            \item Verify that the four symmetries \(a,b,c,d\) that exchange the top and bottom faces of the card are \(a, ra, r^2a, r^3a,\) in some order. Thus, a complete list of symmetries is \(\{e, r, r^2, r^3, a, ra, r^2a, r^3a\}\).

            \item Verify that \(ar=r^{-1}a=r^3a\).

            \item Conclude that \(ar^k=r^{-k}a\) for all \(k\in \Z\).

            \item Show that these relations suffice to compute every product.
        \end{enumerate}
        
        \begin{enumerate}[label=\alph*.]
            \item Verify that the four symmetries \(a, b, c, d\) that exchange the top and bottom faces of the card are \(a, ra, r^2a, r^3a\), in some order. Thus, a complete list of symmetries is:
            \begin{equation*}
            \{e, r, r^2, r^3, a, ra, r^2a, r^3a\}.
            \end{equation*}
    \noindent\rule{\linewidth}{1pt}
            \begin{proof}\( \)
                \begin{itemize}[label=--]
                    \item \(e\) is the identity (no change).
                    \item \(r\) represents a 90° rotation counterclockwise.
                    \item \(r^2\) represents a 180° rotation.
                    \item \(r^3\) represents a 270° rotation counterclockwise (or 90° clockwise).
                    \item \(a\) is a reflection across the vertical axis.
                    \item \(ra\), \(r^2a\), \(r^3a\) represent reflections after 90°, 180°, and 270° rotations respectively.
                    \item By enumerating and verifying these operations on a square, all symmetries are included in the set \(\{e, r, r^2, r^3, a, ra, r^2a, r^3a\}\).
                \end{itemize}
                
            \end{proof}
            \item Verify that \(ar = r^{-1}a = r^3a\).
            \begin{equation*}
            ar = r^{-1}a = r^3a.
            \end{equation*}

            \begin{proof} \( \)
                \begin{itemize}[label=--]
                    \item Consider \(ar\): Rotate by 90° counterclockwise (apply \(r\)) and then reflect across the vertical axis (apply \(a\)).
                    \item This is equivalent to reflecting first (apply \(a\)) and then rotating by 90° clockwise (apply \(r^{-1}\)).
                    \item Therefore, \(ar = r^{-1}a\).
                    \item Since \(r^{-1} = r^3\) in a modulo 4 system (as \(r^4 = e\)), we have:
                    \begin{equation*}
                    ar = r^{-1}a = r^3a.
                    \end{equation*}
                \end{itemize}
            \end{proof}
                \item Conclude that \(ar^k = r^{-k}a\) for all \(k \in \mathbb{Z}\).
                \begin{equation*}
                ar^k = r^{-k}a.
                \end{equation*}
            \begin{proof} \( \)
                \begin{itemize}[label=--]
                    \item We use induction on \(k\).
                    \item Base case: For \(k = 1\):
                    \begin{equation*}
                    ar = r^{-1}a.
                    \end{equation*}
                    \item Assume \(ar^k = r^{-k}a\) for some \(k\).
                    \item For \(k + 1\):
                    \begin{align*}
                    ar^{k+1} &= ar^k \cdot r \\
                             &= r^{-k}a \cdot r \\
                             &= r^{-k} \cdot r \cdot a \\
                             &= r^{-(k+1)}a.
                    \end{align*}
                    \item Similarly, for \(k = -1\):
                    \begin{equation*}
                    ar^{-1} = r \cdot a.
                    \end{equation*}
                    \item Assume \(ar^{-k} = r^ka\) for some \(k\).
                    \item For \(k + 1\):
                    \begin{align*}
                    ar^{-(k+1)} &= ar^{-k-1} \\
                             &= ar^{-k} \cdot r^{-1} \\
                             &= r^ka \cdot r^{-1} \\
                             &= r^k \cdot r^{-1} \cdot a \\
                             &= r^{k-1}a.
                    \end{align*}
                    \item Therefore, \(ar^k = r^{-k}a\) holds for all \(k \in \mathbb{Z}\).
                \end{itemize}
            \end{proof}
                \item Show that these relations suffice to compute every product.
                \begin{proof} \( \)
                \begin{itemize}[label=--]
                    \item Consider any product \(x \cdot y\), where \(x\) and \(y\) are elements from the set \(\{e, r, r^2, r^3, a, ra, r^2a, r^3a\}\).
                    \item Using the relations:
                    \begin{align*}
                    ar^k &= r^{-k}a, \\
                    a^2 &= e,
                    \end{align*}
                    \item We can compute any product:
                    \begin{itemize}[label=--]
                        \item For \(x = r^i\) and \(y = r^j\):
                        \begin{equation*}
                        r^i \cdot r^j = r^{i+j \mod 4}.
                        \end{equation*}
                        \item For \(x = r^i\) and \(y = ar^j\):
                        \begin{equation*}
                        r^i \cdot ar^j = ar^{-i}r^j = ar^{j-i}.
                        \end{equation*}
                        \item For \(x = ar^i\) and \(y = r^j\):
                        \begin{equation*}
                        ar^i \cdot r^j = a \cdot r^{-i}r^j = ar^{j-i}.
                        \end{equation*}
                        \item For \(x = ar^i\) and \(y = ar^j\):
                        \begin{equation*}
                        ar^i \cdot ar^j = a \cdot r^{-i} \cdot ar^j = a^2 \cdot r^{-i}r^j = r^{-i+j}.
                        \end{equation*}
                    \end{itemize}
                    \item Using these relations, any product of the symmetries can be computed.
                \end{itemize}
            \end{proof}
            \end{enumerate}
        \end{exercise}
        
        \newpage
\begin{exercise}{3} An \textbf{affine transformation} of \(\R^n\) is a function \(T: \R^n \rightarrow \R^n \) of the form \(T(x)= Ax+b\), where \(A \in \text{Mat}_{n\times n}(\R)\) and \(b \in \R^n\) 

        \begin{enumerate}[label=\alph*.]
            \item Show that if \(T,U : \R^n \rightarrow \R^n\) are affine transformations, so is the composite function \(U \circ T\).

            \item Show that \(T(x) \stackrel{\text{def}}{=} Ax+b\) is a bijection if and only if \(A\) is an invertible matrix. We call such \(T\) an  \textbf{invertible affine transformation.}

            \item Show that if \(T\) is an invertible affine transformation, then its inverse function is also an invertible affine transformation.
        \end{enumerate}
        \noindent\rule{\linewidth}{1pt}
        
        \begin{enumerate}[label=\alph*.]
            \item \textbf{Composite of Affine Transformations:}
            \begin{proof}
                
                Let \(T(x) = A_1x + b_1\) and \(U(x) = A_2x + b_2\) be affine transformations. We need to show that the composite function \(U \circ T\) is also an affine transformation.
                \begin{align*}
                (U \circ T)(x) &= U(T(x)) \\
                &= U(A_1x + b_1) \\
                &= A_2(A_1x + b_1) + b_2 \\
                &= A_2A_1x + A_2b_1 + b_2
                \end{align*}
                Let \(A = A_2A_1\) and \(b = A_2b_1 + b_2\). Thus,
                \[
                (U \circ T)(x) = Ax + b,
                \]
                which is an affine transformation. Hence, the composite of two affine transformations is also an affine transformation.
            
            \end{proof}
            \item \textbf{Bijection of Affine Transformation:}
            \begin{proof}
                
                Let \(T(x) = Ax + b\).
                
                \textbf{(\(\Rightarrow\)) Suppose \(T\) is a bijection:}
                \begin{itemize}[label=--]
                    \item \textbf{Surjectivity}: For every \(y \in \mathbb{R}^n\), there exists \(x \in \mathbb{R}^n\) such that \(T(x) = y\). Thus, \(Ax + b = y\), or \(Ax = y - b\). Since \(A\) maps to all \(y\), \(A\) must cover all of \(\mathbb{R}^n\), implying \(A\) is surjective and hence invertible.
                    \item \textbf{Injectivity}: If \(T(x_1) = T(x_2)\), then \(Ax_1 + b = Ax_2 + b\), or \(A(x_1 - x_2) = 0\). Since \(A\) is invertible, \(x_1 - x_2 = 0\), hence \(x_1 = x_2\), showing \(T\) is injective.
                \end{itemize}
                Therefore, \(T\) is bijective if \(A\) is invertible.

                \newpage

                \textbf{(\(\Leftarrow\)) Suppose \(A\) is invertible:}
                \begin{itemize}[label=--]
                    \item Define \(A^{-1}\) such that \(A^{-1}A = I\). Let \(T(x_1) = T(x_2)\). Then \(Ax_1 + b = Ax_2 + b\), which simplifies to \(A(x_1 - x_2) = 0\). Since \(A\) is invertible, \(x_1 - x_2 = 0\), so \(x_1 = x_2\), proving injectivity.
                    \item To show surjectivity, for any \(y \in \mathbb{R}^n\), let \(x = A^{-1}(y - b)\). Then \(T(x) = A(A^{-1}(y - b)) + b = y\), proving surjectivity.
                \end{itemize}
                Therefore, \(T(x)\) is a bijection if and only if \(A\) is invertible.
            
            \end{proof}
            \item \textbf{Inverse of an Invertible Affine Transformation:}
            \begin{proof}
                
                Suppose \(T(x) = Ax + b\) is an invertible affine transformation. We need to show that \(T^{-1}\) is also an affine transformation.
                
                
                \begin{itemize}[label=--]
                    \item Assume \(A\) is invertible. To find \(T^{-1}(y)\) for \(y \in \mathbb{R}^n\):
                    \begin{align*}
                    T(x) &= y \\
                    Ax + b &= y \\
                    Ax &= y - b \\
                    x &= A^{-1}(y - b)
                    \end{align*}
                    Define \(T^{-1}(y) = A^{-1}y - A^{-1}b\).
                    \item Verify that \(T^{-1}\) is indeed the inverse:
                    \begin{align*}
                    T(T^{-1}(y)) &= A(A^{-1}y - A^{-1}b) + b \\
                    &= y - b + b \\
                    &= y \\
                    T^{-1}(T(x)) &= A^{-1}(Ax + b - b) \\
                    &= A^{-1}Ax \\
                    &= x
                    \end{align*}
                    \item Therefore, \(T^{-1}(y) = A^{-1}y - A^{-1}b\) is also an affine transformation, with \(A^{-1}\) as the linear part and \(-A^{-1}b\) as the translation part.
                \end{itemize}
            \end{proof}
        \end{enumerate}
        \end{exercise}
        \newpage

\begin{exercise}{1.5.3} Work out the decomposition in disjoint cycles for the following:
            \begin{enumerate}[label=(\alph*)]
                \item \(
                      \begin{pmatrix}
                        1 & 2 & 3 & 4 & 5 & 6 & 7\\
                        2 & 5 & 6 & 3 & 7 & 4 & 1
                      \end{pmatrix}
                    \)
                \item \( (1 \ 2)(1\ 2 \ 3 \ 4 \ 5)\)
                \item \((1 \ 4)(1\ 2 \ 3 \ 4 \ 5)\)
                \item \((1 \ 2)(2 \ 3 \ 4 \ 5)\)
                \item \((1 \ 3)(2 \ 3 \ 4 \ 5)\)
                \item \((1 \ 2)(2 \ 3)(3 \ 4)\)
                \item \((1 \ 2)(1 \ 3)(1 \ 4)\)
                \item \((1 \ 3)(1 \ 2 \ 3 \ 4)(1 \ 3)\)
            \end{enumerate}
            \noindent\rule{\linewidth}{1pt}
        
        \section*{Solutions}
        
        \begin{enumerate}[label=(\alph*)]
            \item \(
                  \begin{pmatrix}
                    1 & 2 & 3 & 4 & 5 & 6 & 7\\
                    2 & 5 & 6 & 3 & 7 & 4 & 1
                  \end{pmatrix}
                \)\\
            \textbf{Solution:} We start with 1: 
            \[ 1 \to 2 \to 5 \to 7 \to 1 \]
            Cycle: \( (1 \ 2 \ 5 \ 7) \)\\
            Next, we start with 3:
            \[ 3 \to 6 \to 4 \to 3 \]
            Cycle: \( (3 \ 6 \ 4) \)\\
            Disjoint cycles: 
            \[
            (1 \ 2 \ 5 \ 7)(3 \ 6 \ 4)
            \]
        
            \item \( (1 \ 2)(1 \ 2 \ 3 \ 4 \ 5) \)\\
            \textbf{Solution:} Apply the permutations in sequence:
            \[ (1 \ 2) \circ (1 \ 2 \ 3 \ 4 \ 5) \]
            Start with 1:
            \[ 1 \to 2 \to 1 \to 3 \to 4 \to 5 \to 2 \]
            Cycle: \( (1 \ 2) \)\\
            Next:
            \[ 1 \to 2 \to 3 \to 5 \to 4 \]
            Cycle: \( (1 \ 2 \ 3 \ 5 \ 4) \)
        
            \item \((1 \ 4)(1 \ 2 \ 3 \ 4 \ 5)\)\\
            \textbf{Solution:} Apply the permutations in sequence:
            \[ (1 \ 4) \circ (1 \ 2 \ 3 \ 4 \ 5) \]
            Start with 1:
            \[ 1 \to 4 \to 5 \to 1 \]
            Cycle: \( (1 \ 4 \ 5) \)\\
            Continue with 2:
            \[ 2 \to 3 \to 2 \]
            Cycle: \( (2 \ 3) \)\\
            Disjoint cycles: 
            \[
            (1 \ 4 \ 5)(2 \ 3)
            \]
        
            \item \((1 \ 2)(2 \ 3 \ 4 \ 5)\)\\
            \textbf{Solution:} Apply the permutations in sequence:
            \[ (1 \ 2) \circ (2 \ 3 \ 4 \ 5) \]
            Start with 1:
            \[ 1 \to 2 \to 3 \to 4 \to 5 \]
            Cycle: \( (1 \ 2 \ 3 \ 4 \ 5) \)\\
            Continue:
            \[ 1 \to 2 \to 5 \to 4 \to 3 \]
            Simplifies to: \( (1 \ 2 \ 5)(3 \ 4) \)
        
            \item \((1 \ 3)(2 \ 3 \ 4 \ 5)\)\\
            \textbf{Solution:} Apply the permutations in sequence:
            \[ (1 \ 3) \circ (2 \ 3 \ 4 \ 5) \]
            Start with 1:
            \[ 1 \to 3 \to 4 \to 5 \]
            Cycle: \( (1 \ 3 \ 5) \)\\
            Continue:
            \[ 2 \to 3 \to 4 \]
            Cycle: \( (2 \ 4) \)\\
            Disjoint cycles: 
            \[
            (1 \ 3 \ 5)(2 \ 4)
            \]
        
            \item \((1 \ 2)(2 \ 3)(3 \ 4)\)\\
            \textbf{Solution:} Apply the permutations in sequence:
            \[ (1 \ 2) \circ (2 \ 3) \circ (3 \ 4) \]
            Start with 1:
            \[ 1 \to 2 \to 3 \to 4 \]
            Cycle: \( (1 \ 2 \ 3 \ 4) \)\\
            Since the rest only flips between 2 and 3, this simplifies to:
            \[
            (1 \ 4)(2 \ 3)
            \]
        
            \item \((1 \ 2)(1 \ 3)(1 \ 4)\)\\
            \textbf{Solution:} Apply the permutations in sequence:
            \[ (1 \ 2) \circ (1 \ 3) \circ (1 \ 4) \]
            Start with 1:
            \[ 1 \to 4 \to 1 \to 3 \to 1 \to 2 \]
            Cycle: \( (1 \ 4 \ 3 \ 2) \)
        
            \item \((1 \ 3)(1 \ 2 \ 3 \ 4)(1 \ 3)\)\\
            \textbf{Solution:} Apply the permutations in sequence:
            \[ (1 \ 3) \circ (1 \ 2 \ 3 \ 4) \circ (1 \ 3) \]
            Start with 1:
            \[ 1 \to 3 \to 1 \to 2 \to 3 \]
            Cycle: \( (1 \ 2 \ 4) \)\\
            Disjoint cycles: 
            \[
            (1 \ 2 \ 4)(3)
            \]
        
        \end{enumerate}        
    \end{exercise}
    \newpage

    \begin{exercise}{1.5.5.} Show that any $k$-cycle \((a_1 \dots a_k)\) can be written as a product of \((k-1)\) 2-cycles. Conclude that any permutation can be written as a product of some number of 2-cycles.
        \noindent\rule{\linewidth}{1pt}
        \begin{proof} \(\)
            \subsection*{Part 1: \( k \)-cycle as a Product of 2-cycles}

Consider a \( k \)-cycle \( \sigma = (a_1 \, a_2 \, \dots \, a_k) \), where \( \sigma \) maps \( a_i \) to \( a_{i+1} \) for \( 1 \leq i < k \), and \( \sigma \) maps \( a_k \) back to \( a_1 \). We will show that \( \sigma \) can be decomposed into \( (k-1) \) 2-cycles.

\begin{align*}
\sigma &= (a_1 \, a_2 \, \dots \, a_k) \\
       &= (a_1 \, a_k)(a_1 \, a_{k-1}) \cdots (a_1 \, a_3)(a_1 \, a_2)
\end{align*}

\noindent To verify this decomposition, consider the action of the product of 2-cycles on each element:

\begin{itemize}[label=--]
    \item For \( a_1 \):
    \[
    (a_1 \, a_k) \cdots (a_1 \, a_2)(a_1) = a_2
    \]
    Each 2-cycle swaps \( a_1 \) with the respective \( a_i \) it pairs with, ultimately resulting in \( a_1 \) being mapped to \( a_2 \).

    \item For \( a_2 \):
    \[
    (a_1 \, a_k) \cdots (a_1 \, a_2)(a_2) = a_3
    \]
    The first 2-cycle fixes \( a_2 \), and each subsequent 2-cycle swaps \( a_2 \) until it is mapped to \( a_3 \).

    \item Continuing similarly:
    \[
    (a_1 \, a_k) \cdots (a_1 \, a_2)(a_i) = a_{i+1} \quad \text{for } 2 \leq i < k
    \]

    \item For \( a_k \):
    \[
    (a_1 \, a_k) \cdots (a_1 \, a_2)(a_k) = a_1
    \]
    Each 2-cycle moves \( a_k \) back to \( a_1 \).
\end{itemize}

\noindent Thus, \( \sigma = (a_1 \, a_k)(a_1 \, a_{k-1}) \cdots (a_1 \, a_3)(a_1 \, a_2) \), confirming the decomposition into \( (k-1) \) 2-cycles.

\subsection*{Part 2: Any Permutation as a Product of 2-cycles}

Any permutation \( \pi \) in \( S_n \) can be written as a product of disjoint cycles. Let \( \pi = \tau_1 \tau_2 \cdots \tau_m \), where \( \tau_i \) are disjoint cycles. Each \( \tau_i \) can be expressed as a product of 2-cycles as shown in Part 1. Therefore, \( \pi \) can be written as a product of 2-cycles.
        \end{proof}
    \end{exercise}
    \newpage
    \begin{exercise}{1.5.9.} Let \(\sigma_n\) denote the perfect shuffle of a deck of \(2n\) cards. Regard \(\sigma_n\) as a bijective function of the set \(\{1,2,\dots, 2n\}\). Find a formula for \(\sigma_n(j)\) when \(1 \leq j \leq n\), and another formula for \(\sigma_n(j)\) when \(n + 1 \leq j \leq 2n\). (The "perfect shuffle" is described in Example 1.5.2.)
        \noindent\rule{\linewidth}{1pt}
        
        \section*{Solution}
        
        A \textbf{perfect shuffle} interleaves the two halves of a deck of \(2n\) cards. Assume the deck is represented by the sequence \(\{1, 2, \ldots, 2n\}\). The perfect shuffle operation, denoted by \(\sigma_n\), interleaves the two halves as follows:
        
        \begin{proof} \( \)
        \subsection*{Formulas for \(\sigma_n\)}
    
        \textbf{Case 1:} \(1 \leq j \leq n\)
    
        For the first half of the deck:
        \[
        \sigma_n(j) = 2j - 1
        \]
        \textbf{Derivation:}
        \begin{itemize}[label=--]
            \item Consider the original position \(j\) in the first half \(\{1, 2, \ldots, n\}\).
            \item In the shuffled deck, the position is odd, and each element from the first half goes to the positions \(1, 3, 5, \ldots, 2n-1\).
            \item Hence, \(j\) is mapped to \(2j - 1\).
        \end{itemize}
    
        \textbf{Case 2:} \(n + 1 \leq j \leq 2n\)
    
        For the second half of the deck:
        \[
        \sigma_n(j) = 2(j - n)
        \]
        \textbf{Derivation:}
        \begin{itemize}[label=--]
            \item Consider the original position \(j\) in the second half \(\{n+1, n+2, \ldots, 2n\}\).
            \item In the shuffled deck, the position is even, and each element from the second half goes to the positions \(2, 4, 6, \ldots, 2n\).
            \item Hence, \(j\) is mapped to \(2(j - n)\), which simplifies to \(2j - 2n\).
        \end{itemize}
    
        \subsection*{Verification}
    
        To verify these formulas, consider the following example with \(n = 3\):
    
        \begin{itemize}[label=--]
            \item Original deck: \(\{1, 2, 3, 4, 5, 6\}\)
            \item First half: \(\{1, 2, 3\}\)
            \item Second half: \(\{4, 5, 6\}\)
            \item Shuffled deck: \(\{1, 4, 2, 5, 3, 6\}\)
        \end{itemize}
    
        \noindent Applying the formulas:
        \begin{itemize}[label=--]
            \item For \(1 \leq j \leq 3\): \(\sigma_3(j) = 2j - 1\)
            \[
            \sigma_3(1) = 2(1) - 1 = 1
            \]
            \[
            \sigma_3(2) = 2(2) - 1 = 3
            \]
            \[
            \sigma_3(3) = 2(3) - 1 = 5
            \]
            \item For \(4 \leq j \leq 6\): \(\sigma_3(j) = 2(j - 3)\)
            \[
            \sigma_3(4) = 2(4 - 3) = 2
            \]
            \[
            \sigma_3(5) = 2(5 - 3) = 4
            \]
            \[
            \sigma_3(6) = 2(6 - 3) = 6
            \]
        \end{itemize}
    
        \noindent Therefore, the shuffled deck is \(\{1, 4, 2, 5, 3, 6\}\), which matches our calculation.
    
        \section*{Conclusion}
    
        We have derived the formulas for the perfect shuffle \(\sigma_n\):
        \[
        \sigma_n(j) = 
        \begin{cases} 
        2j - 1 & \text{if } 1 \leq j \leq n \\
        2(j - n) & \text{if } n + 1 \leq j \leq 2n
        \end{cases}
        \]
        These formulas correctly map each position \(j\) in the original deck to its new position in the shuffled deck.
    
        \end{proof}
    \end{exercise}
    \end{document}