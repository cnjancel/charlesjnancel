\documentclass[12pt]{amsart}
\usepackage[margin=1in]{geometry}
\usepackage{amssymb,amsfonts,amsmath}
\usepackage{color}
\usepackage{enumerate}
\usepackage{mathrsfs}
\usepackage{hyperref}
\usepackage[capitalise]{cleveref}
\usepackage{constants}
\usepackage{parskip}
\usepackage{indentfirst}
\usepackage{enumitem}
\usepackage{tikz}
\usepackage{graphicx}
\usepackage{longtable}
\usetikzlibrary{shapes.geometric, arrows}
\setlength{\parindent}{2em}
\hfuzz=200pt

%----Theorem Environments----
\newtheorem{theorem}{Theorem}[section]
\newtheorem{corollary}[theorem]{Corollary}
\newtheorem{hypothesis}[theorem]{Hypothesis}
\newtheorem{proposition}[theorem]{Proposition}
\newtheorem{lemma}[theorem]{Lemma}
\newtheorem{problem*}{Problem}

\theoremstyle{definition}
\newtheorem{definition}[theorem]{Definition}
\newtheorem{example}[theorem]{Example}
\newcommand{\exercise}[1]{\noindent {\bf Exercise #1.}}

\numberwithin{equation}{section}

\crefname{figure}{Figure}{Figures}
\crefname{theorem}{Theorem}{Theorems}
\crefname{cor}{Corollary}{Corollaries}
\crefname{exercise}{Exercise}{Exercises}
\crefname{cor*}{Corollary}{Corollaries}
\crefname{lem}{Lemma}{Lemmas}
\crefname{prop}{Proposition}{Propositions}
\crefname{conj}{Conjecture}{Conjectures}
\crefname{defn}{Definition}{Definitions}
\crefname{hyp}{Hypothesis}{Hypotheses}

\newcommand{\Z}{\mathbb{Z}}
\renewcommand{\C}{\mathbb{C}}
\newcommand{\R}{\mathbb{R}}
\newcommand{\Q}{\mathbb{Q}}
\newcommand{\F}{\mathbb{F}}
\newcommand{\N}{\mathbb{N}}
\newcommand{\re}{\textup{Re}}
\newcommand{\im}{\textup{Im}}
\renewcommand{\epsilon}{\varepsilon}
\newcommand{\Li}{\mathrm{Li}}

\title{Math 417, Homework 5}
\author{Charles Ancel}

\begin{document}
\maketitle

%-----------------------------
\begin{exercise}{2.4.8} Let \(\phi : G \rightarrow H\) be a homomorphism of $G$ onto $H$ (that is, \(\phi\) is surjective). If $A$ is a normal subgroup of $G$, shat that \(\phi(A)\) is a normal subgroup of $H$.

    \noindent\rule{\linewidth}{1pt}
    
    
    \section*{Solution}

    Let \(\phi : G \rightarrow H\) be a surjective homomorphism and let \(A\) be a normal subgroup of \(G\). We need to show that \(\phi(A)\) is a normal subgroup of \(H\).
    
    \subsection*{Normal Subgroups}
    
    Recall that a subgroup \(A \triangleleft G\) is normal if for all \(g \in G\):
    \[
    gAg^{-1} \subseteq A.
    \]
    
    Similarly, a subgroup \(B \subseteq H\) is normal if for all \(h \in H\):
    \[
    hBh^{-1} \subseteq B.
    \]
    
    \subsection*{Using the Homomorphism Property}
    
    Since \(\phi\) is a homomorphism, it satisfies:
    \[
    \phi(g_1 g_2) = \phi(g_1) \phi(g_2) \quad \text{for all } g_1, g_2 \in G.
    \]
    
    To show that \(\phi(A)\) is normal in \(H\), we must show that for any \(h \in H\) and any \(a' \in \phi(A)\):
    \[
    h a' h^{-1} \in \phi(A).
    \]
    
    \begin{proof} \( \)
        
        Let \(h \in H\) and \(a' \in \phi(A)\). Since \(\phi\) is surjective, there exists some \(g \in G\) such that \(\phi(g) = h\).
        
        Let \(a' \in \phi(A)\). By definition of \(\phi(A)\), there exists some \(a \in A\) such that \(\phi(a) = a'\). Consider:
        \[
        h a' h^{-1} = \phi(g) \phi(a) \phi(g^{-1}).
        \]
        
        Using the homomorphism property:
        \[
        h a' h^{-1} = \phi(g) \phi(a) \phi(g^{-1}) = \phi(g a g^{-1}).
        \]
        
        Since \(A\) is normal in \(G\), we have \(g a g^{-1} \in A\). Therefore:
        \[
        \phi(g a g^{-1}) \in \phi(A).
        \]
        
        Thus:
        \[
        h a' h^{-1} \in \phi(A).
        \]
        
        \subsection*{Conclusion}
        
        We have shown that for any \(h \in H\) and any \(a' \in \phi(A)\):
        \[
        h a' h^{-1} \in \phi(A).
        \]
        
        Therefore, \(\phi(A)\) is a normal subgroup of \(H\).
    \end{proof}
    
    \end{exercise}
    \newpage
    

%-----------------------------
\begin{exercise}{2.5.8} Suppose $N$ is a subgroup of a group $G$ and \([G:N]=2\). Show that $N$ is normal in $G$.(Hint: use the fact that a subgroup is normal if and only if every left coset is also a right coset.)

    \noindent\rule{\linewidth}{1pt}
    
    
    \section*{Solution}

Let \(N\) be a subgroup of \(G\) such that the index \([G:N] = 2\). We need to show that \(N\) is a normal subgroup of \(G\).

\subsection*{Cosets and Index}

Since \([G:N] = 2\), \(N\) has exactly two cosets in \(G\):
\begin{itemize}
    \item The coset \(N\).
    \item Another coset, say \(gN\), where \(g \notin N\).
\end{itemize}

\subsection*{Left and Right Cosets}

To show that \(N\) is normal in \(G\), we will use the fact that a subgroup is normal if and only if every left coset is also a right coset. Specifically, we need to show that for every \(g \in G\):
\[
gN = Ng.
\]

\begin{proof} \( \)
    
    Let \(g \in G\). There are two possible cases for \(g\):
    \begin{enumerate}
        \item \(g \in N\).
        \item \(g \notin N\).
    \end{enumerate}
    
    \subsubsection*{Case 1: \(g \in N\)}
    
    If \(g \in N\), then:
    \[
    gN = N = Ng.
    \]
    Thus, the left coset \(gN\) is equal to the right coset \(Ng\).
    
    \subsubsection*{Case 2: \(g \notin N\)}
    
    If \(g \notin N\), then \(gN\) is the other coset of \(N\) in \(G\), and since there are only two cosets, we have:
    \[
    gN = G \setminus N.
    \]
    
    To show \(gN = Ng\), consider the right coset \(Ng\). Since \(g \notin N\), we have:
    \[
    Ng = \{ng \mid n \in N\}.
    \]
    
    Similarly, since \(N\) is the identity left coset and \(gN\) is the other coset:
    \[
    gN = \{gn \mid n \in N\}.
    \]
    
    We need to show that \(gN = Ng\). Consider an arbitrary element \(gn \in gN\). We can rewrite \(gn\) as \(gn = ng' \in Ng\) for some \(g' \in N\). Since \(gN = G \setminus N\), and similarly for right cosets, the cosets must cover \(G\) without overlap:
    \[
    Ng = G \setminus N.
    \]
    
    Therefore:
    \[
    gN = Ng.
    \]
    
    \subsection*{Conclusion}
    
    We have shown that for every \(g \in G\):
    \[
    gN = Ng.
    \]
    
    Thus, every left coset of \(N\) is also a right coset, and \(N\) is a normal subgroup of \(G\).
\end{proof}

\end{exercise}
\newpage
    

%-----------------------------
\begin{exercise}{3} Let $D$ be the symmetry group of the disk, as described in class and in Goodman 2.6. Show that there is a function \(\phi:D \rightarrow D\) such that \(\phi(r_\theta)=r_{2\theta} \text{ and } \phi(j_\theta)=j_{2\theta}\)(this means: show that \(\phi\) is well-defined), and that this function \(\phi\) is a homomorphism of groups. Also describe the kernel of \(\phi\).

    The following exercise sets up an example which will appear in future problem sets. Here $A$ will be a commutative ring with identity (examples: \(\Z, \ \Q, \ \R, \ \C, \ \Z_n.\)) I'll write \[C(A):= \{(x,y) \ | \ x,y\in A, \ x^2+y^2=1 \}.\] For instance, \(C(\R)\) is the unit circle in \(\R^2\).
   
    \noindent\rule{\linewidth}{1pt}

    \section*{Solution}

\subsection*{Structure of the Symmetry Group \(D\)}

The symmetry group \(D\) of the disk consists of:
\begin{itemize}
    \item Rotations \(r_\theta\) by \(\theta\) radians.
    \item Reflections \(j_\theta\) through a line making an angle \(\theta/2\) with a fixed axis.
\end{itemize}

The group operation involves composition of symmetries:
\[
r_\theta r_\phi = r_{\theta + \phi}, \quad j_\theta j_\phi = r_{\theta + \phi}, \quad r_\theta j_\phi = j_{\theta + \phi}, \quad j_\theta r_\phi = j_{\theta - \phi}.
\]

\subsection*{Defining the Function \(\phi\)}

We define the function \(\phi: D \rightarrow D\) by:
\[
\phi(r_\theta) = r_{2\theta}, \quad \phi(j_\theta) = j_{2\theta}.
\]

We need to show that \(\phi\) is well-defined and a homomorphism.

\subsection*{Well-Definedness of \(\phi\)}

To show that \(\phi\) is well-defined, we must verify that the function \(\phi\) respects the group structure and produces valid elements of \(D\).

\begin{proof} \( \)

\subsubsection*{Rotations}

Consider the rotation \(r_\theta\):
\[
\phi(r_\theta) = r_{2\theta}.
\]

\subsubsection*{Reflections}

Consider the reflection \(j_\theta\):
\[
\phi(j_\theta) = j_{2\theta}.
\]

\subsection*{Homomorphism Property}

To show that \(\phi\) is a homomorphism, we must verify that for any \(g, h \in D\):
\[
\phi(gh) = \phi(g) \phi(h).
\]

Consider the cases for rotations and reflections:

\begin{itemize}
    \item \textbf{Case 1:} \(g = r_\theta\) and \(h = r_\phi\):
    \[
    \phi(r_\theta r_\phi) = \phi(r_{\theta + \phi}) = r_{2(\theta + \phi)} = r_{2\theta + 2\phi} = \phi(r_\theta) \phi(r_\phi).
    \]
    \item \textbf{Case 2:} \(g = j_\theta\) and \(h = j_\phi\):
    \[
    \phi(j_\theta j_\phi) = \phi(r_{\theta + \phi}) = r_{2(\theta + \phi)} = r_{2\theta + 2\phi} = j_{2\theta} j_{2\phi}.
    \]
    \item \textbf{Case 3:} \(g = r_\theta\) and \(h = j_\phi\):
    \[
    \phi(r_\theta j_\phi) = \phi(j_{\theta + \phi}) = j_{2(\theta + \phi)} = j_{2\theta + 2\phi} = \phi(r_\theta) \phi(j_\phi).
    \]
    \item \textbf{Case 4:} \(g = j_\theta\) and \(h = r_\phi\):
    \[
    \phi(j_\theta r_\phi) = \phi(j_{\theta - \phi}) = j_{2(\theta - \phi)} = j_{2\theta - 2\phi} = \phi(j_\theta) \phi(r_\phi).
    \]
\end{itemize}

In all cases, \(\phi(gh) = \phi(g) \phi(h)\), so \(\phi\) is a homomorphism.

\end{proof}

\subsection*{Kernel of \(\phi\)}

The kernel of \(\phi\) is the set of elements in \(D\) that are mapped to the identity element under \(\phi\). We need to identify these elements:

\begin{proof} \( \)

Consider the elements of \(D\):
\begin{itemize}
    \item For rotations \(r_\theta\):
    \[
    \phi(r_\theta) = r_{2\theta} = e \implies 2\theta = 0 \pmod{2\pi} \implies \theta = 0 \pmod{\pi}.
    \]
    Thus, the rotations in the kernel are \(r_0\) and \(r_\pi\).
    \item For reflections \(j_\theta\):
    \[
    \phi(j_\theta) = j_{2\theta} = e \implies j_{2\theta} = j_0 \implies \theta = 0 \pmod{\pi}.
    \]
    Thus, the reflections in the kernel are \(j_0\) and \(j_\pi\).
\end{itemize}

Therefore, the kernel of \(\phi\) is:
\[
\ker(\phi) = \{r_0, r_\pi, j_0, j_\pi\}.
\]


\subsection*{Conclusion}

The function \(\phi: D \rightarrow D\) defined by \(\phi(r_\theta) = r_{2\theta}\) and \(\phi(j_\theta) = j_{2\theta}\) is well-defined and is a homomorphism of groups. The kernel of \(\phi\) is:
\[
    \ker(\phi) = \{r_0, r_\pi, j_0, j_\pi\}.
    \]
    
\end{proof}
\end{exercise}
\newpage
    

%-----------------------------
\begin{exercise}{4} Given \((x_1,y_1),(x_2,y_2) \in C(A)\), define \[(x_1,y_1) \oplus (x_2,y_2) := (x_1x_2 - y_1y_2, x_1y_2+y_1x_2). \] Show that this always takes values in \(C(A)\), and that \((C(A)), \oplus\) is an abelian group.
    
    \noindent\rule{\linewidth}{1pt}
    
    \section*{Solution}

    Let \(C(A)\) be defined as:
    \[
    C(A) := \{(x, y) \mid x, y \in A, \ x^2 + y^2 = 1\}.
    \]
    
    We define the operation \(\oplus\) on \(C(A)\) by:
    \[
    (x_1, y_1) \oplus (x_2, y_2) := (x_1 x_2 - y_1 y_2, x_1 y_2 + y_1 x_2).
    \]
    
    \subsection*{Values in \(C(A)\)}
    
    To show that \((x_1, y_1) \oplus (x_2, y_2)\) always takes values in \(C(A)\), we need to verify that:
    \[
    (x_1 x_2 - y_1 y_2)^2 + (x_1 y_2 + y_1 x_2)^2 = 1.
    \]
    
    \begin{proof} \( \)
    
    Let \((x_1, y_1), (x_2, y_2) \in C(A)\). Then:
    \[
    x_1^2 + y_1^2 = 1 \quad \text{and} \quad x_2^2 + y_2^2 = 1.
    \]
    
    Consider the new pair \((x_1 x_2 - y_1 y_2, x_1 y_2 + y_1 x_2)\):
    \[
    \begin{aligned}
    (x_1 x_2 - y_1 y_2)^2 + (x_1 y_2 + y_1 x_2)^2 &= x_1^2 x_2^2 - 2 x_1 x_2 y_1 y_2 + y_1^2 y_2^2 \\
    &\quad + x_1^2 y_2^2 + 2 x_1 y_2 y_1 x_2 + y_1^2 x_2^2 \\
    &= x_1^2 x_2^2 + y_1^2 y_2^2 + x_1^2 y_2^2 + y_1^2 x_2^2 \\
    &= x_1^2 (x_2^2 + y_2^2) + y_1^2 (x_2^2 + y_2^2) \\
    &= x_1^2 \cdot 1 + y_1^2 \cdot 1 \\
    &= x_1^2 + y_1^2 \\
    &= 1.
    \end{aligned}
    \]
    
    Thus, \((x_1 x_2 - y_1 y_2, x_1 y_2 + y_1 x_2) \in C(A)\).
    
    \end{proof}
    
    \subsection*{Abelian Group Structure}
    
    We need to show that \((C(A), \oplus)\) is an abelian group.
    
    \begin{proof} \( \)
    
    \subsubsection*{Closure}
    
    From the previous proof, we have shown that \((x_1, y_1) \oplus (x_2, y_2) \in C(A)\) for all \((x_1, y_1), (x_2, y_2) \in C(A)\).
    
    \subsubsection*{Associativity}
    
    We need to show that \(\oplus\) is associative:
    \[
    ((x_1, y_1) \oplus (x_2, y_2)) \oplus (x_3, y_3) = (x_1, y_1) \oplus ((x_2, y_2) \oplus (x_3, y_3)).
    \]
    
    Compute both sides:
    \[
    \begin{aligned}
    ((x_1, y_1) \oplus (x_2, y_2)) \oplus (x_3, y_3) &= (x_1 x_2 - y_1 y_2, x_1 y_2 + y_1 x_2) \oplus (x_3, y_3) \\
    &= ((x_1 x_2 - y_1 y_2) x_3 - (x_1 y_2 + y_1 x_2) y_3, \\
    &\quad (x_1 x_2 - y_1 y_2) y_3 + (x_1 y_2 + y_1 x_2) x_3).
    \end{aligned}
    \]
    
    \[
    \begin{aligned}
    (x_1, y_1) \oplus ((x_2, y_2) \oplus (x_3, y_3)) &= (x_1, y_1) \oplus (x_2 x_3 - y_2 y_3, x_2 y_3 + y_2 x_3) \\
    &= (x_1 (x_2 x_3 - y_2 y_3) - y_1 (x_2 y_3 + y_2 x_3), \\
    &\quad x_1 (x_2 y_3 + y_2 x_3) + y_1 (x_2 x_3 - y_2 y_3)).
    \end{aligned}
    \]
    
    Both expressions simplify to the same result using distributivity.
    
    \subsubsection*{Identity Element}
    
    The identity element is \((1, 0)\) since:
    \[
    (x, y) \oplus (1, 0) = (x \cdot 1 - y \cdot 0, x \cdot 0 + y \cdot 1) = (x, y).
    \]
    
    \subsubsection*{Inverses}
    
    The inverse of \((x, y)\) is \((x, -y)\) since:
    \[
    (x, y) \oplus (x, -y) = (x \cdot x - y \cdot (-y), x \cdot (-y) + y \cdot x) = (x^2 + y^2, 0) = (1, 0).
    \]
    
    \subsubsection*{Commutativity}
    
    We need to show that \(\oplus\) is commutative:
    \[
    (x_1, y_1) \oplus (x_2, y_2) = (x_2, y_2) \oplus (x_1, y_1).
    \]
    
    Compute both sides:
    \[
    (x_1 x_2 - y_1 y_2, x_1 y_2 + y_1 x_2) = (x_2 x_1 - y_2 y_1, x_2 y_1 + y_2 x_1).
    \]
    
    Both expressions are the same, showing commutativity.
    
    
    \subsection*{Conclusion}
    
    The set \(C(A)\) with the operation \(\oplus\) forms an abelian group:
    \[
        (C(A), \oplus) \text{ is an abelian group.}
        \]
        
    \end{proof}
    \end{exercise}
    \newpage
    

%-----------------------------
\begin{exercise}{5} Show that \(\phi(t):= (\cos t, \sin t)\) defines a homomorphism \(\phi \ : \ (\R,+) \rightarrow (C(\R), \oplus)\). Show that this homomorphism is surjective and determine its kernel.

    \noindent\rule{\linewidth}{1pt}

    
    \section*{Solution}

    Let \(\phi: \R \rightarrow C(\R)\) be defined by:
    \[
    \phi(t) := (\cos t, \sin t).
    \]
    
    We need to show that \(\phi\) is a homomorphism, that it is surjective, and determine its kernel.
    
    \subsection*{Homomorphism Property}
    
    To show that \(\phi\) is a homomorphism, we must verify that for any \(t_1, t_2 \in \R\):
    \[
    \phi(t_1 + t_2) = \phi(t_1) \oplus \phi(t_2).
    \]
    
    Recall the operation \(\oplus\) in \(C(\R)\):
    \[
    (x_1, y_1) \oplus (x_2, y_2) := (x_1 x_2 - y_1 y_2, x_1 y_2 + y_1 x_2).
    \]
    
    \begin{proof} \( \)
    
    Consider \(\phi(t_1 + t_2)\):
    \[
    \phi(t_1 + t_2) = (\cos(t_1 + t_2), \sin(t_1 + t_2)).
    \]
    
    Using the angle addition formulas:
    \[
    \cos(t_1 + t_2) = \cos t_1 \cos t_2 - \sin t_1 \sin t_2,
    \]
    \[
    \sin(t_1 + t_2) = \sin t_1 \cos t_2 + \cos t_1 \sin t_2.
    \]
    
    Compute \(\phi(t_1) \oplus \phi(t_2)\):
    \[
    \phi(t_1) = (\cos t_1, \sin t_1), \quad \phi(t_2) = (\cos t_2, \sin t_2).
    \]
    \[
    \phi(t_1) \oplus \phi(t_2) = (\cos t_1 \cos t_2 - \sin t_1 \sin t_2, \cos t_1 \sin t_2 + \sin t_1 \cos t_2).
    \]
    
    Compare the results:
    \[
    \phi(t_1 + t_2) = (\cos t_1 \cos t_2 - \sin t_1 \sin t_2, \cos t_1 \sin t_2 + \sin t_1 \cos t_2) = \phi(t_1) \oplus \phi(t_2).
    \]
    
    Thus, \(\phi\) preserves the group operation, and \(\phi\) is a homomorphism.
    
    \end{proof}
    
    \subsection*{Surjectivity}
    
    To show that \(\phi\) is surjective, we need to show that for any \((x, y) \in C(\R)\), there exists \(t \in \R\) such that \(\phi(t) = (x, y)\).
    
    \begin{proof} \( \)
    
    Let \((x, y) \in C(\R)\). By definition, \((x, y)\) satisfies:
    \[
    x^2 + y^2 = 1.
    \]
    
    Choose \(t \in \R\) such that:
    \[
    \cos t = x, \quad \sin t = y.
    \]
    
    Since \((x, y)\) lies on the unit circle, there exists such \(t\). Thus:
    \[
    \phi(t) = (\cos t, \sin t) = (x, y).
    \]
    
    Therefore, \(\phi\) is surjective.
    
    \end{proof}
    
    \subsection*{Kernel of \(\phi\)}
    
    The kernel of \(\phi\) is the set of elements in \(\R\) that are mapped to the identity element in \(C(\R)\) under \(\phi\). The identity element in \(C(\R)\) is \((1, 0)\).
    
    \begin{proof} \( \)
    
    Determine the kernel of \(\phi\):
    \[
    \ker(\phi) = \{t \in \R \mid \phi(t) = (1, 0)\}.
    \]
    
    \[
    \phi(t) = (\cos t, \sin t) = (1, 0) \implies \cos t = 1 \text{ and } \sin t = 0.
    \]
    
    The solutions to \(\cos t = 1\) and \(\sin t = 0\) are:
    \[
    t = 2k\pi \quad \text{for some } k \in \Z.
    \]
    
    Thus:
    \[
    \ker(\phi) = \{2k\pi \mid k \in \Z\}.
    \]
    
    
    \subsection*{Conclusion}
    
    The function \(\phi(t) := (\cos t, \sin t)\) defines a homomorphism \(\phi: (\R, +) \rightarrow (C(\R), \oplus)\) that is surjective. The kernel of \(\phi\) is:
    \[
        \ker(\phi) = \{2k\pi \mid k \in \Z\}.
        \]
        
    \end{proof}
    \end{exercise}
    \newpage
    

%-----------------------------

\end{document}