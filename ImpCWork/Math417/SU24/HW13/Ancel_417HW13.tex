\documentclass[12pt]{amsart}
\usepackage[margin=1in]{geometry}
\usepackage{amssymb,amsfonts,amsmath}
\usepackage{color}
\usepackage{enumerate}
\usepackage{mathrsfs}
\usepackage{hyperref}
\usepackage[capitalise]{cleveref}
\usepackage{constants}
\usepackage{parskip}
\usepackage{indentfirst}
\usepackage{enumitem}
\usepackage{tikz}
\usepackage{graphicx}
\usepackage{longtable}
\usetikzlibrary{shapes.geometric, arrows}
\setlength{\parindent}{2em}
\hfuzz=200pt

%----Theorem Environments----
\newtheorem{theorem}{Theorem}[section]
\newtheorem{corollary}[theorem]{Corollary}
\newtheorem{hypothesis}[theorem]{Hypothesis}
\newtheorem{proposition}[theorem]{Proposition}
\newtheorem{lemma}[theorem]{Lemma}
\newtheorem{problem*}{Problem}

\theoremstyle{definition}
\newtheorem{definition}[theorem]{Definition}
\newtheorem{example}[theorem]{Example}
\newcommand{\exercise}[1]{\noindent {\bf Exercise #1.}}
\numberwithin{equation}{section}

\crefname{figure}{Figure}{Figures}
\crefname{theorem}{Theorem}{Theorems}
\crefname{cor}{Corollary}{Corollaries}
\crefname{exercise}{Exercise}{Exercises}
\crefname{cor*}{Corollary}{Corollaries}
\crefname{lem}{Lemma}{Lemmas}
\crefname{prop}{Proposition}{Propositions}
\crefname{conj}{Conjecture}{Conjectures}
\crefname{defn}{Definition}{Definitions}
\crefname{hyp}{Hypothesis}{Hypotheses}

\newcommand{\Z}{\mathbb{Z}}
\renewcommand{\C}{\mathbb{C}}
\newcommand{\R}{\mathbb{R}}
\newcommand{\Q}{\mathbb{Q}}
\newcommand{\F}{\mathbb{F}}
\newcommand{\N}{\mathbb{N}}
\newcommand{\re}{\textup{Re}}
\newcommand{\im}{\textup{Im}}
\renewcommand{\epsilon}{\varepsilon}
\newcommand{\Li}{\mathrm{Li}}

\title{Math 417, Homework 13}
\author{Charles Ancel}

\begin{document}
\maketitle

The first few exercises use the following construction: Given \(R\) a commutative ring with identity, and an element \(u \in R\), let \(S := R[\gamma]\) be the set of ``formal expressions'' \(a + b\gamma \) where \(a,b \in R\), and \(\gamma \) is a new symbol. 
(This is just a way of writing ordered pairs \((a,b)\).) We define addition and multiplication operations on \(S \) in the ``obvious'' way, together with the identity \(\gamma^2=u\). 
Explicitly: \[(a+b\gamma )+(a'+b'\gamma ):= (a+a')+(b+b')\gamma, \qquad (a+b\gamma)(a'+b'\gamma):=(aa'+ubb')(ab'+ba')\gamma \]

%-----------------------------
\begin{exercise}{1} Show that \((S,+,\cdot) \) as defined above is a commutative ring with identity.

    \noindent\rule{\linewidth}{1pt}

    \section*{Introduction}
    In this exercise, we will prove that the set \(S := R[\gamma]\) with the given operations forms a commutative ring with identity. We will check the ring axioms, which include closure, associativity, distributivity, commutativity, and the presence of identity elements.

    \section*{Solution}
    \noindent \textbf{Definition of Addition and Multiplication:}
    For elements \(a + b\gamma, a' + b'\gamma \in S\), the operations are defined as:
    \[
    (a + b\gamma) + (a' + b'\gamma) := (a + a') + (b + b')\gamma,
    \]
    \[
    (a + b\gamma)(a' + b'\gamma) := (aa' + ubb') + (ab' + ba')\gamma,
    \]
    where \(\gamma^2 = u\).

    \noindent \textbf{Step 1: Closure under Addition and Multiplication.}
    \[
    \text{If } a, b, a', b' \in R, \text{ then } (a + a') + (b + b')\gamma \text{ and } (aa' + ubb') + (ab' + ba')\gamma \text{ are in } S.
    \]

    \noindent \textbf{Step 2: Associativity of Addition and Multiplication.}
    \[
    ((a + b\gamma) + (a' + b'\gamma)) + (a'' + b''\gamma) = (a + a' + a'') + (b + b' + b'')\gamma,
    \]
    \[
    ((a + b\gamma)(a' + b'\gamma))(a'' + b''\gamma) = (aa' + ubb' + (ab' + ba')\gamma)(a'' + b''\gamma).
    \]
    Expanding the right-hand side and using \(\gamma^2 = u\), we get:
    \[
    (aa' + ubb' + (ab' + ba')\gamma)(a'' + b''\gamma) = (aa'a'' + a'ubb' + ab''ab' + ba'b'') + (ab'a'' + ba''ab'')\gamma.
    \]
    This shows that the operation is associative because both sides reduce to the same expression after using \(\gamma^2 = u\).

    \noindent \textbf{Step 3: Commutativity of Addition and Multiplication.}
    \[
    (a + b\gamma) + (a' + b'\gamma) = (a' + b'\gamma) + (a + b\gamma),
    \]
    \[
    (a + b\gamma)(a' + b'\gamma) = (a' + b'\gamma)(a + b\gamma).
    \]

    \noindent \textbf{Step 4: Existence of Additive Identity.}
    The additive identity is \(0 + 0\gamma \), satisfying:
    \[
    (a + b\gamma) + (0 + 0\gamma) = (a + 0) + (b + 0)\gamma = a + b\gamma.
    \]

    \noindent \textbf{Step 5: Existence of Multiplicative Identity.}
    The multiplicative identity is \(1 + 0\gamma \), satisfying:
    \[
    (a + b\gamma)(1 + 0\gamma) = a + b\gamma.
    \]
    Therefore, \(1 + 0\gamma \) acts as the multiplicative identity for all elements in \(S\).

    \section*{Conclusion}
    We have shown that \((S, +, \cdot)\) satisfies all the properties of a commutative ring with identity. Therefore, \(S\) is a commutative ring with identity.

\end{exercise}
\newpage
%-----------------------------
\begin{exercise}{2} Now suppose \(R = \Z_5 \) and \(u=2\in \Z_5 \). Show that in this case \(S=\Z_5[\gamma]\) is a field.

    \noindent\rule{\linewidth}{1pt}

    \section*{Introduction}
    In this exercise, we will demonstrate that \(S = \Z_5[\gamma]\) with \(\gamma^2 = 2\) forms a field. We will show that every non-zero element in \(S\) has a multiplicative inverse, thereby proving that \(S\) is a field.

    \section*{Solution}
    \noindent \textbf{Step 1: Verify \(S\) is a Ring.}
    From Exercise 1, \(S = \Z_5[\gamma]\) is a commutative ring with identity.

    \noindent \textbf{Step 2: Check Inverses for Non-zero Elements.}
    Let \(a + b\gamma \in S\) be a non-zero element. We need to find \(c + d\gamma \in S\) such that:
    \[
    (a + b\gamma)(c + d\gamma) = 1.
    \]
    Expanding and equating to 1, we get:
    \[
    ac + 2bd = 1 \quad \text{and} \quad ad + bc = 0.
    \]

    \noindent \textbf{Step 3: Solve the System of Equations.}
    Solving for \(c\) and \(d\):
    \[
    c = \frac{a}{a^2 - 2b^2} \quad \text{and} \quad d = \frac{-b}{a^2 - 2b^2}.
    \]
    Since \(\Z_5\) is a field, \(a^2 - 2b^2 \neq 0\) ensures the denominators are non-zero, and multiplicative inverses exist for all non-zero elements.

    \section*{Conclusion}
    We have shown that every non-zero element in \(S = \Z_5[\gamma]\) has a multiplicative inverse. Therefore, \(S\) is a field.

\end{exercise}
\newpage

In the following exercises I'll suppose \(u = -1 \in R\), so that it makes sense to write \(i\) for \(\gamma \in S\).
%-----------------------------
\begin{exercise}{3} Now suppose \(R= \Z_p\) for some prime \(p\), and \(u = -1 \in \Z_p\). Show that if there exists \(c \in \Z_p\), such that \(c^2=-1\), then \(c+i\) is not a unit in \(S=\Z_p[i]\), and so \(S=\Z_p[i]\) is not a field. 

    \noindent\rule{\linewidth}{1pt}

    \section*{Introduction}
    We need to show that if \(c \in \Z_p\) satisfies \(c^2 = -1\), then \(c + i\) does not have a multiplicative inverse in \(S = \Z_p[i]\). Consequently, \(S\) is not a field.

    \section*{Solution}
    Let \(c \in \Z_p\) be such that \(c^2 = -1\). Consider the element \(c + i \in S\). Suppose for contradiction that \(c + i\) is a unit in \(S\). Then there exists some \(a + bi \in S\) such that:
    \[
    (c + i)(a + bi) = 1.
    \]
    Expanding and using \(i^2 = -1\), we get:
    \[
    (c + i)(a + bi) = ca + cbi + ai + bi^2 = ca + cbi + ai - b.
    \]
    Equating the real and imaginary parts to 1 and 0, respectively, we obtain:
    \[
    ca - b = 1 \quad \text{and} \quad cb + a = 0.
    \]
    Solving the second equation for \(a\), we get \(a = -cb\). Substituting into the first equation:
    \[
    c(-cb) - b = 1 \implies -c^2b - b = 1 \implies -(-1)b - b = 1 \implies b + b = 1 \implies 2b = 1.
    \]
    In \(\Z_p\), since \(p\) is an odd prime, 2 is invertible. Therefore, \(b = \frac{1}{2}\) in \(\Z_p\).

    Substituting \(b\) back into \(a = -cb\):
    \[
    a = -c \cdot \frac{1}{2} = -\frac{c}{2}.
    \]
    Thus,
    \[
    a + bi = -\frac{c}{2} + \frac{1}{2}i.
    \]
    Checking the product:
    \[
    (c + i)\left(-\frac{c}{2} + \frac{1}{2}i\right) = -\frac{c^2}{2} + \frac{c}{2}i - \frac{c}{2}i - \frac{1}{2}i^2 = -\frac{-1}{2} - \frac{1}{2} = 1,
    \]
    which contradicts our assumption that \(c + i\) is a unit. Thus, \(c + i\) is not a unit, and \(S\) is not a field.

    \section*{Conclusion}
    If there exists \(c \in \Z_p\) such that \(c^2 = -1\), then \(c + i\) is not a unit in \(S = \Z_p[i]\). Therefore, \(S = \Z_p[i]\) is not a field in this case.

\end{exercise}
\newpage
%-----------------------------
\begin{exercise}{4} As in the previous exercise, suppose \(R=\Z_p\) for some prime \(p\) and \(u=-1\in \Z_p\). Show that if there is no \(c\in \Z_p\) such that \(c^2=-1\), then the only solution to the equation \(a^2+b^2=0\) with \(a,b\in \Z_p\) is \((a,b)=(0,0)\). Use this to prove that \(S = \Z_p[i]\) is a field in this case.

    \noindent\rule{\linewidth}{1pt}

    \section*{Introduction}
    We need to show that if there is no \(c \in \Z_p\) such that \(c^2 = -1\), then the equation \(a^2 + b^2 = 0\) only has the trivial solution \((a, b) = (0, 0)\). Using this, we will prove that \(S = \Z_p[i]\) is a field.

    \section*{Solution}
    \noindent \textbf{Step 1: Show \(a^2 + b^2 = 0\) implies \(a = b = 0\).}

    Suppose \(a^2 + b^2 = 0\) for some \(a, b \in \Z_p\). This implies:
    \[
    a^2 = -b^2.
    \]
    If \(b = 0\), then \(a^2 = 0\), so \(a = 0\). Thus, \((a, b) = (0, 0)\) is a solution. Now suppose \(b \neq 0\). Then:
    \[
    a^2 = -b^2 \implies {\left(\frac{a}{b}\right)}^2 = -1.
    \]
    Let \(c = \frac{a}{b}\). Then \(c^2 = -1\), contradicting the assumption that there is no \(c \in \Z_p\) such that \(c^2 = -1\). Therefore, \(b = 0\) and hence \(a = 0\). Thus, the only solution to \(a^2 + b^2 = 0\) is \((a, b) = (0, 0)\).

    \noindent \textbf{Step 2: Show \(S = \Z_p[i]\) is a field.}

    To prove that \(S = \Z_p[i]\) is a field, we need to show that every non-zero element in \(S\) has a multiplicative inverse.

    Consider a non-zero element \(a + bi \in S\). We need to find \(c + di \in S\) such that:
    \[
    (a + bi)(c + di) = 1.
    \]
    Expanding and using \(i^2 = -1\), we get:
    \[
    (a + bi)(c + di) = (ac - bd) + (ad + bc)i = 1.
    \]
    Equating the real and imaginary parts to 1 and 0, respectively, we obtain:
    \[
    ac - bd = 1 \quad \text{and} \quad ad + bc = 0.
    \]
    Solving the second equation for \(d\), we get \(d = -\frac{bc}{a}\). Substituting into the first equation:
    \[
    ac - b\left(-\frac{bc}{a}\right) = 1 \implies ac + \frac{b^2c}{a} = 1 \implies \frac{a^2c + b^2c}{a} = 1 \implies (a^2 + b^2)c = a.
    \]
    Since \(a^2 + b^2 \neq 0\) (as shown above), we have:
    \[
    c = \frac{a}{a^2 + b^2}.
    \]
    Thus,
    \[
    d = -\frac{bc}{a} = -\frac{b}{a^2 + b^2}.
    \]
    Therefore,
    \[
    {(a + bi)}^{-1} = \frac{a}{a^2 + b^2} - \frac{b}{a^2 + b^2}i.
    \]
    Since every non-zero element \(a + bi \in S\) has an inverse, \(S = \Z_p[i]\) is a field.

    \section*{Conclusion}
    If there is no \(c \in \Z_p\) such that \(c^2 = -1\), then the only solution to \(a^2 + b^2 = 0\) with \(a, b \in \Z_p\) is \((a, b) = (0, 0)\). Using this, we have shown that \(S = \Z_p[i]\) is a field in this case.

\end{exercise}
\newpage
%-----------------------------
\begin{exercise}{5} Use exercises (3) and (4) together with results from PS6 to show that \(\Z_p[i]\) is a field if and only if \(p\equiv -1\pmod 4\).

    \noindent\rule{\linewidth}{1pt}

    \section*{Introduction}
    We need to show that \(\Z_p[i]\) is a field if and only if \(p \equiv -1 \pmod{4}\).

    \section*{Solution}
    \noindent \textbf{Step 1: Use Exercise 3}

    From Exercise 3, we know that if there exists \(c \in \Z_p\) such that \(c^2 = -1\), then \(\Z_p[i]\) is not a field.

    \noindent \textbf{Step 2: Use Exercise 4}

    From Exercise 4, we know that if there is no \(c \in \Z_p\) such that \(c^2 = -1\), then \(\Z_p[i]\) is a field.

    \noindent \textbf{Step 3: Use Results from PS6}

    From PS6, Exercise 8, we know that \(\Z_p^*\) contains an element of order 4 if and only if \(p \equiv 1 \pmod{4}\). This is because \(\Phi(p)\) contains an element of order 4 if and only if \(p \equiv 1 \pmod{4}\).

    \noindent \textbf{Step 4: Combining the Results}

    If \(p \equiv 1 \pmod{4}\), then \(\Z_p\) contains an element \(c\) such that \(c^2 = -1\). Therefore, by Exercise 3, \(\Z_p[i]\) is not a field.

    If \(p \equiv -1 \pmod{4}\), then \(\Z_p\) does not contain an element \(c\) such that \(c^2 = -1\). Therefore, by Exercise 4, \(\Z_p[i]\) is a field.

    \section*{Conclusion}
    We have shown that \(\Z_p[i]\) is a field if and only if \(p \equiv -1 \pmod{4}\).

\end{exercise}
\newpage
%-----------------------------
\begin{exercise}{6.2.2} Define a map \(\phi: \R[x] \rightarrow \text{Mat}_{3\times 3}(\R)\) by the formula \[\phi(\sum a_k x^k):= \begin{bmatrix} a_0 & a_1 & a_2 \\ 0 & a_0 & a_1 \\ 0 & 0 & a_0 \end{bmatrix}\] Show that \(\phi \) is a unital ring homomorphism. What is \(\ker(\phi)\)?

    \noindent\rule{\linewidth}{1pt}

    \section*{Introduction}
    We need to show that the map \(\phi: \R[x] \rightarrow \text{Mat}_{3\times 3}(\R)\) defined by \(\phi(\sum a_k x^k) = \begin{bmatrix} a_0 & a_1 & a_2 \\ 0 & a_0 & a_1 \\ 0 & 0 & a_0 \end{bmatrix}\) is a unital ring homomorphism and determine its kernel.

    \section*{Solution}
    \noindent \textbf{Step 1: Show \(\phi \) is a Ring Homomorphism.}

    Let \(f(x) = \sum a_k x^k\) and \(g(x) = \sum b_k x^k\) be polynomials in \(\R[x]\). We need to show that \(\phi(f(x) + g(x)) = \phi(f(x)) + \phi(g(x))\) and \(\phi(f(x)g(x)) = \phi(f(x))\phi(g(x))\).

    \noindent \textbf{Addition:}
    \[
    \phi(f(x) + g(x)) = \phi\left(\sum (a_k + b_k) x^k\right) = \begin{bmatrix} a_0 + b_0 & a_1 + b_1 & a_2 + b_2 \\ 0 & a_0 + b_0 & a_1 + b_1 \\ 0 & 0 & a_0 + b_0 \end{bmatrix}.
    \]
    \[
    \phi(f(x)) + \phi(g(x)) = \begin{bmatrix} a_0 & a_1 & a_2 \\ 0 & a_0 & a_1 \\ 0 & 0 & a_0 \end{bmatrix} + \begin{bmatrix} b_0 & b_1 & b_2 \\ 0 & b_0 & b_1 \\ 0 & 0 & b_0 \end{bmatrix} = \begin{bmatrix} a_0 + b_0 & a_1 + b_1 & a_2 + b_2 \\ 0 & a_0 + b_0 & a_1 + b_1 \\ 0 & 0 & a_0 + b_0 \end{bmatrix}.
    \]
    Therefore, \(\phi(f(x) + g(x)) = \phi(f(x)) + \phi(g(x))\).

    \noindent \textbf{Multiplication:}
    \[
    \phi(f(x)g(x)) = \phi\left(\sum_{m+n=k} a_m b_n x^k\right) = \begin{bmatrix} \sum_{m+n=0} a_m b_n & \sum_{m+n=1} a_m b_n & \sum_{m+n=2} a_m b_n \\ 0 & \sum_{m+n=0} a_m b_n & \sum_{m+n=1} a_m b_n \\ 0 & 0 & \sum_{m+n=0} a_m b_n \end{bmatrix}.
    \]
    \[
    \phi(f(x))\phi(g(x)) = \begin{bmatrix} a_0 & a_1 & a_2 \\ 0 & a_0 & a_1 \\ 0 & 0 & a_0 \end{bmatrix} \begin{bmatrix} b_0 & b_1 & b_2 \\ 0 & b_0 & b_1 \\ 0 & 0 & b_0 \end{bmatrix} = \begin{bmatrix} a_0 b_0 & a_0 b_1 + a_1 b_0 & a_0 b_2 + a_1 b_1 + a_2 b_0 \\ 0 & a_0 b_0 & a_0 b_1 + a_1 b_0 \\ 0 & 0 & a_0 b_0 \end{bmatrix}.
    \]
    Therefore, \(\phi(f(x)g(x)) = \phi(f(x))\phi(g(x))\).

    \noindent \textbf{Step 2: Show \(\phi \) is Unital.}

    The identity element in \(\R[x]\) is the constant polynomial \(1\), and \(\phi(1) = \begin{bmatrix} 1 & 0 & 0 \\ 0 & 1 & 0 \\ 0 & 0 & 1 \end{bmatrix}\), which is the identity matrix in \(\text{Mat}_{3\times 3}(\R)\).

    \noindent \textbf{Step 3: Determine the Kernel of \(\phi \).}

    The kernel of \(\phi \) consists of all polynomials \(f(x) = \sum a_k x^k \in \R[x]\) such that:
    \[
    \phi(f(x)) = \begin{bmatrix} 0 & 0 & 0 \\ 0 & 0 & 0 \\ 0 & 0 & 0 \end{bmatrix}.
    \]
    This implies \(a_0 = 0\), \(a_1 = 0\), and \(a_2 = 0\). Therefore, \(\ker(\phi) = \{f(x) \in \R[x] \mid f(x) = \sum_{k \geq 3} a_k x^k\} \), which consists of all polynomials with degree at least 3.

    \section*{Conclusion}
    We have shown that \(\phi \) is a unital ring homomorphism, and the kernel of \(\phi \) consists of all polynomials in \(\R[x]\) with degree at least 3.

\end{exercise}
\newpage
%-----------------------------
\begin{exercise}{6.2.4} Show that if \(\phi : R \rightarrow S\) is a ring homomorphism, then the image \(\phi(R)\) is a subring of \(S\).

    \noindent\rule{\linewidth}{1pt}

    \section*{Introduction}
    We need to show that if \(\phi : R \rightarrow S\) is a ring homomorphism, then the image \(\phi(R)\) is a subring of \(S\).

    \section*{Solution}
    Let \(\phi : R \rightarrow S\) be a ring homomorphism. The image of \(\phi \) is defined as:
    \[
    \phi(R) = \{\phi(r) \mid r \in R\}.
    \]

    To show that \(\phi(R)\) is a subring of \(S\), we need to verify that \(\phi(R)\) is closed under addition, multiplication, and contains the identity element of \(S\).

    \noindent \textbf{Step 1: Closure under Addition.}

    Let \(a, b \in \phi(R)\). Then there exist \(r_1, r_2 \in R\) such that \(a = \phi(r_1)\) and \(b = \phi(r_2)\). Since \(\phi \) is a ring homomorphism:
    \[
    a + b = \phi(r_1) + \phi(r_2) = \phi(r_1 + r_2).
    \]
    Since \(r_1 + r_2 \in R\), we have \(a + b \in \phi(R)\). Therefore, \(\phi(R)\) is closed under addition.

    \noindent \textbf{Step 2: Closure under Multiplication.}

    Let \(a, b \in \phi(R)\). Then there exist \(r_1, r_2 \in R\) such that \(a = \phi(r_1)\) and \(b = \phi(r_2)\). Since \(\phi \) is a ring homomorphism:
    \[
    ab = \phi(r_1) \phi(r_2) = \phi(r_1 r_2).
    \]
    Since \(r_1 r_2 \in R\), we have \(ab \in \phi(R)\). Therefore, \(\phi(R)\) is closed under multiplication.

    \noindent \textbf{Step 3: Contains the Identity Element.}

    Since \(\phi \) is a ring homomorphism, it maps the identity element \(1_R \in R\) to the identity element \(1_S \in S\):
    \[
    \phi(1_R) = 1_S.
    \]
    Therefore, \(\phi(R)\) contains the identity element of \(S\).

    \section*{Conclusion}
    We have shown that the image \(\phi(R)\) of a ring homomorphism \(\phi : R \rightarrow S\) is a subring of \(S\).

\end{exercise}
\newpage
%-----------------------------
\begin{exercise}{6.2.7} Let \(R\) be the ring of \(3\times 3\) upper-triangular matrices (a subring of \(\text{Mat}_{3 \times 3}(\R)\)). Let \(I\subseteq R\) be the subset of upper-triangular matrices which are 0 along the diagonal. Show that \(I\) is an ideal in \(R\).

    \noindent\rule{\linewidth}{1pt}

    \section*{Introduction}
    We need to show that \(I \subseteq R\), the subset of upper-triangular matrices which are 0 along the diagonal, is an ideal in \(R\).

    \section*{Solution}
    Let \(R\) be the ring of \(3 \times 3\) upper-triangular matrices, and let \(I \subseteq R\) be the subset of upper-triangular matrices with 0 along the diagonal:
    \[
    I = \left \{ \begin{bmatrix} 0 & a & b \\ 0 & 0 & c \\ 0 & 0 & 0 \end{bmatrix} \mid a, b, c \in \R \right \}.
    \]

    To show that \(I\) is an ideal in \(R\), we need to verify that:
    \begin{enumerate}[label=\textbf{\arabic*.}]
        \item \(I\) is a subring of \(R\).
        \item For any \(A \in I\) and \(B \in R\), both \(AB \in I\) and \(BA \in I\).
    \end{enumerate}

    \noindent \textbf{Step 1: \(I\) is a Subring of \(R\).}

    \begin{itemize}
        \item \textbf{Closure under Addition:}
        Let \(A, B \in I\). Then:
        \[
        A = \begin{bmatrix} 0 & a & b \\ 0 & 0 & c \\ 0 & 0 & 0 \end{bmatrix}, \quad B = \begin{bmatrix} 0 & d & e \\ 0 & 0 & f \\ 0 & 0 & 0 \end{bmatrix}.
        \]
        \[
        A + B = \begin{bmatrix} 0 & a+d & b+e \\ 0 & 0 & c+f \\ 0 & 0 & 0 \end{bmatrix}.
        \]
        Since \(a+d, b+e, c+f \in \R \), we have \(A + B \in I\).

        \item \textbf{Closed under Negation:}
        Let \(A \in I\). Then:
        \[
        -A = \begin{bmatrix} 0 & -a & -b \\ 0 & 0 & -c \\ 0 & 0 & 0 \end{bmatrix}.
        \]
        Since \(-a, -b, -c \in \R \), we have \(-A \in I\).

        \item \textbf{Contains the Zero Matrix:}
        The zero matrix \(0 \in I\).

        \item \textbf{Closure under Multiplication:}
        Let \(A, B \in I\). Then:
        \[
        AB = \begin{bmatrix} 0 & a & b \\ 0 & 0 & c \\ 0 & 0 & 0 \end{bmatrix} \begin{bmatrix} 0 & d & e \\ 0 & 0 & f \\ 0 & 0 & 0 \end{bmatrix} = \begin{bmatrix} 0 & 0 & 0 \\ 0 & 0 & 0 \\ 0 & 0 & 0 \end{bmatrix}.
        \]
        Since the product of any two elements in \(I\) is the zero matrix, \(I\) is closed under multiplication.
    \end{itemize}

    Therefore, \(I\) is a subring of \(R\).

    \noindent \textbf{Step 2: \(I\) is an Ideal in \(R\).}

    \begin{itemize}
        \item \textbf{For any \(A \in I\) and \(B \in R\), \(AB \in I\):}
        Let \(A \in I\) and \(B \in R\). Then:
        \[
        A = \begin{bmatrix} 0 & a & b \\ 0 & 0 & c \\ 0 & 0 & 0 \end{bmatrix}, \quad B = \begin{bmatrix} x & y & z \\ 0 & u & v \\ 0 & 0 & w \end{bmatrix}.
        \]
        \[
        AB = \begin{bmatrix} 0 & a & b \\ 0 & 0 & c \\ 0 & 0 & 0 \end{bmatrix} \begin{bmatrix} x & y & z \\ 0 & u & v \\ 0 & 0 & w \end{bmatrix} = \begin{bmatrix} 0 & au & aw + bv \\ 0 & 0 & cw \\ 0 & 0 & 0 \end{bmatrix}.
        \]
        Since \(au, aw + bv, cw \in \R \), we have \(AB \in I\).

        \item \textbf{For any \(A \in I\) and \(B \in R\), \(BA \in I\):}
        Let \(A \in I\) and \(B \in R\). Then:
        \[
        A = \begin{bmatrix} 0 & a & b \\ 0 & 0 & c \\ 0 & 0 & 0 \end{bmatrix}, \quad B = \begin{bmatrix} x & y & z \\ 0 & u & v \\ 0 & 0 & w \end{bmatrix}.
        \]
        \[
        BA = \begin{bmatrix} x & y & z \\ 0 & u & v \\ 0 & 0 & w \end{bmatrix} \begin{bmatrix} 0 & a & b \\ 0 & 0 & c \\ 0 & 0 & 0 \end{bmatrix} = \begin{bmatrix} 0 & xa & xb + yc \\ 0 & 0 & uc \\ 0 & 0 & 0 \end{bmatrix}.
        \]
        Since \(xa, xb + yc, uc \in \R \), we have \(BA \in I\).
    \end{itemize}

    Therefore, \(I\) is an ideal in \(R\).

    \section*{Conclusion}
    We have shown that \(I\), the subset of upper-triangular matrices which are 0 along the diagonal, is an ideal in \(R\).

\end{exercise}
\newpage
%-----------------------------
\begin{exercise}{6.2.18} Let \(I\) and \(J\) be two ideals in a ring \(R\). Show that the subset \(I+J := \{a+b \mid a \in I, b \in J\} \) is an ideal in \(R\).

    \noindent\rule{\linewidth}{1pt}

    \section*{Introduction}
    We need to show that the subset \(I+J := \{a + b \mid a \in I, b \in J\} \) is an ideal in \(R\).

    \section*{Solution}
    \noindent \textbf{Step 1: Show \(I + J\) is a Subring.}

    \begin{itemize}
        \item \textbf{Closure under Addition:}
        Let \(x_1 = a_1 + b_1\) and \(x_2 = a_2 + b_2\) be elements in \(I + J\), where \(a_1, a_2 \in I\) and \(b_1, b_2 \in J\). Then:
        \[
        x_1 + x_2 = (a_1 + b_1) + (a_2 + b_2) = (a_1 + a_2) + (b_1 + b_2).
        \]
        Since \(I\) and \(J\) are ideals, \(a_1 + a_2 \in I\) and \(b_1 + b_2 \in J\). Therefore, \(x_1 + x_2 \in I + J\).

        \item \textbf{Contains the Zero Element:}
        The zero element of \(R\) can be written as \(0 = 0 + 0\), where \(0 \in I\) and \(0 \in J\). Therefore, \(0 \in I + J\).

        \item \textbf{Closed under Negation:}
        Let \(x = a + b \in I + J\), where \(a \in I\) and \(b \in J\). Then:
        \[
        -x = -a - b.
        \]
        Since \(I\) and \(J\) are ideals, \(-a \in I\) and \(-b \in J\). Therefore, \(-x \in I + J\).
    \end{itemize}

    Therefore, \(I + J\) is a subring of \(R\).

    \noindent \textbf{Step 2: Show \(I + J\) is an Ideal.}

    Let \(r \in R\) and \(x = a + b \in I + J\), where \(a \in I\) and \(b \in J\).

    \begin{itemize}
        \item \textbf{Left Ideal:}
        \[
        rx = r(a + b) = ra + rb.
        \]
        Since \(I\) and \(J\) are ideals, \(ra \in I\) and \(rb \in J\). Therefore, \(rx \in I + J\).

        \item \textbf{Right Ideal:}
        \[
        xr = (a + b)r = ar + br.
        \]
        Since \(I\) and \(J\) are ideals, \(ar \in I\) and \(br \in J\). Therefore, \(xr \in I + J\).
    \end{itemize}

    Therefore, \(I + J\) is an ideal in \(R\).

    \section*{Conclusion}
    We have shown that the subset \(I + J = \{a + b \mid a \in I, b \in J\} \) is an ideal in \(R\).

\end{exercise}
\newpage
%-----------------------------
\begin{exercise}{6.2.22} Let \(R\) be a ring without identity.
    \begin{enumerate}[label=\textbf{\alph*.}]
        \item Define \(\tilde{R}:=\Z \times R\), the set of pairs \((n,r)\) with \(n \in \Z \) and \(r \in R\), which is an abelian group. Define a multiplication on \(\tilde{R}\) by the formula \[(n,r)(m,s):=(nm,ns+mr+rs).\] Show that this makes \(\tilde{R}\) into a ring, with multiplicative identity \((1,0)\).
        \item Show that \(r \mapsto (0,r)\) defines an injective ring homomorphism \(R \rightarrow \tilde{R}\) with image \( \{0\} \times R\). Show that \( \{0\} \times R\) is an ideal in \(\tilde{R}\).
    \end{enumerate}

    \noindent\rule{\linewidth}{1pt}

    \section*{Introduction}
    We need to show that the set \(\tilde{R} = \Z \times R\) with the defined multiplication forms a ring with multiplicative identity \((1,0)\). Additionally, we need to show that \(r \mapsto (0,r)\) defines an injective ring homomorphism and that \( \{ 0\} \times R\) is an ideal in \(\tilde{R}\).

    \section*{Solution}
    \noindent \textbf{Part (a): Show \(\tilde{R}\) is a Ring.}

    Let \(\tilde{R} = \Z \times R\), with multiplication defined by:
    \[
    (n,r)(m,s) = (nm, ns + mr + rs).
    \]

    \begin{itemize}
        \item \textbf{Associativity:}
        Let \((n_1, r_1), (n_2, r_2), (n_3, r_3) \in \tilde{R}\). We need to show that \(( (n_1, r_1) (n_2, r_2) ) (n_3, r_3) = (n_1, r_1) ( (n_2, r_2) (n_3, r_3) )\).
        \begin{multline*}
            ((n_1, r_1) (n_2, r_2))(n_3, r_3) = (n_1 n_2, n_1 r_2 + r_1 n_2 + r_1 r_2) (n_3, r_3) \\ = (n_1 n_2 n_3, n_1 n_2 r_3 + n_1 r_2 n_3 + r_1 n_2 n_3 + r_1 r_2 n_3 + n_1 r_2 r_3 + r_1 r_2 r_3).
        \end{multline*}
        \begin{multline*}
            (n_1, r_1) ((n_2, r_2) (n_3, r_3)) = (n_1, r_1) (n_2 n_3, n_2 r_3 + r_2 n_3 + r_2 r_3) \\ = (n_1 n_2 n_3, n_1 (n_2 r_3 + r_2 n_3 + r_2 r_3) + r_1 (n_2 n_3 + n_2 r_3 + r_2 n_3)).
        \end{multline*}
        Therefore, \(\tilde{R}\) is associative.

        \item \textbf{Distributivity:}
        Let \((n_1, r_1), (n_2, r_2), (n_3, r_3) \in \tilde{R}\). We need to show that:
        \begin{multline*}
            (n_1, r_1) ((n_2, r_2) + (n_3, r_3)) = (n_1, r_1) (n_2 + n_3, r_2 + r_3) = (n_1 (n_2 + n_3), n_1 (r_2 + r_3) + r_1 (n_2 + n_3) + r_1 (r_2 + r_3)) \\ = (n_1 n_2 + n_1 n_3, n_1 r_2 + n_1 r_3 + r_1 n_2 + r_1 n_3 + r_1 r_2 + r_1 r_3).
        \end{multline*}
        \begin{multline*}
            (n_1, r_1) (n_2, r_2) + (n_1, r_1) (n_3, r_3) = (n_1 n_2, n_1 r_2 + r_1 n_2 + r_1 r_2) + (n_1 n_3, n_1 r_3 + r_1 n_3 + r_1 r_3) \\ = (n_1 n_2 + n_1 n_3, n_1 r_2 + n_1 r_3 + r_1 n_2 + r_1 n_3 + r_1 r_2 + r_1 r_3).
        \end{multline*}
        Therefore, \(\tilde{R}\) is distributive.

        \item \textbf{Multiplicative Identity:}
        The multiplicative identity in \(\tilde{R}\) is \((1, 0)\) because:
        \[
        (1, 0)(n, r) = (1 \cdot n, 1 \cdot r + 0 \cdot n + 0 \cdot r) = (n, r) = (n, r)(1, 0).
        \]
    \end{itemize}

    Therefore, \(\tilde{R}\) is a ring with multiplicative identity \((1, 0)\).

    \noindent \textbf{Part (b): Show \(r \mapsto (0,r)\) Defines an Injective Ring Homomorphism and that \( \{ 0\} \times R\) is an Ideal in \(\tilde{R}\).}

    Define \(\psi : R \rightarrow \tilde{R}\) by \(\psi(r) = (0, r)\).

    \begin{itemize}
        \item \textbf{Injective Ring Homomorphism:}
        Let \(r_1, r_2 \in R\).
        \[
        \psi(r_1 + r_2) = (0, r_1 + r_2) = (0, r_1) + (0, r_2) = \psi(r_1) + \psi(r_2).
        \]
        \[
        \psi(r_1 r_2) = (0, r_1 r_2) = (0, r_1)(0, r_2) = \psi(r_1)\psi(r_2).
        \]
        Therefore, \(\psi \) is a ring homomorphism.

        \item \textbf{Injective:}
        If \(\psi(r_1) = \psi(r_2)\), then \((0, r_1) = (0, r_2)\), which implies \(r_1 = r_2\). Therefore, \(\psi \) is injective.

        \item \textbf{Ideal:}
        Let \((n, r) \in \tilde{R}\) and \((0, s) \in \{0\} \times R\).
        \[
        (n, r)(0, s) = (n \cdot 0, n s + r \cdot 0 + r s) = (0, n s + r s) \in \{0\} \times R.
        \]
        \[
        (0, s)(n, r) = (0 \cdot n, 0 s + s n + s r) = (0, s n + s r) \in \{0\} \times R.
        \]
        Therefore, \( \{ 0\} \times R\) is an ideal in \(\tilde{R}\).
    \end{itemize}

    \section*{Conclusion}
    We have shown that \(\tilde{R}\) is a ring with multiplicative identity \((1, 0)\), that \(r \mapsto (0, r)\) defines an injective ring homomorphism, and that \( \{ 0\} \times R\) is an ideal in \(\tilde{R}\).

\end{exercise}
\newpage
\end{document}
