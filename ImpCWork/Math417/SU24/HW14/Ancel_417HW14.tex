\documentclass[12pt]{amsart}
\usepackage[margin=1in]{geometry}
\usepackage{amssymb,amsfonts,amsmath}
\usepackage{color}
\usepackage{enumerate}
\usepackage{mathrsfs}
\usepackage{hyperref}
\usepackage[capitalise]{cleveref}
\usepackage{constants}
\usepackage{parskip}
\usepackage{indentfirst}
\usepackage{enumitem}
\usepackage{tikz}
\usepackage{graphicx}
\usepackage{longtable}
\usetikzlibrary{shapes.geometric, arrows}
\setlength{\parindent}{2em}
\hfuzz=200pt

%----Theorem Environments----
\newtheorem{theorem}{Theorem}[section]
\newtheorem{corollary}[theorem]{Corollary}
\newtheorem{hypothesis}[theorem]{Hypothesis}
\newtheorem{proposition}[theorem]{Proposition}
\newtheorem{lemma}[theorem]{Lemma}
\newtheorem{problem*}{Problem}

\theoremstyle{definition}
\newtheorem{definition}[theorem]{Definition}
\newtheorem{example}[theorem]{Example}
\newcommand{\exercise}[1]{\noindent {\bf Exercise #1.}}
\numberwithin{equation}{section}

\crefname{figure}{Figure}{Figures}
\crefname{theorem}{Theorem}{Theorems}
\crefname{cor}{Corollary}{Corollaries}
\crefname{exercise}{Exercise}{Exercises}
\crefname{cor*}{Corollary}{Corollaries}
\crefname{lem}{Lemma}{Lemmas}
\crefname{prop}{Proposition}{Propositions}
\crefname{conj}{Conjecture}{Conjectures}
\crefname{defn}{Definition}{Definitions}
\crefname{hyp}{Hypothesis}{Hypotheses}

\newcommand{\Z}{\mathbb{Z}}
\renewcommand{\C}{\mathbb{C}}
\newcommand{\R}{\mathbb{R}}
\newcommand{\Q}{\mathbb{Q}}
\newcommand{\F}{\mathbb{F}}
\newcommand{\N}{\mathbb{N}}
\newcommand{\re}{\textup{Re}}
\newcommand{\im}{\textup{Im}}
\renewcommand{\epsilon}{\varepsilon}
\newcommand{\Li}{\mathrm{Li}}

\title{Math 417, Homework 14}
\author{Charles Ancel}

\begin{document}
\maketitle

%-----------------------------
\begin{exercise}{6.3.1} Work out the formula for multiplication in the ring \( \R[x]/I \) in terms of canonical forms, where \( I = (x^2 - 1) \). Note that canonical forms look like: \(a + bx + I\), \(a, b \in \R \).

    \noindent\rule{\linewidth}{1pt}

    \section*{Introduction}
    In this exercise, we will derive the multiplication formula for the ring \( \R[x]/(x^2 - 1) \) using the canonical forms of elements. The canonical forms are \(a + bx + I\) where \(a, b \in \R \).

    \section*{Solution}
    Let \(f(x) = a + bx\) and \(g(x) = c + dx\) be elements in \( \R[x] \). We need to determine the product \(f(x)g(x) \mod (x^2 - 1)\).

    \noindent \textbf{Step 1: Compute the Product \(f(x)g(x)\)}
    \[
    f(x)g(x) = (a + bx)(c + dx) = ac + adx + bcx + bdx^2.
    \]

    \noindent \textbf{Step 2: Reduce \(x^2\) Using the Relation \(x^2 \equiv 1 \mod (x^2 - 1)\)}
    Since \(x^2 \equiv 1 \mod (x^2 - 1)\), we can replace \(x^2\) with 1 in the expression:
    \[
    bdx^2 \equiv bd \mod (x^2 - 1).
    \]

    \noindent \textbf{Step 3: Substitute and Combine Like Terms}
    Substituting \(x^2\) with 1, we get:
    \[
    f(x)g(x) \equiv ac + adx + bcx + bd \mod (x^2 - 1).
    \]
    Grouping like terms, we obtain:
    \[
    f(x)g(x) \equiv (ac + bd) + (ad + bc)x \mod (x^2 - 1).
    \]

    \noindent \textbf{Canonical Form:}
    Therefore, the multiplication formula in \( \R[x]/(x^2 - 1) \) is:
    \[
    (a + bx)(c + dx) \equiv (ac + bd) + (ad + bc)x.
    \]

    \section*{Conclusion}
    In the ring \( \R[x]/(x^2 - 1) \), the product of two canonical forms \(a + bx\) and \(c + dx\) is given by:
    \[
    (a + bx)(c + dx) \equiv (ac + bd) + (ad + bc)x.
    \]

\end{exercise}
\newpage
%-----------------------------
\begin{exercise}{6.3.2} Work out the formula for multiplication in the ring \( \R[x]/I \) in terms of canonical forms, where \( I = (x^3 - 1) \). Note that canonical forms look like: \(a + bx + cx^2 + I\), \(a, b, c \in \R \).

    \noindent\rule{\linewidth}{1pt}

    \section*{Introduction}
    In this exercise, we will derive the multiplication formula for the ring \( \R[x]/(x^3 - 1) \) using the canonical forms of elements. The canonical forms are \(a + bx + cx^2 + I\) where \(a, b, c \in \R \).

    \section*{Solution}
    Let \(f(x) = a + bx + cx^2\) and \(g(x) = d + ex + fx^2\) be elements in \( \R[x] \). We need to determine the product \(f(x)g(x) \mod (x^3 - 1)\).

    \noindent \textbf{Step 1: Compute the Product \(f(x)g(x)\)}
    \[
    f(x)g(x) = (a + bx + cx^2)(d + ex + fx^2).
    \]
    Expanding this, we get:
    \[
    f(x)g(x) = ad + aex + afx^2 + bdx + bex^2 + bfx^3 + cdx^2 + cex^3 + cfx^4.
    \]

    \noindent \textbf{Step 2: Reduce Higher Powers of \(x\) Using the Relation \(x^3 \equiv 1 \mod (x^3 - 1)\)}
    Since \(x^3 \equiv 1 \mod (x^3 - 1)\), we can replace higher powers of \(x\):
    \[
    bfx^3 \equiv bf \mod (x^3 - 1),
    \]
    \[
    cex^3 \equiv ce \mod (x^3 - 1),
    \]
    \[
    cfx^4 = cfx \cdot x^3 \equiv cfx \mod (x^3 - 1).
    \]

    \noindent \textbf{Step 3: Substitute and Combine Like Terms}
    Substituting the reduced terms, we get:
    \[
    f(x)g(x) \equiv ad + bf + aex + afx^2 + bdx + bex^2 + cdx^2 + ce + cfx \mod (x^3 - 1).
    \]
    Grouping like terms, we obtain:
    \[
    f(x)g(x) \equiv (ad + bf + ce) + (ae + bd + cf)x + (af + be + cd)x^2 \mod (x^3 - 1).
    \]

    \noindent \textbf{Canonical Form:}
    Therefore, the multiplication formula in \( \R[x]/(x^3 - 1) \) is:
    \[
    (a + bx + cx^2)(d + ex + fx^2) \equiv (ad + bf + ce) + (ae + bd + cf)x + (af + be + cd)x^2.
    \]

    \section*{Conclusion}
    In the ring \( \R[x]/(x^3 - 1) \), the product of two canonical forms \(a + bx + cx^2\) and \(d + ex + fx^2\) is given by:
    \[
    (a + bx + cx^2)(d + ex + fx^2) \equiv (ad + bf + ce) + (ae + bd + cf)x + (af + be + cd)x^2.
    \]

\end{exercise}
\newpage
%-----------------------------
\begin{exercise}{6.3.10} Let \(R\) be any commutative ring. Show that there is an isomorphism \(R[x]/(x) \simeq R\).

    \noindent\rule{\linewidth}{1pt}

    \section*{Introduction}
    We need to show that there is an isomorphism \(R[x]/(x) \simeq R\) for any commutative ring \(R\). We will use the First Isomorphism Theorem for rings.

    \section*{Solution}
    Consider the ring homomorphism \(\phi: R[x] \to R\) defined by \(\phi(f(x)) = f(0)\). This map evaluates the polynomial \(f(x)\) at \(x = 0\).

    \noindent \textbf{Step 1: Homomorphism \(\phi \)}
    Let \(f(x) = a_0 + a_1 x + a_2 x^2 + \cdots + a_n x^n\). Then,
    \[
    \phi(f(x)) = \phi(a_0 + a_1 x + a_2 x^2 + \cdots + a_n x^n) = a_0.
    \]
    Clearly, \(\phi \) is a ring homomorphism because:
    \[
    \phi(f(x) + g(x)) = (f(x) + g(x))|_{x=0} = f(0) + g(0) = \phi(f(x)) + \phi(g(x)),
    \]
    \[
    \phi(f(x)g(x)) = (f(x)g(x))|_{x=0} = f(0)g(0) = \phi(f(x))\phi(g(x)).
    \]

    \noindent \textbf{Step 2: Kernel of \(\phi \)}
    The kernel of \(\phi \) is:
    \[
    \ker(\phi) = \{ f(x) \in R[x] \mid f(0) = 0 \}.
    \]
    This implies that \(f(x) \in \ker(\phi)\) can be written as \(f(x) = xg(x)\) for some \(g(x) \in R[x]\), which means \(\ker(\phi) = (x)\).

    \noindent \textbf{Step 3: First Isomorphism Theorem}
    By the First Isomorphism Theorem for rings, we have:
    \[
    R[x]/(x) \simeq \text{Im}(\phi).
    \]
    Since \(\phi(f(x)) = f(0) \in R\), the image of \(\phi \) is \(R\). Therefore,
    \[
    R[x]/(x) \simeq R.
    \]

    \section*{Conclusion}
    By applying the First Isomorphism Theorem for rings, we have shown that \(R[x]/(x) \simeq R\) for any commutative ring \(R\).

\end{exercise}
\newpage
%-----------------------------
\begin{exercise}{4} Let \(R\) be any commutative ring, and let \(c \in R\) be any element. Show that there is an isomorphism \(R[x]/(x-c) \simeq R\).

    \noindent\rule{\linewidth}{1pt}

    \section*{Introduction}
    We need to show that there is an isomorphism \(R[x]/(x-c) \simeq R\) for any commutative ring \(R\) and any element \(c \in R\). We will use the First Isomorphism Theorem for rings.

    \section*{Solution}
    Consider the ring homomorphism \(\phi: R[x] \to R\) defined by \(\phi(f(x)) = f(c)\). This map evaluates the polynomial \(f(x)\) at \(x = c\).

    \noindent \textbf{Step 1: Homomorphism \(\phi \)}
    Let \(f(x) = a_0 + a_1 x + a_2 x^2 + \cdots + a_n x^n\). Then,
    \[
    \phi(f(x)) = \phi(a_0 + a_1 x + a_2 x^2 + \cdots + a_n x^n) = a_0 + a_1 c + a_2 c^2 + \cdots + a_n c^n = f(c).
    \]
    Clearly, \(\phi \) is a ring homomorphism because:
    \[
    \phi(f(x) + g(x)) = (f(x) + g(x))|_{x=c} = f(c) + g(c) = \phi(f(x)) + \phi(g(x)),
    \]
    \[
    \phi(f(x)g(x)) = (f(x)g(x))|_{x=c} = f(c)g(c) = \phi(f(x))\phi(g(x)).
    \]

    \noindent \textbf{Step 2: Kernel of \(\phi \)}
    The kernel of \(\phi \) is:
    \[
    \ker(\phi) = \{ f(x) \in R[x] \mid f(c) = 0 \}.
    \]
    This implies that \(f(x) \in \ker(\phi)\) can be written as \(f(x) = (x - c)g(x)\) for some \(g(x) \in R[x]\), which means \(\ker(\phi) = (x - c)\).

    \noindent \textbf{Step 3: First Isomorphism Theorem}
    By the First Isomorphism Theorem for rings, we have:
    \[
    R[x]/(x - c) \simeq \text{Im}(\phi).
    \]
    Since \(\phi(f(x)) = f(c) \in R\), the image of \(\phi \) is \(R\). Therefore,
    \[
    R[x]/(x - c) \simeq R.
    \]

    \section*{Conclusion}
    By applying the First Isomorphism Theorem for rings, we have shown that \(R[x]/(x - c) \simeq R\) for any commutative ring \(R\) and any element \(c \in R\).

\end{exercise}
\newpage
%-----------------------------
\begin{exercise}{6.4.9} Let \(R\) be a domain, \(X = \{(a,b) \ | \ a,b \in R, b \neq 0 \} \), and define a relation \(\sim \) on \(X\) by \((a,b) \sim (a',b')\) iff \(ab' = a'b\). Show that this relation is an equivalence relation. (Hint: Your proof should make use at once of the fact that \(R\) is a domain.) \\
    Recall that the group \(C(R)\) we defined for any commutative ring \(R \) with 1 (it is defined on PS 5, and appears on PS 9 and the optional PS).

    \noindent\rule{\linewidth}{1pt}

    \section*{Introduction}
    We need to show that the relation \(\sim \) on \(X\) defined by \((a,b) \sim (a',b')\) iff \(ab' = a'b\) is an equivalence relation. To do this, we will prove that \(\sim \) is reflexive, symmetric, and transitive.

    \section*{Solution}
    To show that \(\sim \) is an equivalence relation, we must prove that it is reflexive, symmetric, and transitive.

    \noindent \textbf{Step 1: Reflexivity}
    We need to show that \((a,b) \sim (a,b)\) for all \((a,b) \in X\). This means proving \(ab = ab\), which is trivially true. Therefore, \(\sim \) is reflexive.

    \noindent \textbf{Step 2: Symmetry}
    We need to show that if \((a,b) \sim (a',b')\), then \((a',b') \sim (a,b)\). Suppose \((a,b) \sim (a',b')\). Then \(ab' = a'b\). By commutativity of multiplication in \(R\), we have \(a'b = ab'\), which shows \((a',b') \sim (a,b)\). Therefore, \(\sim \) is symmetric.

    \noindent \textbf{Step 3: Transitivity}
    We need to show that if \((a,b) \sim (a',b')\) and \((a',b') \sim (a'',b'')\), then \((a,b) \sim (a'',b'')\). Suppose \((a,b) \sim (a',b')\) and \((a',b') \sim (a'',b'')\). This means \(ab' = a'b\) and \(a'b'' = a''b'\). We need to show that \(ab'' = a''b\).

    Starting from \(ab' = a'b\), we can multiply both sides by \(b''\) to get:
    \[
    ab'b'' = a'b b''.
    \]
    Using \(a'b'' = a''b'\) from the second equivalence, we substitute \(a'b''\) with \(a''b'\) in the equation above:
    \[
    a(b'b'') = a''b'b.
    \]
    Since \(R\) is a domain, we can cancel \(b'\) (which is nonzero) from both sides:
    \[
    ab'' = a''b.
    \]
    Therefore, \((a,b) \sim (a'',b'')\), proving that \(\sim \) is transitive.

    \section*{Conclusion}
    We have shown that the relation \(\sim \) on \(X\) defined by \((a,b) \sim (a',b')\) iff \(ab' = a'b\) is reflexive, symmetric, and transitive. Therefore, \(\sim \) is an equivalence relation.

\end{exercise}
\newpage
%-----------------------------
\begin{exercise}{7} Let \(K \) be a field such that \(2=0\). Show that every non-identity element of \((x,y) \in C(K)\) has order 2.

    \noindent\rule{\linewidth}{1pt}

    \section*{Introduction}
    We need to show that every non-identity element of \((x,y) \in C(K)\) has order 2, given that \(K\) is a field such that \(2=0\).

    \section*{Solution}
    Let \((x,y) \in C(K)\) be a non-identity element. This means \((x,y) \neq (1,0)\). The group operation in \(C(K)\) is defined as:
    \[
    (x_1, y_1) \cdot (x_2, y_2) = (x_1 x_2 - y_1 y_2, x_1 y_2 + y_1 x_2).
    \]
    We need to show that \((x,y) \cdot (x,y) = (1,0)\).

    \noindent \textbf{Step 1: Compute the Square of \((x,y)\)}
    \[
    (x,y) \cdot (x,y) = (x x - y y, x y + y x).
    \]
    Simplifying the expressions using \(2 = 0\) in \(K\), we get:
    \[
    (x x - y y, x y + y x) = (x^2 - y^2, 2xy).
    \]
    Since \(2 = 0\), we have \(2xy = 0\). Therefore, the expression simplifies to:
    \[
    (x^2 - y^2, 0).
    \]

    \noindent \textbf{Step 2: Non-identity Element Condition}
    Since \((x,y)\) is a non-identity element, we have \((x,y) \neq (1,0)\). This implies \(x \neq 1\) or \(y \neq 0\).

    \noindent \textbf{Step 3: Solve for \((x,y) \cdot (x,y) = (1,0)\)}
    We need to show that \( (x^2 - y^2, 0) = (1,0) \):
    \[
    x^2 - y^2 = 1 \quad \text{and} \quad 0 = 0.
    \]

    Therefore, the square of \((x,y)\) is \((1,0)\), which is the identity element in \(C(K)\).

    \section*{Conclusion}
    We have shown that every non-identity element of \((x,y) \in C(K)\) has order 2 if \(K\) is a field such that \(2=0\).

\end{exercise}
\newpage
%-----------------------------
\begin{exercise}{8} Let \(K\) be a field which contains an element \(i \in K\) such that \(i^2 = -1\). Show that the function \[\phi : C(K) \rightarrow K^{\times}, \qquad \phi(x,y) := x+iy\] is a homomorphism of groups.

    \noindent\rule{\linewidth}{1pt}

    \section*{Introduction}
    We need to show that the function \(\phi : C(K) \rightarrow K^{\times}\) defined by \(\phi(x,y) = x + iy\) is a homomorphism of groups, given that \(K\) is a field containing an element \(i \in K\) such that \(i^2 = -1\).

    \section*{Solution}
    Let \((x_1, y_1), (x_2, y_2) \in C(K)\). The group operation in \(C(K)\) is defined as:
    \[
    (x_1, y_1) \cdot (x_2, y_2) = (x_1 x_2 - y_1 y_2, x_1 y_2 + y_1 x_2).
    \]
    We need to show that \(\phi((x_1, y_1) \cdot (x_2, y_2)) = \phi(x_1, y_1) \phi(x_2, y_2)\).

    \noindent \textbf{Step 1: Compute \(\phi((x_1, y_1) \cdot (x_2, y_2))\)}
    \[
    \phi((x_1, y_1) \cdot (x_2, y_2)) = \phi(x_1 x_2 - y_1 y_2, x_1 y_2 + y_1 x_2).
    \]
    By the definition of \(\phi \), we have:
    \[
    \phi(x_1 x_2 - y_1 y_2, x_1 y_2 + y_1 x_2) = (x_1 x_2 - y_1 y_2) + i(x_1 y_2 + y_1 x_2).
    \]

    \noindent \textbf{Step 2: Compute \(\phi(x_1, y_1) \phi(x_2, y_2)\)}
    \[
    \phi(x_1, y_1) = x_1 + iy_1 \quad \text{and} \quad \phi(x_2, y_2) = x_2 + iy_2.
    \]
    Therefore,
    \[
    \phi(x_1, y_1) \phi(x_2, y_2) = (x_1 + iy_1)(x_2 + iy_2).
    \]
    Expanding the product using \(i^2 = -1\), we get:
    \[
    (x_1 + iy_1)(x_2 + iy_2) = x_1 x_2 + ix_1 y_2 + iy_1 x_2 + i^2 y_1 y_2 = x_1 x_2 + ix_1 y_2 + iy_1 x_2 - y_1 y_2.
    \]
    Simplifying, we get:
    \[
    x_1 x_2 - y_1 y_2 + i(x_1 y_2 + y_1 x_2).
    \]

    \noindent \textbf{Step 3: Verify the Homomorphism Property}
    Comparing the results from Steps 1 and 2, we have:
    \[
    \phi((x_1, y_1) \cdot (x_2, y_2)) = (x_1 x_2 - y_1 y_2) + i(x_1 y_2 + y_1 x_2) = \phi(x_1, y_1) \phi(x_2, y_2).
    \]
    Therefore, \(\phi \) is a homomorphism.

    \section*{Conclusion}
    We have shown that the function \(\phi : C(K) \rightarrow K^{\times}\) defined by \(\phi(x,y) = x + iy\) is a homomorphism of groups.

\end{exercise}
\newpage
%-----------------------------
\begin{exercise}{9} Let \(K\) as in the previous problem, and suppose also that \(2 \neq 0 \) in \(K\). Show that the homomorphism \(\phi \) is injective. Note: on the optional assignment, it is shown that if \(K \) is a finite field, then \(K^{\times}\) is cyclic. Thus for every finite field \(K\) as in (9), \(C(K)\) is a subgroup of a cyclic subgroup of a cyclic group and thus cyclic. This includes \(\Z_p\) when \(p \equiv 1 \pmod 4\). When \(p \equiv -1 \pmod 4\), we know that \(\Z_p\) is a subfield of \(K = \Z_p[i]\), as shown in PS 13, so \(C(\Z_p) \leq C(\Z_p[i])\) is also cyclic.

    \noindent\rule{\linewidth}{1pt}

    \section*{Introduction}
    We need to show that the homomorphism \(\phi \) defined by \(\phi(x,y) = x + iy\) is injective, given that \(K\) is a field containing an element \(i \in K\) such that \(i^2 = -1\) and \(2 \neq 0\).

    \section*{Solution}
    To show that \(\phi \) is injective, we need to prove that \(\phi(x,y) = \phi(x',y')\) implies \((x,y) = (x',y')\) for \((x,y), (x',y') \in C(K)\).

    \noindent \textbf{Step 1: Assume \(\phi(x,y) = \phi(x',y')\)}
    Suppose \(\phi(x,y) = \phi(x',y')\). This means:
    \[
    x + iy = x' + iy'.
    \]

    \noindent \textbf{Step 2: Equate Real and Imaginary Parts}
    Since \(i\) is an element in \(K\) with \(i^2 = -1\), we can equate the real and imaginary parts:
    \[
    x = x' \quad \text{and} \quad iy = iy'.
    \]
    Given that \(i \neq 0\) and \(K\) is a field (so there are no zero divisors), we can divide by \(i\):
    \[
    y = y'.
    \]

    \noindent \textbf{Step 3: Conclusion}
    Therefore, if \(\phi(x,y) = \phi(x',y')\), then \((x,y) = (x',y')\). This shows that \(\phi \) is injective.

    \section*{Conclusion}
    We have shown that the homomorphism \(\phi \) defined by \(\phi(x,y) = x + iy\) is injective, given that \(K\) is a field containing an element \(i \in K\) such that \(i^2 = -1\) and \(2 \neq 0\).

\end{exercise}
\newpage

\end{document}
