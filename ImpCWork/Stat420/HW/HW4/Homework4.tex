% Options for packages loaded elsewhere
\PassOptionsToPackage{unicode}{hyperref}
\PassOptionsToPackage{hyphens}{url}
%
\documentclass[
]{article}
\usepackage{amsmath,amssymb}
\usepackage{iftex}
\ifPDFTeX
  \usepackage[T1]{fontenc}
  \usepackage[utf8]{inputenc}
  \usepackage{textcomp} % provide euro and other symbols
\else % if luatex or xetex
  \usepackage{unicode-math} % this also loads fontspec
  \defaultfontfeatures{Scale=MatchLowercase}
  \defaultfontfeatures[\rmfamily]{Ligatures=TeX,Scale=1}
\fi
\usepackage{lmodern}
\ifPDFTeX\else
  % xetex/luatex font selection
\fi
% Use upquote if available, for straight quotes in verbatim environments
\IfFileExists{upquote.sty}{\usepackage{upquote}}{}
\IfFileExists{microtype.sty}{% use microtype if available
  \usepackage[]{microtype}
  \UseMicrotypeSet[protrusion]{basicmath} % disable protrusion for tt fonts
}{}
\makeatletter
\@ifundefined{KOMAClassName}{% if non-KOMA class
  \IfFileExists{parskip.sty}{%
    \usepackage{parskip}
  }{% else
    \setlength{\parindent}{0pt}
    \setlength{\parskip}{6pt plus 2pt minus 1pt}}
}{% if KOMA class
  \KOMAoptions{parskip=half}}
\makeatother
\usepackage{xcolor}
\usepackage[margin=1in]{geometry}
\usepackage{color}
\usepackage{fancyvrb}
\newcommand{\VerbBar}{|}
\newcommand{\VERB}{\Verb[commandchars=\\\{\}]}
\DefineVerbatimEnvironment{Highlighting}{Verbatim}{commandchars=\\\{\}}
% Add ',fontsize=\small' for more characters per line
\usepackage{framed}
\definecolor{shadecolor}{RGB}{248,248,248}
\newenvironment{Shaded}{\begin{snugshade}}{\end{snugshade}}
\newcommand{\AlertTok}[1]{\textcolor[rgb]{0.94,0.16,0.16}{#1}}
\newcommand{\AnnotationTok}[1]{\textcolor[rgb]{0.56,0.35,0.01}{\textbf{\textit{#1}}}}
\newcommand{\AttributeTok}[1]{\textcolor[rgb]{0.13,0.29,0.53}{#1}}
\newcommand{\BaseNTok}[1]{\textcolor[rgb]{0.00,0.00,0.81}{#1}}
\newcommand{\BuiltInTok}[1]{#1}
\newcommand{\CharTok}[1]{\textcolor[rgb]{0.31,0.60,0.02}{#1}}
\newcommand{\CommentTok}[1]{\textcolor[rgb]{0.56,0.35,0.01}{\textit{#1}}}
\newcommand{\CommentVarTok}[1]{\textcolor[rgb]{0.56,0.35,0.01}{\textbf{\textit{#1}}}}
\newcommand{\ConstantTok}[1]{\textcolor[rgb]{0.56,0.35,0.01}{#1}}
\newcommand{\ControlFlowTok}[1]{\textcolor[rgb]{0.13,0.29,0.53}{\textbf{#1}}}
\newcommand{\DataTypeTok}[1]{\textcolor[rgb]{0.13,0.29,0.53}{#1}}
\newcommand{\DecValTok}[1]{\textcolor[rgb]{0.00,0.00,0.81}{#1}}
\newcommand{\DocumentationTok}[1]{\textcolor[rgb]{0.56,0.35,0.01}{\textbf{\textit{#1}}}}
\newcommand{\ErrorTok}[1]{\textcolor[rgb]{0.64,0.00,0.00}{\textbf{#1}}}
\newcommand{\ExtensionTok}[1]{#1}
\newcommand{\FloatTok}[1]{\textcolor[rgb]{0.00,0.00,0.81}{#1}}
\newcommand{\FunctionTok}[1]{\textcolor[rgb]{0.13,0.29,0.53}{\textbf{#1}}}
\newcommand{\ImportTok}[1]{#1}
\newcommand{\InformationTok}[1]{\textcolor[rgb]{0.56,0.35,0.01}{\textbf{\textit{#1}}}}
\newcommand{\KeywordTok}[1]{\textcolor[rgb]{0.13,0.29,0.53}{\textbf{#1}}}
\newcommand{\NormalTok}[1]{#1}
\newcommand{\OperatorTok}[1]{\textcolor[rgb]{0.81,0.36,0.00}{\textbf{#1}}}
\newcommand{\OtherTok}[1]{\textcolor[rgb]{0.56,0.35,0.01}{#1}}
\newcommand{\PreprocessorTok}[1]{\textcolor[rgb]{0.56,0.35,0.01}{\textit{#1}}}
\newcommand{\RegionMarkerTok}[1]{#1}
\newcommand{\SpecialCharTok}[1]{\textcolor[rgb]{0.81,0.36,0.00}{\textbf{#1}}}
\newcommand{\SpecialStringTok}[1]{\textcolor[rgb]{0.31,0.60,0.02}{#1}}
\newcommand{\StringTok}[1]{\textcolor[rgb]{0.31,0.60,0.02}{#1}}
\newcommand{\VariableTok}[1]{\textcolor[rgb]{0.00,0.00,0.00}{#1}}
\newcommand{\VerbatimStringTok}[1]{\textcolor[rgb]{0.31,0.60,0.02}{#1}}
\newcommand{\WarningTok}[1]{\textcolor[rgb]{0.56,0.35,0.01}{\textbf{\textit{#1}}}}
\usepackage{graphicx}
\makeatletter
\def\maxwidth{\ifdim\Gin@nat@width>\linewidth\linewidth\else\Gin@nat@width\fi}
\def\maxheight{\ifdim\Gin@nat@height>\textheight\textheight\else\Gin@nat@height\fi}
\makeatother
% Scale images if necessary, so that they will not overflow the page
% margins by default, and it is still possible to overwrite the defaults
% using explicit options in \includegraphics[width, height, ...]{}
\setkeys{Gin}{width=\maxwidth,height=\maxheight,keepaspectratio}
% Set default figure placement to htbp
\makeatletter
\def\fps@figure{htbp}
\makeatother
\setlength{\emergencystretch}{3em} % prevent overfull lines
\providecommand{\tightlist}{%
  \setlength{\itemsep}{0pt}\setlength{\parskip}{0pt}}
\setcounter{secnumdepth}{-\maxdimen} % remove section numbering
\ifLuaTeX
  \usepackage{selnolig}  % disable illegal ligatures
\fi
\IfFileExists{bookmark.sty}{\usepackage{bookmark}}{\usepackage{hyperref}}
\IfFileExists{xurl.sty}{\usepackage{xurl}}{} % add URL line breaks if available
\urlstyle{same}
\hypersetup{
  pdftitle={Homework 4},
  pdfauthor={Charles Ancel},
  hidelinks,
  pdfcreator={LaTeX via pandoc}}

\title{Homework 4}
\author{Charles Ancel}
\date{9/21/2023}

\begin{document}
\maketitle

\section{Homework Instructions}\label{homework-instructions}

\textbf{Make sure to add your name to the header of the document. When
submitting the assignment on Gradescope, be sure to assign the
appropriate pages of your submission to each Exercise.}

For questions that require code, please create or use the code chunk
directly below the question and type your code there. Your knitted pdf
will then show both the code and the output, so that we can assess your
understanding and award any partial credit.

For written questions, please provide your answer after the indicated
\emph{Answer} prompt.

You are encouraged to knit your file as you work, to check that your
coding and formatting are done so appropriately. This will also help you
identify and locate any errors more easily.

\section{Homework Setup}\label{homework-setup}

We'll use the following packages for this homework assignment.

\begin{Shaded}
\begin{Highlighting}[]
\FunctionTok{library}\NormalTok{(ggplot2)}
\end{Highlighting}
\end{Shaded}

\section{Exercise 1: Formatting {[}5
points{]}}\label{exercise-1-formatting-5-points}

The first five points of the assignment will be earned for properly
formatting your final document. Check that you have:

\begin{itemize}
\tightlist
\item
  included your name on the document
\item
  properly assigned pages to exercises on Gradescope
\item
  select \textbf{page 1 (with your name)} and this page for this
  exercise (Exercise 1)
\item
  all code is printed and readable for each question
\item
  generated a pdf file
\end{itemize}

\begin{center}\rule{0.5\linewidth}{0.5pt}\end{center}

\section{Exercise 2: Mammalian Sleep Model {[}23
points{]}}\label{exercise-2-mammalian-sleep-model-23-points}

We'll use the \texttt{msleep} dataset from the \texttt{ggplot2} package
for this assignment. At first glance of the \texttt{msleep} data, you
may notice some missing values encoded as NAs. For this question, we
will use the sleep (\texttt{sleep\_total}) and bodyweight of an animal,
which have no missing values.

\begin{Shaded}
\begin{Highlighting}[]
\NormalTok{?msleep}
\end{Highlighting}
\end{Shaded}

\subsection{part a}\label{part-a}

I wonder about how the amount of sleep required by an animal changes
based on the bodyweight of an animal. For example, do animals who weigh
more require more sleep. What would be the primary purpose of fitting a
model like this? \textbf{Bold your answer} below by surrounding your
selection with two asterisks ``**``.

\begin{enumerate}
\def\labelenumi{(\alph{enumi})}
\tightlist
\item
  predicting an observation \textbf{(b) explaining a structure/system}
\end{enumerate}

\subsection{part b}\label{part-b}

Fit a linear model to estimate the sleep required based on the
bodyweight. Print the summary of this linear model. Then, write the
fitted model based on this output. Be sure to use proper notation.

\begin{Shaded}
\begin{Highlighting}[]
\NormalTok{sleep\_model }\OtherTok{\textless{}{-}} \FunctionTok{lm}\NormalTok{(sleep\_total }\SpecialCharTok{\textasciitilde{}}\NormalTok{ bodywt, }\AttributeTok{data=}\NormalTok{msleep)}

\FunctionTok{summary}\NormalTok{(sleep\_model)}
\end{Highlighting}
\end{Shaded}

\begin{verbatim}
## 
## Call:
## lm(formula = sleep_total ~ bodywt, data = msleep)
## 
## Residuals:
##     Min      1Q  Median      3Q     Max 
## -7.7008 -2.3787 -0.4268  3.2732  9.1731 
## 
## Coefficients:
##               Estimate Std. Error t value Pr(>|t|)    
## (Intercept) 10.7269205  0.4773797  22.470  < 2e-16 ***
## bodywt      -0.0017647  0.0005971  -2.956  0.00409 ** 
## ---
## Signif. codes:  0 '***' 0.001 '**' 0.01 '*' 0.05 '.' 0.1 ' ' 1
## 
## Residual standard error: 4.254 on 81 degrees of freedom
## Multiple R-squared:  0.09735,    Adjusted R-squared:  0.08621 
## F-statistic: 8.736 on 1 and 81 DF,  p-value: 0.004085
\end{verbatim}

\textbf{Answer:} - The fitted linear model is: \[
  \text{{sleep\_total}} = 10.7269 - 0.0017647 \times \text{{bodywt}}
  \]

\begin{itemize}
\item
  \(\beta_0\) (Intercept): 10.7269205
\item
  \(\beta_1\) (Coefficient of bodywt): -0.0017647
\item
  The t-value for the intercept is 22.470 and is highly significant
  (p-value \textless{} 2e-16).
\item
  The t-value for the body weight is -2.956 and is also significant
  (p-value = 0.00409).
\item
  The coefficient of determination (\(R^2\)) is 0.09735. This means that
  approximately 9.735\% of the variability in total sleep can be
  explained by the body weight of the animals.
\end{itemize}

\subsection{part c}\label{part-c}

Interpret the intercept for this model.

\textbf{Answer:}

Intercept (\(\beta_0\)) is 10.7269. It means for an animal with a body
weight of 0, the predicted total sleep is about 10.7269 hours. However,
an animal with a weight of 0 isn't realistic.

\subsection{part d}\label{part-d}

Is the intercept for this model meaningful? Is it reliable? Explain.

\textbf{Answer:} Given the intercept is 10.7269, it represents the
predicted sleep for an animal with a body weight of 0.

\textbf{Meaningfulness}: The intercept isn't meaningful in a biological
context, as animals can't have a weight of 0.

\textbf{Reliability}: While statistically significant, the intercept's
real-world applicability is limited due to the aforementioned reason.

\subsection{part e}\label{part-e}

Interpret the slope for this model.

\textbf{Answer:}

\textbf{part e}: Interpret the slope for the model.

The slope (coefficient of \texttt{bodywt}) is \(-0.0017647\).

\textbf{Interpretation}: For every 1 unit increase in body weight, the
predicted total sleep decreases by approximately 0.0017647 hours.

\subsection{part f}\label{part-f}

Report and interpret the coefficient of determination for this
relationship.

\begin{Shaded}
\begin{Highlighting}[]
\NormalTok{r\_squared }\OtherTok{\textless{}{-}} \FunctionTok{summary}\NormalTok{(sleep\_model)}\SpecialCharTok{$}\NormalTok{r.squared}
\NormalTok{r\_squared}
\end{Highlighting}
\end{Shaded}

\begin{verbatim}
## [1] 0.09735061
\end{verbatim}

\textbf{Answer:} Multiple \(R^2\) is 0.09735.

\textbf{Interpretation}: Approximately 9.735\% of the variability in
total sleep can be explained by the body weight of the animals.

\subsection{part g}\label{part-g}

Based on part f, calculate the correlation.

\begin{Shaded}
\begin{Highlighting}[]
\NormalTok{r }\OtherTok{\textless{}{-}} \FunctionTok{sqrt}\NormalTok{(r\_squared)}

\ControlFlowTok{if}\NormalTok{ (}\FunctionTok{coef}\NormalTok{(sleep\_model)[}\StringTok{"bodywt"}\NormalTok{] }\SpecialCharTok{\textless{}} \DecValTok{0}\NormalTok{) \{}
\NormalTok{  r }\OtherTok{\textless{}{-}} \SpecialCharTok{{-}}\NormalTok{r}
\NormalTok{\}}
\NormalTok{r}
\end{Highlighting}
\end{Shaded}

\begin{verbatim}
## [1] -0.3120106
\end{verbatim}

\textbf{Answer:} The value \(-0.3120106\) indicates a weak negative
linear relationship between \texttt{bodywt} and \texttt{sleep\_total}.
This means that as body weight increases, the total sleep tends to
decrease slightly, consistent with the negative slope we observed in the
regression model.

\begin{center}\rule{0.5\linewidth}{0.5pt}\end{center}

\section{Exercise 3: Hand Calculations {[}30
points{]}}\label{exercise-3-hand-calculations-30-points}

We've used \texttt{R} to generate the summary statistics for the
\texttt{msleep} dataset so far. Now let's take a moment and confirm some
of these calculations ``by hand'' (still using \texttt{R} to perform the
calculations).

\subsection{part a}\label{part-a-1}

Calculate \(\bar{x}\), \(\bar{y}\), \(S_{xx}\), and \(S_{xy}\) for the
msleep dataset. Clearly label and print your results.

\begin{Shaded}
\begin{Highlighting}[]
\NormalTok{x\_bar }\OtherTok{\textless{}{-}} \FunctionTok{mean}\NormalTok{(msleep}\SpecialCharTok{$}\NormalTok{bodywt, }\AttributeTok{na.rm =} \ConstantTok{TRUE}\NormalTok{)}
\NormalTok{y\_bar }\OtherTok{\textless{}{-}} \FunctionTok{mean}\NormalTok{(msleep}\SpecialCharTok{$}\NormalTok{sleep\_total, }\AttributeTok{na.rm =} \ConstantTok{TRUE}\NormalTok{)}

\NormalTok{S\_xx }\OtherTok{\textless{}{-}} \FunctionTok{sum}\NormalTok{((msleep}\SpecialCharTok{$}\NormalTok{bodywt }\SpecialCharTok{{-}}\NormalTok{ x\_bar)}\SpecialCharTok{\^{}}\DecValTok{2}\NormalTok{, }\AttributeTok{na.rm =} \ConstantTok{TRUE}\NormalTok{)}
\NormalTok{S\_xy }\OtherTok{\textless{}{-}} \FunctionTok{sum}\NormalTok{((msleep}\SpecialCharTok{$}\NormalTok{bodywt }\SpecialCharTok{{-}}\NormalTok{ x\_bar) }\SpecialCharTok{*}\NormalTok{ (msleep}\SpecialCharTok{$}\NormalTok{sleep\_total }\SpecialCharTok{{-}}\NormalTok{ y\_bar), }\AttributeTok{na.rm =} \ConstantTok{TRUE}\NormalTok{)}

\NormalTok{x\_bar}
\end{Highlighting}
\end{Shaded}

\begin{verbatim}
## [1] 166.1363
\end{verbatim}

\begin{Shaded}
\begin{Highlighting}[]
\NormalTok{y\_bar}
\end{Highlighting}
\end{Shaded}

\begin{verbatim}
## [1] 10.43373
\end{verbatim}

\begin{Shaded}
\begin{Highlighting}[]
\NormalTok{S\_xx}
\end{Highlighting}
\end{Shaded}

\begin{verbatim}
## [1] 50767575
\end{verbatim}

\begin{Shaded}
\begin{Highlighting}[]
\NormalTok{S\_xy}
\end{Highlighting}
\end{Shaded}

\begin{verbatim}
## [1] -89590.99
\end{verbatim}

\subsection{part b}\label{part-b-1}

Calculate the estimates for \(\beta_0\) and \(\beta_1\), using the
values calculated in part a. Clearly label and print your results.
Compare these estimates to what you found in Exercise 2b.

\begin{Shaded}
\begin{Highlighting}[]
\NormalTok{beta\_1 }\OtherTok{\textless{}{-}}\NormalTok{ S\_xy }\SpecialCharTok{/}\NormalTok{ S\_xx}
\NormalTok{beta\_0 }\OtherTok{\textless{}{-}}\NormalTok{ y\_bar }\SpecialCharTok{{-}}\NormalTok{ beta\_1 }\SpecialCharTok{*}\NormalTok{ x\_bar}

\NormalTok{beta\_0}
\end{Highlighting}
\end{Shaded}

\begin{verbatim}
## [1] 10.72692
\end{verbatim}

\begin{Shaded}
\begin{Highlighting}[]
\NormalTok{beta\_1}
\end{Highlighting}
\end{Shaded}

\begin{verbatim}
## [1] -0.001764729
\end{verbatim}

\textbf{Comparison:}

From the manual calculations (part b of Exercise 3): 1. \(\beta_0\)
(Intercept): 10.72692 2. \(\beta_1\) (Coefficient of \texttt{bodywt}):
-0.001764729

From the linear model in Exercise 2b: 1. \(\beta_0\) (Intercept):
10.7269205 2. \(\beta_1\) (Coefficient of \texttt{bodywt}): -0.0017647

Comparison: The values of \(\beta_0\) and \(\beta_1\) from the manual
calculations match very closely with the values from the linear
regression model in Exercise 2b. This is expected, as both methods aim
to estimate the same linear relationship between \texttt{bodywt} and
\texttt{sleep\_total}.

In summary, the manually calculated estimates for the intercept and
slope are consistent with the results obtained from the linear
regression model.

\subsection{part c}\label{part-c-1}

Calculate the residuals for the dataset. You can use any built-in
\texttt{R} function to generate the fitted values of the dataset, but
you should not use a built-in function to calculate the residuals. No
need to print all of the residuals, but please do print the first few
residuals.

\begin{Shaded}
\begin{Highlighting}[]
\NormalTok{y\_hat }\OtherTok{\textless{}{-}}\NormalTok{ beta\_0 }\SpecialCharTok{+}\NormalTok{ beta\_1 }\SpecialCharTok{*}\NormalTok{ msleep}\SpecialCharTok{$}\NormalTok{bodywt}

\NormalTok{residuals }\OtherTok{\textless{}{-}}\NormalTok{ msleep}\SpecialCharTok{$}\NormalTok{sleep\_total }\SpecialCharTok{{-}}\NormalTok{ y\_hat}

\FunctionTok{head}\NormalTok{(residuals)}
\end{Highlighting}
\end{Shaded}

\begin{verbatim}
## [1]  1.461316  6.273927  3.675462  4.173113 -5.668083  3.679874
\end{verbatim}

\subsection{part d}\label{part-d-1}

Calculate the SSR, SSE, and SST for the model. You may use anything that
you have calculated in the earlier parts of this question. Clearly label
and print these three values.

\begin{Shaded}
\begin{Highlighting}[]
\NormalTok{SSR }\OtherTok{\textless{}{-}} \FunctionTok{sum}\NormalTok{((y\_hat }\SpecialCharTok{{-}}\NormalTok{ y\_bar)}\SpecialCharTok{\^{}}\DecValTok{2}\NormalTok{)}
\NormalTok{SSE }\OtherTok{\textless{}{-}} \FunctionTok{sum}\NormalTok{(residuals}\SpecialCharTok{\^{}}\DecValTok{2}\NormalTok{)}
\NormalTok{SST }\OtherTok{\textless{}{-}} \FunctionTok{sum}\NormalTok{((msleep}\SpecialCharTok{$}\NormalTok{sleep\_total }\SpecialCharTok{{-}}\NormalTok{ y\_bar)}\SpecialCharTok{\^{}}\DecValTok{2}\NormalTok{)}

\NormalTok{SSR}
\end{Highlighting}
\end{Shaded}

\begin{verbatim}
## [1] 158.1038
\end{verbatim}

\begin{Shaded}
\begin{Highlighting}[]
\NormalTok{SSE}
\end{Highlighting}
\end{Shaded}

\begin{verbatim}
## [1] 1465.962
\end{verbatim}

\begin{Shaded}
\begin{Highlighting}[]
\NormalTok{SST}
\end{Highlighting}
\end{Shaded}

\begin{verbatim}
## [1] 1624.066
\end{verbatim}

\subsection{part e}\label{part-e-1}

Given the formulas for these quantities: - \(SST\) represents the total
variability in the response variable (\texttt{sleep\_total}). - \(SSR\)
represents the variability in the response variable that is explained by
the predictor (\texttt{bodywt}). - \(SSE\) represents the variability in
the response that is not explained by the predictor; it's the error or
the unexplained variability.

The relationship between them is: \[ SST = SSR + SSE \]

This is evident from the results provided: 1624.066 (SST) ≈ 158.1038
(SSR) + 1465.962 (SSE)

The relationship indicates that the total variability in the response
can be partitioned into the variability that's explained by the model
(SSR) and the variability that's unexplained (SSE). \#\# part f

Calculate the coefficient of determination from the values calculated in
part d.~How does this compare to what you found in Exercise 2d (based on
Exercise 2b)?

\begin{Shaded}
\begin{Highlighting}[]
\NormalTok{R\_squared\_calculated }\OtherTok{\textless{}{-}}\NormalTok{ SSR }\SpecialCharTok{/}\NormalTok{ SST}

\NormalTok{R\_squared\_calculated}
\end{Highlighting}
\end{Shaded}

\begin{verbatim}
## [1] 0.09735061
\end{verbatim}

\textbf{Comparison:}

The calculated \(R^2\) from \textbf{part f} is \(0.09735061\).

From Exercise 2f, the \(R^2\) value was \(0.09735\).

The two values match, which is expected since both methods aim to
estimate the proportion of the total variation in the response variable
(\texttt{sleep\_total}) that is explained by the predictor
(\texttt{bodywt}).

\begin{center}\rule{0.5\linewidth}{0.5pt}\end{center}

\section{Exercise 4: Inference for the Mammal Sleep Model {[}21
points{]}}\label{exercise-4-inference-for-the-mammal-sleep-model-21-points}

\subsection{part a}\label{part-a-2}

For the Chinchilla (bodyweight of 0.420), calculate the predicted total
amount of sleep and the corresponding residual.

\begin{Shaded}
\begin{Highlighting}[]
\NormalTok{chinchilla\_bodywt }\OtherTok{\textless{}{-}} \FloatTok{0.420}
\NormalTok{predicted\_sleep }\OtherTok{\textless{}{-}}\NormalTok{ beta\_0 }\SpecialCharTok{+}\NormalTok{ beta\_1 }\SpecialCharTok{*}\NormalTok{ chinchilla\_bodywt}

\NormalTok{predicted\_sleep}
\end{Highlighting}
\end{Shaded}

\begin{verbatim}
## [1] 10.72618
\end{verbatim}

\begin{Shaded}
\begin{Highlighting}[]
\NormalTok{chinchilla\_observed\_sleep }\OtherTok{\textless{}{-}}\NormalTok{ msleep}\SpecialCharTok{$}\NormalTok{sleep\_total[msleep}\SpecialCharTok{$}\NormalTok{name }\SpecialCharTok{==} \StringTok{"Chinchilla"}\NormalTok{]}

\NormalTok{chinchilla\_residual }\OtherTok{\textless{}{-}}\NormalTok{ chinchilla\_observed\_sleep }\SpecialCharTok{{-}}\NormalTok{ predicted\_sleep}

\NormalTok{chinchilla\_residual}
\end{Highlighting}
\end{Shaded}

\begin{verbatim}
## [1] 1.773821
\end{verbatim}

\textbf{Answer:}

\subsection{part b}\label{part-b-2}

Calculate the 80\% confidence interval for the slope of this model.
\emph{Note: the multiplier (\(t^{*}\)) is 1.292 for this situation.}
Make sure to show your setup for the calculation. Report the 80\%
confidence interval below. Based on this confidence interval, is -0.002
a plausible value for the slope predicting sleep from body weight for
all mammals?

\begin{Shaded}
\begin{Highlighting}[]
\NormalTok{std\_error\_slope }\OtherTok{\textless{}{-}} \FunctionTok{summary}\NormalTok{(sleep\_model)}\SpecialCharTok{$}\NormalTok{coefficients[}\StringTok{"bodywt"}\NormalTok{, }\StringTok{"Std. Error"}\NormalTok{]}

\NormalTok{alpha }\OtherTok{\textless{}{-}} \FloatTok{0.2}
\NormalTok{df }\OtherTok{\textless{}{-}} \FunctionTok{nrow}\NormalTok{(msleep) }\SpecialCharTok{{-}} \DecValTok{2}
\NormalTok{t\_critical }\OtherTok{\textless{}{-}} \FunctionTok{qt}\NormalTok{(}\DecValTok{1} \SpecialCharTok{{-}}\NormalTok{ alpha}\SpecialCharTok{/}\DecValTok{2}\NormalTok{, df)}

\NormalTok{margin\_error }\OtherTok{\textless{}{-}}\NormalTok{ t\_critical }\SpecialCharTok{*}\NormalTok{ std\_error\_slope}

\NormalTok{lower\_bound }\OtherTok{\textless{}{-}}\NormalTok{ beta\_1 }\SpecialCharTok{{-}}\NormalTok{ margin\_error}
\NormalTok{upper\_bound }\OtherTok{\textless{}{-}}\NormalTok{ beta\_1 }\SpecialCharTok{+}\NormalTok{ margin\_error}

\FunctionTok{c}\NormalTok{(lower\_bound, upper\_bound)}
\end{Highlighting}
\end{Shaded}

\begin{verbatim}
## [1] -0.0025361978 -0.0009932593
\end{verbatim}

\textbf{Answer:} The interval is:
\[ -0.0025361978 \leq \beta_1 \leq -0.0009932593 \]

Given this interval, the value -0.002 falls within the range. This means
that -0.002 is a plausible value for the slope when predicting sleep
from body weight for all mammals, based on the 80\% confidence interval.

\subsection{part c}\label{part-c-2}

Suppose that a previous study had found that for each increase in the
body weight by 1 kg, the estimated average sleep time decreased by
-0.01, on average.

Might this same relationship be true for all mammals, including those in
our data? Or is there evidence that this relationship for mammals is
different from the one for reptiles?

In other words, use a t-test to test:

\begin{itemize}
\tightlist
\item
  \(H_0: \beta_1 = -0.01\)
\item
  \(H_1: \beta_1 \neq -0.01\)
\end{itemize}

Calculate and report the value of your test statistic. Then, based on
the size of your test statistic, anticipate what the decision for the
statistical test would be with a significance level of
\(\alpha = 0.05\).

\begin{Shaded}
\begin{Highlighting}[]
\NormalTok{t\_statistic }\OtherTok{\textless{}{-}}\NormalTok{ (beta\_1 }\SpecialCharTok{{-}}\NormalTok{ (}\SpecialCharTok{{-}}\FloatTok{0.01}\NormalTok{)) }\SpecialCharTok{/}\NormalTok{ std\_error\_slope}

\NormalTok{p\_value }\OtherTok{\textless{}{-}} \DecValTok{2} \SpecialCharTok{*}\NormalTok{ (}\DecValTok{1} \SpecialCharTok{{-}} \FunctionTok{pt}\NormalTok{(}\FunctionTok{abs}\NormalTok{(t\_statistic), df))}

\FunctionTok{c}\NormalTok{(t\_statistic, p\_value)}
\end{Highlighting}
\end{Shaded}

\begin{verbatim}
## [1] 13.7928  0.0000
\end{verbatim}

\textbf{Answer:} Here's the interpretation:

\begin{enumerate}
\def\labelenumi{\arabic{enumi}.}
\tightlist
\item
  \(t\)-statistic: 13.7928
\item
  \(p\)-value: 0.0000 (rounded, so it's very close to 0)
\end{enumerate}

The results indicate that the relationship between body weight and sleep
duration for mammals is statistically different from the hypothesized
value for reptiles. The p-value is very close to 0, so we reject the
null hypothesis.

\subsection{part d}\label{part-d-2}

If a news article used this data to make the claim ``Gaining weight will
make you sleep less!'' would you agree with that conclusion? Why or why
not?

\emph{There is not an objectively right answer for this question.}
Consider all of the information that we've compiled and calculations
that we've performed to help guide your response.

\textbf{Answer:} The data indicates a negative association between body
weight and sleep duration for mammals. However, the relationship is
weak, and correlation doesn't imply causation. While weight might
influence sleep to some extent, it's not the sole factor. Claiming
``Gaining weight will make you sleep less!'' oversimplifies the findings
and doesn't account for other potential influencing factors.

\subsection{part e}\label{part-e-2}

For the default hypothesis test for the intercept, report the following:

\begin{itemize}
\tightlist
\item
  The null and alternative hypotheses (in symbols)
\item
  The value of the test statistic
\item
  The \emph{p}-value of the test
\end{itemize}

Be sure to report this information in text below. No need to calculate
any values directly for this part; observing and identifying appropriate
information from R output is sufficient.

\textbf{Answer:} For the default hypothesis test for the intercept, the
information you need can be extracted from the output of the
\texttt{lm()} function, specifically from the summary of the linear
model.

\begin{enumerate}
\def\labelenumi{\arabic{enumi}.}
\item
  \textbf{The null and alternative hypotheses (in symbols) for the
  intercept}:

  \begin{itemize}
  \tightlist
  \item
    \(H_0: \beta_0 = 0\)
  \item
    \(H_1: \beta_0 \neq 0\)
  \end{itemize}

  This means that the null hypothesis assumes there's no effect
  (intercept is zero), while the alternative hypothesis posits that
  there is an effect (intercept is different from zero).
\item
  \textbf{The value of the test statistic}:

  \begin{itemize}
  \tightlist
  \item
    You'll find this value in the summary output of the linear model
    under the ``Coefficients'' section, specifically in the row for
    ``(Intercept)'' and the column ``t value.''
  \end{itemize}
\item
  \textbf{The p-value of the test}:

  \begin{itemize}
  \tightlist
  \item
    Similarly, this value can be found in the summary output of the
    linear model under the ``Coefficients'' section, in the row for
    ``(Intercept)'' and the column
    ``Pr(\textgreater\textbar t\textbar).''
  \end{itemize}
\end{enumerate}

From the output you provided for Exercise 2b:

\begin{itemize}
\tightlist
\item
  Test statistic value for the intercept: 22.470
\item
  \(p\)-value for the intercept: \textless{} 2e-16 (which is essentially
  0)
\end{itemize}

Now looking at our hypothesis and responding to the question: - Null
Hypothesis \(H_0\): \(\beta_0 = 0\) - Alternative Hypothesis \(H_1\):
\(\beta_0 \neq 0\) - Test statistic value for the intercept: 22.470 -
\(p\)-value for the intercept: \textless{} 2e-16 (essentially 0)

Given the extremely low p-value, there's strong evidence against the
null hypothesis, suggesting the intercept is different from zero.

\begin{center}\rule{0.5\linewidth}{0.5pt}\end{center}

\section{Exercise 5: Assumptions for the Mammal Sleep Model {[}21
points{]}}\label{exercise-5-assumptions-for-the-mammal-sleep-model-21-points}

\subsection{part a}\label{part-a-3}

For the fitted model to be appropriate and for the inference procedures
from Exercise 4 to be valid, certain assumptions must be met. First,
write out the four assumptions.

\textbf{Answer:} For the fitted linear regression model to be
appropriate and for the inference procedures to be valid, the four
primary assumptions are:

\begin{enumerate}
\def\labelenumi{\arabic{enumi}.}
\item
  \textbf{Linearity}: The relationship between the independent
  variable(s) and the dependent variable is linear.
\item
  \textbf{Independence}: The residuals are independent. In other words,
  the residuals from one prediction have no effect on the residuals from
  another prediction.
\item
  \textbf{Homoscedasticity (Constant Variance)}: The residuals have
  constant variance at every level of the independent variable(s).
\item
  \textbf{Normality}: The residuals are approximately normally
  distributed.
\item
  Linearity: The relationship between the predictor(s) and the response
  is linear.
\item
  Independence: Observations are independent of each other.
\item
  Homoscedasticity: The variance of the residuals is constant across all
  levels of the independent variable(s).
\item
  Normality: The residuals are approximately normally distributed.
\end{enumerate}

\subsection{part b}\label{part-b-3}

Which of these four assumptions cannot be checked with a plot?

\textbf{Answer:} Out of the four assumptions, the \textbf{Independence}
assumption is the one that cannot be directly checked with a plot. While
scatter plots and residual plots can provide insights into linearity,
homoscedasticity, and normality (e.g., through Q-Q plots), the
independence of observations is often based on the study design and
domain knowledge.

The assumption of Independence cannot be directly checked with a plot.

\subsection{part c}\label{part-c-3}

Create the two plots (it's ok if you end up with 4 plots) that we can
use to check our assumptions for the mammal sleep data. Interpret these
two plots. Be sure to specify if the assumptions seem reasonable and
describe what you see that supports your conclusions.

\begin{Shaded}
\begin{Highlighting}[]
\FunctionTok{par}\NormalTok{(}\AttributeTok{mfrow=}\FunctionTok{c}\NormalTok{(}\DecValTok{2}\NormalTok{,}\DecValTok{2}\NormalTok{))}

\FunctionTok{plot}\NormalTok{(sleep\_model}\SpecialCharTok{$}\NormalTok{fitted.values, }\FunctionTok{residuals}\NormalTok{(sleep\_model), }
     \AttributeTok{main=}\StringTok{"Residuals vs. Fitted"}\NormalTok{, }\AttributeTok{xlab=}\StringTok{"Fitted values"}\NormalTok{, }\AttributeTok{ylab=}\StringTok{"Residuals"}\NormalTok{)}
\FunctionTok{abline}\NormalTok{(}\AttributeTok{h=}\DecValTok{0}\NormalTok{, }\AttributeTok{col=}\StringTok{"red"}\NormalTok{)}

\FunctionTok{qqnorm}\NormalTok{(}\FunctionTok{residuals}\NormalTok{(sleep\_model), }\AttributeTok{main=}\StringTok{"Normal Q{-}Q Plot"}\NormalTok{)}
\FunctionTok{qqline}\NormalTok{(}\FunctionTok{residuals}\NormalTok{(sleep\_model))}
\end{Highlighting}
\end{Shaded}

\includegraphics{Homework4_files/figure-latex/exercise5c-1.pdf}

\textbf{Answer:}

\begin{enumerate}
\def\labelenumi{\arabic{enumi}.}
\tightlist
\item
  \textbf{Residuals vs.~Fitted Values}:

  \begin{itemize}
  \tightlist
  \item
    The residuals are scattered without a clear pattern around zero,
    suggesting a linear relationship between predictors and response.
  \item
    The consistent spread across the fitted values indicates
    homoscedasticity, meaning the variance remains constant.
  \end{itemize}
\item
  \textbf{Normal Q-Q Plot}:

  \begin{itemize}
  \tightlist
  \item
    Most data points align with the diagonal, implying residuals are
    approximately normally distributed.
  \item
    Some deviations at the tails hint at slight non-normality, but these
    are minor.
  \end{itemize}
\end{enumerate}

In summary, the plots suggest the linear regression assumptions are
predominantly satisfied for the mammal sleep data, with minor concerns
about perfect normality.

\subsection{part d}\label{part-d-3}

There are two additional statistical tests that can be used to help
assess our assumptions for the linear model. Perform these two tests
here. Then, based just on these two tests, assess the corresponding
assumptions.

\begin{Shaded}
\begin{Highlighting}[]
\FunctionTok{library}\NormalTok{(lmtest)}
\end{Highlighting}
\end{Shaded}

\begin{verbatim}
## Loading required package: zoo
\end{verbatim}

\begin{verbatim}
## 
## Attaching package: 'zoo'
\end{verbatim}

\begin{verbatim}
## The following objects are masked from 'package:base':
## 
##     as.Date, as.Date.numeric
\end{verbatim}

\begin{Shaded}
\begin{Highlighting}[]
\FunctionTok{library}\NormalTok{(stats)}

\NormalTok{bp\_test }\OtherTok{\textless{}{-}} \FunctionTok{bptest}\NormalTok{(sleep\_model)}

\NormalTok{shapiro\_test }\OtherTok{\textless{}{-}} \FunctionTok{shapiro.test}\NormalTok{(}\FunctionTok{residuals}\NormalTok{(sleep\_model))}

\FunctionTok{list}\NormalTok{(}\AttributeTok{Breusch\_Pagan\_Test =}\NormalTok{ bp\_test, }\AttributeTok{Shapiro\_Wilk\_Test =}\NormalTok{ shapiro\_test)}
\end{Highlighting}
\end{Shaded}

\begin{verbatim}
## $Breusch_Pagan_Test
## 
##  studentized Breusch-Pagan test
## 
## data:  sleep_model
## BP = 0.20069, df = 1, p-value = 0.6542
## 
## 
## $Shapiro_Wilk_Test
## 
##  Shapiro-Wilk normality test
## 
## data:  residuals(sleep_model)
## W = 0.97894, p-value = 0.1908
\end{verbatim}

\textbf{Answer:}

\begin{enumerate}
\def\labelenumi{\arabic{enumi}.}
\item
  \textbf{Breusch-Pagan Test} (Homoscedasticity):

  \begin{itemize}
  \tightlist
  \item
    Test Statistic (BP): \(0.20069\)
  \item
    \(p\)-value: \(0.6542\)
  \end{itemize}

  The Breusch-Pagan test suggests that the residuals have constant
  variance across levels of the independent variable, as the \(p\)-value
  of \(0.6542\) is greater than \(0.05\). This means the assumption of
  homoscedasticity is met.
\item
  \textbf{Shapiro-Wilk Test} (Normality of Residuals):

  \begin{itemize}
  \tightlist
  \item
    Test Statistic (W): \(0.97894\)
  \item
    \(p\)-value: \(0.1908\)
  \end{itemize}

  The Shapiro-Wilk test indicates that the residuals are approximately
  normally distributed. The \(p\)-value of \(0.1908\) (greater than
  \(0.05\)) suggests no significant deviation from normality.
\end{enumerate}

In conclusion, based on these two tests, the assumptions of
homoscedasticity and normality for the mammal sleep data linear
regression model appear to be satisfied.

\begin{center}\rule{0.5\linewidth}{0.5pt}\end{center}

\end{document}
